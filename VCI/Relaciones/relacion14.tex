\begin{ejer}
	Probar que, para $a\in]0,1[$, se tiene:
	$$ \int_0^{2\pi} \frac{\cos^2(3t)dt}{1+a^2-2a\cos(2t)} = \pi\frac{a^2-a+1}{1-a} $$
\end{ejer}

\begin{sol}

% Esto huele a la circunferencia unidad
$$1+a^2-2a\cos(2 t) = |1-ae^{2it}|^2 = 1+a^2-2aRe(e^{2it}) =  1+a^2-2a\cos(2t)$$
$$ = (1+ae^{2it})(\overline{1-ae^{2it}}()
= (1-ae^{2it})(1-a\overline{e^{2it}}) = (1+ae^{2it})(1-ae^{-2it})$$
$$\gamma(t) = e^{it} \hspace{1cm} \gamma'(t) = ie^{it}$$
Entonces
$$(1+ae^{2it})(1-ae^{-2it}) = (1-az^2)(1-\frac{a}{z^2})$$
por lo que consideramos la función $\frac{z^2}{(1-az^2)(z^2-a)}$, 
como tenemos que multiplicar por $\gamma'(t)$ consideramos $\frac{z}{(1-az^2)(z^2-a)}$, lo que es igual a
$$\frac{e^{it}}{(1-ae^{2it})(e^{2it}-a)} ie^{it}$$

Haciendo el mismo procedimiento con el numerador
$$\cos^2(3t) = \frac{1+\cos(6t)}{2} = \frac{1+Re(e^{i6t})}{2} = Re(\frac{1+e^{i6t}}{2})$$
Así vemos que la función que finalmente tendríamos que considerar es
$$f(z) = \frac{(1+z^6)z}{2(1-az^2)(z^2-a)} \ \ f:\mathbb{C}\backslash A \rightarrow \mathbb{C}, \ \ f\in\mathcal{H}(\mathbb{C}\backslash A) \hspace{1cm} A = \left\{ \pm\frac{1}{\sqrt{a}}, \pm\sqrt{a} \right\}$$

El camino $\gamma:[0,2\pi] \rightarrow \mathbb{C}$, $\gamma(t) = e^{it}$ es nulhomólogo con respecto a $\mathbb{C}$.
Como $A' = \emptyset$ por el teorema de los residuos
$$\int_{\gamma} f(z)dz = 2\pi i \left( Ind_{\gamma}(-\sqrt{a})Res(f(z),-\sqrt{a}) + Res(f(z), \sqrt{a}) \right)$$
\end{sol}

\begin{ejer}
	Probar que, para $n\in\mathbb{N}$, se tiene:
	$$ \int_0^{2\pi} \frac{(1+2\cos(t))^n \cos(nt)}{3+2\cos(t)} dt = \frac{2\pi}{\sqrt{5}} (3-\sqrt{5})^n $$
\end{ejer}


\begin{ejer}
	Probar que, para $n\in\mathbb{N}$, se tiene:
	$$ \int_{0}^{2\pi} e^{\cos(t)} \cos(nt-\sin(t))dt = \frac{2\pi}{n!} $$
\end{ejer}


\begin{ejer}
	Probar que, para cualesquiera $a,b\in\mathbb{R}^+$, se tiene:
	$$ \int_{-\infty}^{\infty} \frac{dx}{(x^2+a^2)(x^2+b^2)^2} = \frac{\pi (a+2b)}{2ab^3(a+b)^2} $$
\end{ejer}
\begin{sol}
Sean $a,b\in\mathbb{R}^+$.
%Método de Hermite:
%$\frac{1}{(x^2+a^2)(x^2+b^2)^2} = \frac{Ax+B}{x^2+a^2} + %\frac{Cx+D}{x^2+b^2} + \frac{d}{dx} \frac{Mx+N}{x^2+b^2}$
En el caso $a\not = b$ tomamos $$f(z) = \frac{1}{(z^2+a^2)(z^2+b^2)^2} \hspace{1cm} f\in\mathcal{H}( \mathbb{C}\backslash\{ \pm ia, \pm ib \} )$$

Tomamos $R>\max\{a,b\}$, consideramos el camino cerrado $\Gamma = [-R,R]+SC(0,R)$ (semicircunferencia recorrida en sentido positivo) y definimos
$$\gamma : [-R,R]\rightarrow \mathbb{C} \hspace{1cm} \gamma(x)=x \hspace{1cm} \gamma '(x) = 1$$
Usamos que $\Gamma$ es nul-homologa con respecto a $\mathbb{C}$
$$\int_{\Gamma}f(z)dz = \int_{-R}^{R} f(x)dx + \int_{SC(0,R)} f(x)dx$$
Por el teorema de los residuos 
$$\int_{\Gamma}f(z)dz = 2\pi i [ Ind_{\Gamma}(ia)Res(f(z),ia) + Ind_{\Gamma}(ib)Res(f(z),ib) ]$$
$$\int_{\Gamma}f(z)dz = 2\pi i [Res(f(z),ia) + Res(f(z),ib) ]$$

Ambos índices son $1$
$$|\int_{SC(0,R)} f(z)dz| \leq l(SC(0,R))\max \{ |f(z)| : z\in SC(0,R) \} \leq \frac{\pi R}{(R^2-a^)(R^2-a^2)^2}$$
que tiende a $0$ cuando $R\rightarrow \infty$
$$|f(z)| = \frac{1}{|z^2+a^2||z^2+b^2|} \leq \frac{1}{(R^2-a^2)(R^2-b^2)^2}$$
si $|z|=R$
$|z^2+a^2| \geq |z|^2 -a^2 = R^2-a^2$

----------

$$Res(f(z),ia) = \lim_{z\rightarrow ia} (z-ia)f(z) = \lim_{z\rightarrow ia} \frac{(z-ia)}{(z^2+a^2)(z^2+b^2)^2} = \frac{1}{(b^2-a^2)^2} \lim_{z\rightarrow ia} \frac{z-ia}{z^2+a^2}$$ 
por l'Hopital es igual a
$$\frac{1}{(b^2-a^2)^2}\frac{1}{2ia}$$

k es el orden del polo $ib$

$Res(f(z),ib) = \frac{1}{(z-1)'} \lim_{z\rightarrow ib} \frac{d^{k-1)}}{dz^{k-1}} ((z-ib)^k f(z)) = \lim_{z\rightarrow ib} \frac{d}{dz} ((z-ib)^2 f(z)) $
$= \lim_{z\rightarrow ib} \frac{d}{dz} \left( \frac{(z-ib)^2}{(z^2+a^2)(z-ib)^2(z+ib)^2}  \right) 
= \lim_{z\rightarrow ib}\frac{2z*(z+ib)^2 + 2(z+ib)(z^2+a^2)}{(z^2+a^2)^2 (z+ib)^4}
= \frac{4b + 2(-a^2+b^2)}{(a^2-b^2)^2 (-ib^3)8}  $

por tanto

$$ \int_{-\infty}^{\infty} \frac{dx}{(x^2+a^2)(x^2+b^2)^2} = 
2\pi i \left( \frac{1}{2ia}\frac{1}{(b^2-a^2)^2} + \frac{4b+2(b^2-a^2)}{(b^2-a^2)(-i)8b^3 4} \right) =
\frac{4b^3-a(4b+2(b^2-a^2))}{4ab^3(b^2-a^2)^2}$$

continuará

\end{sol}


\begin{ejer}
	Probar que, para $a\in\mathbb{R}^+$, se tiene:
	$$ \int_{-\infty}^{\infty} \frac{x^6dx}{(x^4+a^4)^2} = \frac{3\pi\sqrt{2}}{8a} $$
\end{ejer}


\begin{ejer}
	Dado $n\in\mathbb{N}$ con $n>2$, integrar una conveniente función sobre un camino cerrado que recorra la frontera del sector $D(0,R) \cup \{ z\in\mathbb{C}^{\ast} : 0<arg(z)<2\pi/n \}$ con $R\in\mathbb{R}^+$, para probar que:
	$$ \int_{0}^{+\infty} \frac{dx}{1+x^n} = \frac{\pi}{n} cosec(\pi/n) $$
\end{ejer}

\begin{ejer}
	Probar que, para $a,t\in\mathbb{R}^+$, se tiene:
	$$ \int_{-\infty}^{+\infty} \frac{\cos(tx)dx}{(x^2+a^2)^2} = \frac{\pi}{2a^3}(1+at)e^{-at} $$
\end{ejer}

\begin{ejer}
	Probar que: $\int_{-\infty}^{+\infty} \frac{x\sin(\pi x)}{x^2-5x+6} dx = -5\pi $ 
\end{ejer}


\begin{ejer}
	Integrando la función $z\rightarrow\frac{1-e^{2iz}}{z^2}$ sobre un camino cerrado que recorra la frontera de la mitad superior del anillo $A(0;\epsilon,R)$, probar que
	$$ \int_{0}^{+\infty} \frac{\sin^2(x)}{x^2}dx = \frac{\pi}{2} $$
\end{ejer}
\begin{sol}

Usaremos el teorema general de Cauchy
\begin{equation}
	0 = \int_{\Gamma_{\epsilon,R}} f(z)dz = \int_{[-R,-\epsilon]}f(z)dz + \int_{[\epsilon,R]} f(z)dz + \int_{\gamma_R} f(z)dz + \int_{\gamma_{\epsilon}} f(z)dz
	\label{ej9-integrales}
\end{equation}

Sabemos que
$$\int_{[-R,-\epsilon]}f(z)dz = \int_{-R}^{-\epsilon}f(z)dz =  \int_{-R}^{-\epsilon}\frac{1-e^{2ix}}{x^2}dx
\hspace{1cm}\gamma_1 : [-R,-\epsilon] \rightarrow\mathbb{C}, \gamma_1(x)=x$$
$$\int_{[\epsilon,R]} f(z)dz = \int_{\epsilon}^{R} \frac{1-e^{2ix}}{x^2}dx
\hspace{1cm}\gamma_2:[\epsilon,R]\rightarrow\mathbb{C}, \gamma_2(x)=x$$

Para los dos primeros términos usamos
$$\frac{1-(\cos(2x)+i\sin(2x))}{x^2} = 2\frac{\sin^2(x)}{x^2}-i\frac{\sin(2x)}{x^2}$$
$$\int_{[-R,-\epsilon]} f(z)dz + \int_{[\epsilon,R]} f(z)dz = 2\left[ \int_{-R}^{-\epsilon} \frac{\sin^2(x)}{x^2}dx + \int_{\epsilon}^{R} \frac{\sin^2(x)}{x^2}dx \right] + i\left[ -\int_{-R}^{-\epsilon}\frac{\sin(2x)}{x^2}dx - \int_{\epsilon}^{R}\frac{\sin(2x)}{x^2}dx \right]$$
$$ \int_{[-R,-\epsilon]} f(z)dz + \int_{[\epsilon,R]} f(z)dz = 2\left[ \int_{-R}^{-\epsilon} \frac{\sin^2(x)}{x^2}dx + \int_{\epsilon}^{R} \frac{\sin^2(x)}{x^2}dx \right]$$

Resolvemos los dos últimos términos
$$ \int_{\gamma_R} f(z)dz + \int_{\gamma_{\epsilon}} f(z)dz $$
$$ \gamma_R : [0,\pi] \rightarrow\mathbb{R}\hspace{1cm} \gamma_R (t)=Re^{it}\hspace{0.5cm}\gamma_{R'}(t) = iRe^{it} $$

Para la primera parte tomamos módulos:
$$ \left| \int_{\gamma_R} f(z)dz \right| = \left| \int_0^{\pi} \frac{1-e^{2iRe^{it}}}{R^2e^{2it}} iRe^{it} dt \right| \leq \int_{0}^{\pi} \frac{|1|+|e^{2iRe^{it}}|}{R}dt \leq \frac{2\pi}{R} $$
Donde en el último paso hemos usado que
$$ |e^{2iRe^{it}}| = e^{-2R\sin(t)} \leq 1 \hspace{1cm} \forall t\in[0,\pi] $$
De esa forma sabemos que cuando $R\rightarrow\infty$ la integral tiende a $0$.

\


	\textit{Proposición}

	Sea $a\in\mathbb{C}, g\in\mathcal{H}(D(a,r)\backslash\{a\}), r>0$. Si $g$ tiene un polo de orden $1$ en $a$ y $\gamma_{\epsilon} : [c,a] \rightarrow\mathbb{C}$ es un trozo de circunferencia centrado en $a$ con radio $\epsilon<r$ (recorrido en sentido positivo) entonces
	$$ \lim_{h\rightarrow 0}\epsilon\rightarrow 0 \int_{\gamma_{\epsilon}} g(z)dz = (d-c)iRes(g(z),a) $$
\

Utilizando la proposición obtenemos que
$$ \lim_{\epsilon\rightarrow 0} \int_{\gamma_{\epsilon}} f(z)dz = -\lim_{\epsilon\rightarrow 0} \int_{-\gamma_{\epsilon}} f(z)dz = -\pi iRes(f(z),0) $$
Calculamos el residuo
$$ Res(f(z),0) = \lim_{z\rightarrow 0} zf(z) = \lim_{z\rightarrow 0} z \frac{1-e^{2iz}}{z^2} =  \lim_{z\rightarrow 0} \frac{1-e^{2iz}}{z} $$
Usando l'Hopital nos queda $-2i$, por lo que la integral queda:
$$ -\pi iRes(f(z),0) = -2\pi $$

Tomando límites en \ref{ej9-integrales} tenemos
$$ 0=4\int_{0}^{+\infty} \frac{\sin^2(x)}{x^2} dx -2\pi  \implies  \int_{0}^{+\infty} \frac{\sin^2(x)}{x^2}dx = \pi/2 $$
\end{sol}


\begin{ejer}
	Dado $a\in\mathbb{R}$ con $a>1$, integrar la función $z\rightarrow \frac{z}{a-e^{-iz}}$ sobre la poligonal $[-\pi,\pi,\pi+in,-\pi+in,-\pi]$, con $n\in\mathbb{N}$, para probar que
	$$ \int_{-\pi}^{\pi} $$
	%%%%%%%%%%%%%%%%%%%
\end{ejer}


\begin{ejer}
	Integrando una conveniente función compleja a lo largo de la frontera de la mitad superior del anillo $A(0;\epsilon,R)$, probar que, para $\alpha\in]-1,3[$, se tiene:
	$$ \int_{0}^{\infty} \frac{x^{\alpha}dx}{(1+x^2)^2} = \frac{\pi}{4}(1-\alpha)sec(\frac{\pi\alpha}{2}) $$
\end{ejer}


\begin{ejer}
	Probar que, para $\alpha\in ]0,2[$, se tiene:
	$$ \int_{-\infty}^{+\infty} \frac{e^{\alpha x}dx}{1+e^x+e^{2x}} = \int_{0}^{+\infty} \frac{t^{\alpha-1} dt}{1+t+t^2} = \frac{2\pi}{\sqrt{3}}\frac{\sin(\pi(1-\alpha)/3)}{\sin(\pi\alpha)} $$
\end{ejer}


\begin{ejer}
	Integrando la función $z\rightarrow\frac{\log(z+i)}{1+z^2}$ sobre un camino cerrado que recorra la frontera del conjunto $\{ z\in\mathbb{C} : |z|<R, Im(z)>0 \}$, con $R\in\mathbb{R}$ y $R>1$, calcular la integral
	$$ \int_{-\infty}^{+\infty} \frac{\log(1+x^2)}{1+x^2}dx $$
\end{ejer}



\begin{ejer}
	Integrando una conveniente función sobre la poligonal $[-R,R,R+\pi i,-R+\pi i, -R]$, con $R\in\mathbb{R}^+$, calcular la integral
	$$ \int_{-\infty}^{\infty} \frac{\cos(x)dx}{e^x+e^{-x}} $$
\end{ejer}
\begin{sol}


Probamos con la función
$$ f(z) = \frac{e^{iz}}{e^z+e^{-z}} \hspace{2cm} e^{i(R+ti)} = e^{iR}e^{-t} $$
$$ e^z+e^{-z} =0\Longleftrightarrow  e^{2z}+1=0 \Longleftrightarrow e^{2z}=-1 \Longleftrightarrow 2z\in Log(-1) $$
$$ 2z\in\left\{ \ln(|1|) + i(\pi+2k\pi) : k\in\mathbb{Z} \right\} \Longleftrightarrow z\in\left\{ (\pi/2+k\pi)i : k\in\mathbb{Z} \right\} $$
$$ \Omega=\mathbb{C} \hspace{1cm} \Gamma = [ -R,R,R+\pi i, -R+\pi i, -R ] $$
$$ A = \left\{ \left( \pi/2 +k\pi \right)i : k\in\mathbb{Z} \right\} \hspace{1cm} f\in\mathcal{H}(\Omega\backslash A) \hspace{0.5cm} \Gamma \text{ es nulhomólogo con respecto a }\Omega=\mathbb{C} $$
Sabeos que $A\cup\Omega = \emptyset$, luego por el teorema de los residuos sabemos que
$$ \int_{\Gamma} f(z)dz = 2\pi iInd_{\Gamma}(\frac{\pi}{2}i) Res(f(z), \frac{\pi}{2}i) = 2\pi i Res(f(z), \frac{\pi}{2}i)$$
Que es igual a la siguiente expresión (integrales (1), (2), (3) y (4))
$$ \int_{-R}^{R} \frac{e^{ix}}{e^x+e^{-x}} dx + \int_{R}^{R+\pi i} f(z)dz + \int_{[R+\pi i,-R+\pi i]} f(z)dz + \int_{[-R+\pi i, -R]} f(z)dz $$

Vemos $(3)$ y usamos la parametrización $x\rightarrow x+\pi i, x\in[-R,R]$
$$ \int_{[R+\pi i,-R+\pi i]} f(z)dz = \int_{[-R+\pi i,R+\pi i]} f(z)dz = \int_{-R}^{R} \frac{e^{i(x+\pi i)}dx}{e^{x+\pi i}+e^{-(x+\pi i)}} = e^{-\pi} \int_{-R}^{R} \frac{\cos(x)+i\sin(x)}{e^x+e^{-x}}dx $$

Veamos que $\lim{R\rightarrow\infty} \int_{R}^{R+\pi i} f(z)dz$, donde usaremos la parametrización $R+ti, \ t\in[0,\pi]$
$$ \left| \int_{[R,R+\pi i]} \frac{e^{iz}}{e^z+e^{-z}} \right|  = \left| \int_{0}^{\pi} \frac{e^{i(R+ti)}}{e^{R+ti}+e^{-R-ti}} \right| \leq \int_{0}^{\pi} \frac{e^{-t}}{e^R-e^{-R}}dt \leq \frac{\pi}{e^R-e^{-R}} \text{ que tiende a $0$ cuando $R \rightarrow\infty$}$$
donde hemos usado que
$$ | e^{R+ti}+e^{-R-ti} | \geq |e^{R+ti}|-|e^{-R-ti}| = e^R-e^{-R} $$

Tomando límite con $R\rightarrow\infty$ en la expresión
$$ (1)+(2)+(3)+(4) = 2\pi i Res(f(z),\frac{\pi}{2}i)) $$
$$ (1)+(3) = 2\pi iRes(f(z),\frac{\pi}{2}i) $$
Y ahora tomamos las partes reales
$$ \int_{-\infty}^{+\infty} \frac{\cos(x)}{e^x+e^{-x}} dx + e^{-\pi} \int_{-\infty}^{+\infty} \frac{\cos(x)}{e^x+e^{-x}}dx = Re(2\pi iRes(f(z), \frac{\pi}{2}i)) $$
Tras hacer los cálculos vemos que
$$ Res(f(z),\frac{\pi}{2}i) = \frac{e^{-\pi/2}}{2i}  $$

Por lo que 
$$ Re(2\pi iRes(f(z), \frac{\pi}{2}i)) = \pi e^{-\pi/2} $$
y tenemos que
$$ \int_{-\infty}^{+\infty} \frac{\cos(x)}{e^x+e^{-x}} dx = \frac{\pi e^{-\pi/2}}{1+e^{-\pi}} $$
\end{sol}



\begin{ejer}
	Integrando una conveniente función sobre un camino cerrado que recorra la frontera del conjunto $\{ z\in\mathbb{C} : \epsilon < |z| < R, 0<arg(z)<\pi/2 \}$, con $0<\epsilon<1<R$, calcular la integral
	$$ \int_{0}^{+\infty} \frac{\log(x)}{1+x^4}dx $$
\end{ejer}
\begin{sol}
Probamos con la función
$$ f(z) = \frac{\log(z)}{1+z^4} $$
entonces
$$ \int_{[\epsilon, R]}f(z)dz = \int_{\epsilon}^{R} \frac{\log(x)}{1+x^4} $$
$$ \int_{[iR,i\epsilon]} f(z)dz = -\int_{[i\epsilon,iR]} f(z)dz = -\int_{\epsilon}^{R} \frac{\log(ix)}{1+(ix)^4} idx $$
Vemos
$$ \log(ix) = \log(|ix|)+i\pi/2 $$
entonces
$$ -\int_{\epsilon}^{R} \frac{\log(ix)}{1+(ix)^4} idx = -i\int_{\epsilon}^{R} \frac{\log(x)}{1+x^4} + \frac{\pi}{2} \int_{\epsilon}^{R} \frac{dx}{1+x^4} $$
% Consejo: Integrar con complejos

Como \textbf{idea} para las partes que quedan:
$$ |f(z)| = \frac{|\log(|z|)+i\theta_z|}{|1+z^4|} \leq \frac{\pi/2+\log(R)}{R^4-1} \hspace{1cm} z=|z|e^{i\theta_z} \in\gamma_R^{\ast} \text{ tal que }\theta_z\in[0,\pi/2]$$
$$ \left| \int_{\gamma_R}f(z)dz \right| \leq \frac{l(\gamma_R)(\pi/2+\log(R))}{R^4-1} \hspace{1cm}\text{ que tiene a 0 cuando $R\rightarrow\infty$} $$

Si $z\in\gamma_{\epsilon}^{\ast}$
$$ |f(z)| = \frac{|\log(\epsilon)|+\pi/2}{1-\epsilon^4} $$
$$ \left| \int_{\gamma_{\epsilon}} f(z)dz \right| \leq \frac{l(\gamma_{\epsilon}) (|\log(\epsilon)|+\pi/2))}{1-\epsilon^4}  $$

Como tenemos
$$ \Omega = \mathbb{C}^{\ast}\backslash\mathbb{R}^- \text{ que es homologicamente conexo y }\Gamma = [\epsilon,R] +\gamma_R + [iR,i\epsilon] +\gamma_{\epsilon} $$
podemos deducir que
$\Gamma$ es nulhomólogo con respecto a $\Omega$ y que $f\in\mathcal{H}(\Omega\backslash A)$
con
$$ A = \{ e^{i\pi/4}, e^{3\pi i/4}, e^{5\pi i/4}, e^{7\pi i/4} \} $$
Y aplicando el teorema de los residuos tenemos que
$$ \int_{\Gamma}f(z)dz = 2\pi i Ind_{\Gamma}(e^{i\pi/4}) Res(f(z),e^{i\pi/4}) = 2\pi i Res(f(z),e^{i\pi/4}) $$
donde
$$ \lim_{z\rightarrow e^{i\pi/4}} \frac{(z-e^{i\pi/4})}{1+z^4}\log(z) = -\frac{\pi i}{16}e^{i\pi/4} = -\frac{\pi i}{16}\frac{\sqrt{2}}{2} + \frac{\pi}{16}\frac{\sqrt{2}}{2} $$
\end{sol}

\begin{ejer}
	Integrando una conveniente función sobre la poligonal $[-R,R,R+2\pi i,-R+2\pi i,-R]$ con $R\in\mathbb{R}^+$, calcular la integral
	$$ \int_{-\infty}^{+\infty} \frac{e^{x/2}}{e^x+1}dx $$
\end{ejer}
\begin{sol}

Consideramos 
$f(z) = \frac{e^{z/2}}{e^z+1}$
$$e^z+1=0 \Longleftrightarrow e^z = -1 \Longleftrightarrow z\in Log(-1) = \{ 0+i(\pi+2k\pi) : k\in\mathbb{Z} \} = A$$
$$f : \mathbb{C}\backslash A \rightarrow \mathbb{C}\hspace{0.5cm} f\in\mathcal{H}(\mathbb{C}\backslash A)$$
El camino $\Gamma_R  = [-R,R,R+2\pi i, -R+2\pi i]$ es nulhomólogo con respecto a $\mathbb{C}$, además $A$ no tiene puntos de acumulación en $\mathbb{C}$.
Por el teorema de los residuos
$$\int_{\Gamma_R} f(z)dz = 2\pi i Ind_{\Gamma_R}(i\pi) Res(f(z),i\pi)$$
$$\int_{\Gamma_R} f(z)dz = 2\pi i Res(f(z),i\pi) = \int_{[-R,R]} f(z)dz + \int_{[R,R+2\pi i]} f(z)dz + \int_{[R+2\pi i, -R+2\pi i]} f(z)dz + \int_{[-R+2\pi i, -R]} f(z)dz$$
Vemos el segundo término
$$\left| \int_{[R,R+2\pi i]} \right| \leq 2\pi\max\{ |f(z)| : z\in [R,R+2\pi i] \} \leq 2\pi \frac{e^{R/2}}{e^R-1} $$
Que tiende a $0$ cuando $R\rightarrow \infty$
donde hemos usado:
$$|f(z)| = |\frac{e^{z/2}}{e^z+1}| = \frac{e^{R/2} e^{ti/2}}{e^R-1} = \frac{e^{R/2}}{e^R-1} \text{ y } |e^z+1| \geq |e^z|-1 = e^R-1 \hspace{0.5cm}\text{ ya que \ }z=R+ti$$
Vemos el cuarto término
$$\left| \int_{-R+2\pi i, -R} f(z)dz \right| \leq 2\pi \max \{ |f(z)| : z\in[-R+2\pi i, -R] \} \leq 2\pi \frac{e^{-R/2}}{1-e^{-R}} $$ que igualmente tiende a $0$ cuando $R\rightarrow\infty$ y hemos usado que si $z= -R+ti, \ t\in[0,2\pi]$, entonces
$$|f(z)| = \left| \frac{e^{-R/2}e^{ti/2}}{e^{-R}e^{ti}+1} \right| \leq \frac{e^{-R/2}}{1-e^{-R}}$$
El primer término:
$$\int_{[-R,R]} f(z)dz = \int_{-R}^{R} \frac{e^{x/2}}{e^x+1} \gamma'(x)dx \text{ con } \gamma(x)=x \text{ y }x\in[-R,R]$$
Y por último el segundo término:
$$\int_{[R+2\pi i, -R+2\pi i]} f(z)dz = -\int_{[-R+2\pi i, R+2\pi i]} f(z)dz = -\int_{-R}^{R} \frac{e^{x/2}e^{\pi i}}{e^x+1}dz = \int_{[-R,R]} \frac{e^{x/2}}{e^x+1}dx$$
donde hemos usado
$$\varphi (x) = x+2\pi i : x\in[-R,R], \ \varphi'(x) = 1 \hspace{1cm} f(\varphi(x)) = \frac{e^{x/2}e^{\pi i}}{e^{x+2\pi i}+1} = \frac{-e^{x/2}}{e^2+a}$$
Tomando límite con $R\rightarrow\infty$ obtenemos que
$2\int_{-\infty}^{\infty} \frac{e^{x/2}}{e^x+1} dx = 2\pi iRes(f(z),i\pi)$
$$Res(f(z),i\pi) = \lim_{z\rightarrow i\pi} f(z) = \lim_{z\rightarrow i\pi} (z-i\pi) \frac{e^{z/2}}{e^z+1} = e^{i\pi/2} \lim_{z\rightarrow i\pi} \frac{z-i\pi}{e^z+1}$$
y usando l'Hopital nos queda
$-i$.
Conclusión:
$$\int_{-\infty}^{\infty} \frac{e^{x/2}}{e^x+1} dx = \pi iRes(f(z),i\pi) = \pi i(-i) = \pi $$
\end{sol}

% integrales propuestas: 14, 15, 9 (mitad superior del anillo)
