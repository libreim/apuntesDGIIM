\begin{ejer}
	Estudiar la continuidad de la función argumento principal, $\arg : \mathbb{C}^{\ast} \rightarrow \mathbb{R}$.
\end{ejer}

\begin{sol}


Usamos la fórmula de la relación anterior para probar que es continua en $\mathbb{C}^{\ast} \backslash \mathbb{R}$
$$  
arg z = 2\arctan \left(\frac{Imz}{Rez + |z|}\right) \hspace{1cm} \forall z\in\mathbb{C}^{\ast}\backslash\mathbb{R}
$$
Luego nos aproximamos por sucesiones
$$  
\left\{ z+\frac{i}{n} \right\} \rightarrow z \hspace{1cm}
\lim_{n\rightarrow\infty}\left(arg\left(z+\frac{i}{n}\right)\right) = \lim_{n\rightarrow\infty}\left( \arctan\left(\frac{1}{nz}+\pi\right) \right) = \pi
$$
$$
\left\{ z-\frac{i}{n} \right\} \rightarrow z \hspace{1cm}
\lim_{n\rightarrow\infty}\left(arg\left(z-\frac{i}{n}\right)\right) = \lim_{n\rightarrow\infty}\left( \arctan\left(\frac{-1}{nz}-\pi\right) \right) = -\pi
$$
Como los límites no coinciden no existe dicho límite y la función no es continua.
\end{sol}

\begin{ejer}
	Dado $\theta\in\mathbb{R}$, se considera el conjunto $S_{\theta} = \{ z\in\mathbb{C}^{\ast} : \theta\not\in Arg(z) \}$. 
	Probar que existe una función $\varphi\in\mathcal{C}(S_{\theta})$ que verifica $\varphi(z)\in Arg(z)$ para todo $z\in S_{\theta}$
\end{ejer}

\begin{sol}
Definimos
$$
f(z) = z( \cos(\pi-\theta) + i\sin(\pi-\theta) ) \hspace{2cm} f:\mathbb{C}\rightarrow \mathbb{C} \text{ continua}
$$
$$
\phi(z) = arg(f(z)) - (\pi-\theta) \hspace{2cm} \phi: S_{\theta} \rightarrow \mathbb{C} \text{ continua}
$$
$$ 
\phi(z) \in Arg (f(z)) + \theta-\pi \subset Arg (f(z)) + Arg( \cos(\theta-\pi)+i\sin(\theta-\pi) ) 
$$
$$= Arg ( f(z)-\cos(\theta-\pi)+i(\theta-\pi) ) = Argz
$$
\end{sol}

\begin{ejer}
	Probar que no existe ninguna función $\phi\in\mathcal{C}^{\ast}$ tal que $\varphi(z)\in Arg(z)$ para todo $z\in\mathcal{C}^{\ast}$, y que el mismo resultado es cierto, sustituyendo $\mathbb{C}^{\ast}$ por $\mathbb{T}=\{ z\in\mathbb{C} : |z|=1 \}$
\end{ejer}

\begin{sol}
$\mathbb{T}$ compacto y conexo $\implies \phi(\mathbb{T})$ es un intervalo cerrado y acotado

Sea $\alpha\in\mathbb{T} : \phi(\alpha) \not =\min(\phi(\mathbb{T})), \max(\phi(\mathbb{T}))$ 
$\phi ( \mathbb{T}-\{ \alpha \} )$ conexo $\implies [ \min(\phi(\mathbb{T})), \phi(\alpha)[ \cup ]\phi(\alpha), \max( \phi(\mathbb{T}) ) ] $ que es una contradicción.

Si $\exists h:\mathbb{C}^{\ast} \rightarrow \mathbb{R}$ con $h(z)\in Arg(z)$ $\forall z\in\mathbb{C}^{\ast}$ y $h$ continua, entonces
$h_{| \mathbb{T}} : \mathbb{T} \rightarrow \mathbb{R}$ es continua y $h_{|\mathbb{T}} (z) \in Arg(z) \ \forall z\in\mathbb{T}$
\end{sol}

\begin{ejer}
	Probar que la función $Arg : \mathbb{C}^{\ast} \rightarrow \mathbb{R}\backslash \mathbb{R}/2\pi\mathbb{Z}$ es continua, considerando en $\mathbb{R}/2\pi\mathbb{Z}$ la topología cociente. Más concretamente, se trata de probar que, si $\{z_n\}$ es una sucesión de números complejos no nulos, tal que $\{z_n\} \rightarrow z\in\mathbb{C}^{\ast}$ y $\theta\in Arg(z)$, se puede elegir $\theta_n\in Arg(z_n)$ para todo $n\in\mathbb{N}$, de forma que $\{ \theta_n \} \rightarrow \theta$.
\end{ejer}




\begin{sol}


$$
\{ z_n \} \rightarrow z \implies \forall \epsilon>0\ \exists m\in\mathbb{N} : n\geq m \ |z_n-z|z\epsilon
$$
Sea $\epsilon = \frac{|z|}{2}$, entonces 
$\exists m\in\mathbb{N}$ tal que para $n\geq m$ se tiene $z_n\in D(z, |z|/2)$

A partir de aquí usamos el ejercicio $2$ para concluir.
\end{sol}



\begin{ejer}
	Dado $z\in\mathbb{C}$, probar que la sucesión $\left\{ \left( 1+\frac{z}{n} \right)^n \right\}$ es convergente y calcular su límite.
\end{ejer}


\textbf{Idea}
$$
\left|1+\frac{z}{n}\right|^n \rightarrow e^{Re z}
$$
$$
\lim_{n\rightarrow\infty} \left|1+\frac{z}{n}\right|^n 
= e^{ \lim_{n\rightarrow\infty} n\left(\left|1+\frac{z}{n}\right|-1\right) }
$$
$$
arg\left(\left(1+\frac{z}{n}\right)^n\right) \rightarrow Im z
$$

\begin{sol}


Sea $z_n\in\mathbb{C}$ con$\phi_n\in Arg(z_n)$
$$\{ |z_n| \} \rightarrow |z| \hspace{1cm} \{y_n\} \rightarrow y\in Arg(z)$$
Vamos a usar que is tiene sun sucesión de números complejos y la sucesión de los módulos converge y hay una sucesión de los argumentos de forma que convergen, la sucesión converge al módulo por $cos(x) +isen(x)$

Sabemos que
$$\phi_n \in Arg(z_n)$$
Con lo que tenemos

$z_n = \left(1+\frac{z}{n}\right)^n = \left(1+\frac{a+ib}{n}\right)^n
= \left[\left(1+\frac{a}{n}\right) + \left(\frac{ib}{n}\right)\right] ^n$
$|z_n| = \left[\left| \left(1+\frac{a}{n}\right) + \frac{ib}{n} \right|\right] ^n
= \left[ \left(1+\frac{a}{n}\right)^2 + \left(\frac{b}{n}\right)^n \right]^{1/2}
= lim n \left[ \left(1+\frac{a^2}{n^2} + \left(\frac{a}{n}+\frac{b^2}{n^2}\right) \right)\right]^{n/2}
= e^{  \lim n/2 \left(\frac{a^2}{n^2} + 2a/n + b^2/n^2 \right)  } = e^a = e^{Re z}$ 

Vamos a utilizar la fórmula de Moivre.
%$ z_m = (1+z/n)^n$
$$Arg\left(a+ \frac{a+ib}{n}\right)^n = n Arg\left(a+\frac{a+ib}{n}\right) = n*\arctan \left(\frac{b/n}{1+a/n}\right) = n*\arctan\left(\frac{b}{n+a}\right)$$
Llamamos $y=b/(n+a)$ y usamos $\lim_{y\rightarrow0} \frac{\arctan(y)}{y} = 1$:
$$n*\arctan\left(\frac{b}{n+a}\right) = \frac{\arctan\left(\frac{b}{n+a}\right)}{ \frac{n+a}{n}\frac{b}{n+a}\frac{1}{b} } = b$$
\end{sol}