\begin{ejer}
	En cada uno de los siguientes casos, estudiar la derivabilidad de la función $f:\mathbb{C} \rightarrow\mathbb{C}$ definida como se indica:
	\begin{enumerate}[label=(\alph*)]
		\item $f(z)  z(Re(z))^2 \hspace{1cm}\forall z\in\mathbb{C}$
		\item $f(x+iy) = x^3 -y+i\left( y^3 +\frac{x^2}{2} \right) \hspace{1cm} \forall x,y\in\mathbb{R}$
		\item $f(x+iy) = \frac{x^3+iy^3}{x^2+y^2} \hspace{1cm} \forall (x,y)\in\mathbb{R}^2\backslash\{(0,0)\}, f(0)=0$
	\end{enumerate}	
\end{ejer}
\begin{sol}

\textbf{a)}
Cauchy Riemman




\textbf{c)}
$$f(x+iy) = \frac{x^3+iy^3}{x^2+y^2}\hspace{0.5cm} \forall (x,y)\in\mathbb{R}^2 \backslash \{ (0,0) \}\hspace{1cm}\text{ con } f(0)=0$$
$$u(x,y) = Re f = \frac{x^3}{x^2+y^2}, \hspace{1cm} v(x,y) = Imf = \frac{y^3}{x^2+y^2}$$
En $\mathbb{C}^{\ast}$ tenemos:
$$ \frac{\partial u}{\partial x} = \frac{3x^2}{x^2+y^2} + \frac{x^4(-1)2}{(x^2+y^2)^2} \hspace{1cm}
\frac{\partial v}{\partial y} = \frac{3y^2}{x^2+y^2} = \frac{y^4(-1)2}{(x^2+y^2)^2}$$
$$ \frac{\partial u}{\partial y} = \frac{x^3 (-1)2y}{(x^2+y^2)^2} \hspace{1cm}
\frac{\partial v}{\partial x} = \frac{y^3(-1)2x}{(x^2+y^2)^2}$$

Usando las ecuaciones de Cauchy-Riemman de las dos primeras ecuaciones deducimos
$$3x^2(x²+y^2) - 2x^4 = 3y^2(x^2+y^2)-2y^4 \implies x^4-y^4 = 0 \implies |x|=|y|$$
de las últimas dos deducimos
$$2yx^3=-2xy^3 \implies xy(x^2+y^2) = 0 \implies xy=0 \text{ ó } x^2+y^2=0$$
Como conclusión, para que la función sea derivable en un punto $x\in\mathbb{C}$ dicho punto debe cumplir que el módulo de su parte real sea igual que el módulo de su parte imaginaria.

\end{sol}



\begin{ejer}
	Probar que existe una función entera $f$ tal que
	$$Re(f(x+iy)) = x^4-6x^2y^2+y^4 \hspace{1cm}\forall x,y\in\mathbb{R}$$
	Si se exige además que $f(0)=0$, entonces $f$ es única.
\end{ejer}
\begin{sol}
Sabemos que dicha función debe satisfacer las ecuaciones de Cauchy-Riemann, así que tenemos
$$ \frac{\partial Re(f)}{\partial x}(x+iy) = 4x^3-12xy^2 = \frac{\partial Im(f)}{\partial y}(x+iy) $$
Integrando con respecto de la variable $y$ el término central nos queda que
$Im(f) = 4x^3*y - 4xy^3+k$ tal que $k\in\mathbb{R}$, ya que también se cumple la condición 
$$ \frac{\partial Re(f)}{\partial y} (x+iy) = -\frac{\partial Im(f)}{\partial x} (x+iy) $$

Si se exige $f(0)=0$ nos queda una única ecuación como resultado de sustituir $k=0$.
\end{sol}

\begin{ejer}
	Encontrar la condición necesaria y suficiente que deben cumplir $a,b,c\in\mathbb{R}$ para que exista una función entera $f$ tal que
	$$ Re(f(x+iy)) = ax^2+bxy+cy^2\hspace{1cm}\forall x,y\in\mathbb{R} $$
\end{ejer}
Usamos Cauchy-Riemann, deben cumplir la condición $a=-c$.


\begin{ejer}
	Sea $\Omega$ un dominio y $f\in\mathcal{H}(\Omega)$. Supongamos que existen $a,b,c\in\mathbb{R}$ con $a^2+b^2>0$, tales que
	$$ aRe(f(z)) + bIm(f(z)) = c\hspace{1cm} \forall z\in\Omega $$
	Probar que $f$ es constante.
\end{ejer}

\begin{sol}

Definimos las funciones $u$ y $v$: 
$Re(f(x+iy)) = u(x,y), \ Im(f(x+iy)) = v(x,y)$

Los casos extremos son $a=0$ o $b=0$, donde o la imagen o la parte real que quedan serían constantes.
$$au(x,y)+bv(x,y)=c \implies -bv(x,y) = c-au(x,y) \implies v(x,y)= c'-a'u(x,y)$$
Derivamos con respecto a $x$ e $y$ a ambos lados de la última ecuación:
$$\frac{\partial v(x,y)}{\partial x} = -a' \frac{\partial u(x,y)}{\partial x} = -a' \frac{\partial v(x,y)}{\partial y} \hspace{1cm} \frac{\partial v(x,y)}{\partial y} = -a' \frac{\partial u(x,y)}{\partial y} = a' \frac{\partial v(x,y)}{\partial x}$$
Con lo que nos queda
$$\frac{\partial u (x,y)}{\partial x} = \frac{\partial v(x,y)}{\partial y} = -a'\frac{\partial v(x,y)}{\partial x}$$
% Quedaba lo siguiente pero creo que está mal
%$$\frac{\partial u (x,y)}{\partial x} = -a'\frac{\partial v(x,y)}{\partial y} = -a'a'\frac{\partial v(x,y)}{\partial x}$$
por tanto
$$\frac{\partial v(x,y)}{\partial x} = 0 = \frac{\partial v(x,y)}{\partial y} \hspace{1cm}\forall (x,y)\in\Omega$$
Como tenemos que la parte imaginaria es constante dentro del dominio $\Omega$ y $f\in\mathcal{H}(\Omega)$, con lo que deducimos que la función $f$ es constante.
\end{sol}



\begin{ejer}
	Sea $\Omega$ un dominio y $f\in\mathcal{H}(\Omega)$. Probar que si $\overline{f}\in\mathcal{H}(\Omega)$, entonces $f$ es constante.
\end{ejer}


\begin{sol}
	
Un dominio es un abierto conexo no vacío.

La función $g := f+\overline{f} = 2Re(f)$ es holomorfa por ser suma de holomorfas, y su parte imaginaria es constante, por tanto $g$ es constante.

Por tanto $Re(f)$ es constante, como $f\in\mathcal{H}(\Omega)$ y $ \Omega$ es un dominio deducimos que $f$ es constante.
\end{sol}





\begin{ejer}
	Sea $\Omega$ un dominio y $f\in\mathcal{H}(\Omega)$. Sea $\Omega^{\ast} = \{ \overline{z} : z\in\Omega \}$ y $f^{\ast}:\Omega^{\ast} \rightarrow \mathbb{C}$ la función definida por
	$$ f^{\ast}(z) = \overline{ f(\overline{z} ) } \hspace{1cm} \forall z\in\Omega^{\ast}$$
	Probar que $f^{\ast}\in\mathcal{H}(\Omega^{\ast})$.
\end{ejer}

\begin{sol}
	
\begin{comment}
% No sé para qué servía esto
Tenemos que 
$P(z) = a_n z^n + a_{n-1}z^{n-1}+...+a_1z + a_0$, 
$P(\overline{z}) = a_n \overline{z}^n + a_{n-1}\overline{z}^{n-1}+...+a_1\overline{z} + a_0 $

$\overline{ P(\overline{z}) } = \overline{a_n}z^n + \overline{a_{n-1}}z^{n-1}+...+\overline{a_1}z + \overline{a_0}$

$\overline{z}^n = \overline{z} ... \overline{z} = \overline{(z^n)}$

resolvemos
\end{comment}
$$a\in\Omega^{\ast} \Longleftrightarrow \overline{a}\in\Omega$$
$$\lim_{z\rightarrow a} \frac{f^{\ast}(z)-f^{\ast}(a)}{z-a} = \lim_{z\rightarrow a} \frac{\overline{f(\overline{z})} - \overline{f(\overline{a})}}{z-a} =
\lim_{z\rightarrow a} \frac{\overline{\overline{f(\overline{z})} - \overline{f(\overline{a})}}}{z-a} =
\lim_{z\rightarrow a} \overline{\left( \frac{f(\overline{z})-f(\overline{a})}{z-a} \right)} =
 \overline{f'(\overline{a})}$$
\end{sol}




\begin{ejer}
	Probar que la restricción de la función exponencial a un subconjunto abierto no vacío del plano, nunca es una función racional. %(*)
\end{ejer}

\begin{sol}

Suponemos que existe dicha función racional:
$$R(z) = \frac{\sum_{i=0}^k \lambda_i z^i}{\sum_{i=0}^m \mu_i z^i} = \frac{A(z)}{B(z)}$$

Usamos que la derivada de la exponencial es ella misma:
$$R(z) = R'(z) = \frac{A'(z)B(z)-A(z)B'(z)}{B(z)^2} = \frac{A(z)}{B(z)}$$
Entonces
$$A(z) = A'(z) - \frac{A(z)B'(z)}{B(z)} \implies
B(z)A(z) = A'(z)B(z) - B'(z)A(z) \hspace{1cm}\forall z\in D(a,r), a\in\mathbb{C}, r>0$$
Tienes dos polinomios que son iguales en un entorno no vacío de un punto.
%$gr(A(z)) \leq gr(A'(z))$
$$gr(B(z)) + gr(A(z)) \leq \max \{ gr(A'(z)) + gr(B(z)), gr(A(z))+gr(B'(z)) \}$$
Por tanto llegamos a una contradicción, al derivar un polinomio su grado disminuiría, con lo que no puede cumplirse la última desigualdad.
Como nuestra hipótesis no es cierta tenemos que dicha función nunca es una función racional.

\end{sol}