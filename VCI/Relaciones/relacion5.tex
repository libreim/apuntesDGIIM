\subsection{Ejercicio 1}


\textbf{Solución}

$f(0) = f(0+0) = f(0)^2$
$f(0)= 0$ ó $f(0) = 1$

Si $f(0)=0$ la función es constante.
EN el caso $f(0)\not =0$, $f(0)=1$, 
Si $f$ es derivable en $\alpha\in\mathbb{C}$

$\exists \lim_{n\rightarrow 0} \frac{f(\alpha+h)-f(\alpha)}{n}$
vemos que cumple la fórmula de adición 
$\frac{f(\alpha+h)-f(\alpha)}{n} = f(\alpha) \frac{f(n)-f(0)}{n} \Longleftrightarrow f$ es derivable en $0$
Y eso se puede aplicar $\forall z\in\Omega$

Ahora encontramos todas las funciones enteras que cumplan la condición del enunciado.

Sea $z\in\Omega$, $f(z)=0$, $f(z) = e^{wz} : w\in\mathbf{C}$

Sea $f$ tal que $f\in\mathbb{H}(\mathbb{C})$

$ f'(z) = f(z) \lim_{h\rightarrow 0} \frac{f(h)-f(0)}{h}$
$(f(z) e^{-wz})' = f'(z) e^{-w}z + (-w) +ze^{-wz} = f(z)we^{-wz} - f(z)we^{-wz} = 0$
las funciones son la misma salvo una constante
En el punto $0$ las dos funciones valen lo mismo, por lo que la función es la constante $1$, ya que $f(0)=1$


El ejemplo es $f(z) = e^{Re(z)}$


\subsection{Ejercicio 2}

$e^z = e^{Re(z) (\cos(Imz) + i\sin(Imz))}$
$B_V = \{ z\in\mathbb{C} : a\leq Rez \leq b \}$, $a,b\in\mathbb{R}$ con $a<b$
$B_H = \{ z\in\mathbb{C} : a\leq Imz \leq b \}$
si $a\leq Rez \leq b$, $e^a \leq e^{Rez} \leq e^b$
lo que pasa es que se puede mover en toda la circunferencia unidad

cuando la parte imaginaria se puede mover donde quiera le das infinitas vueltas a la circunferencia unidad. Tenemos que

$e^B_V$ e la corona circular de centro $0$ y radios $e^a$ y $e^b$

donde

$a\leq Imz \leq b$

Tenemos que $\exp(B_H)$ es el sector del plano encerrado entre los ángulos $a$ y $b$




\subsection{Ejercicio 4}
\textbf{Enunciado}

Probar que si $\{z_n\}$ y $\{w_n\}$ son sucesiones de números complejos, con $z_n \not = 0$ para todo
$n\in \mathbb{N}$ y $\{ z_n \} \rightarrow 1$, entonces

$$ \{w_n(z_n-1)\} \rightarrow \lambda\in\mathbb{C} \implies \{ z_m^{w_n} \}\rightarrow e^{\lambda} $$

\textbf{Solución}
Como la función exponencial es continua
$\{ w_n (z_n-1) \} \rightarrow \lambda \implies \{ e^{w_n(z_n-1)} \} \rightarrow e^{\lambda}$

$\lim \{ \log(z_n) \} = 0 \implies \lim \{ z_n-\log(z_n) \} = 1$
$\implies \lim \{ w_n(z_n-\log(z_n))-w_n \} = 0$

$\lim \{ z_n^{w_n} = e^{\log(z_n)-w_n} \}= \lim \{ e^{w_n (z_n-1)} \}$
$\Longleftrightarrow \lim\{ \frac{e^{w_n(z_n-1)}}{e^{w_n (\log(z_n))}} = e^{w_n (z_n-\log(z_n))-w_n} \} = 1$

Vemos que
$z_n^{w_n} = e^{w_n \frac{\log(z_n)}{z_n-1} (z_n-1)}$
$= e^{w_n (z_n-1) \frac{\log(z_n)}{z_n-1}}$
Sabemos que $ \frac{\log(z_n)}{z_n-1} \rightarrow 1$ ya que
$\lim_{z\rightarrow 1} \frac{\log(z)-\log(1)}{z-1} = \log'(1) = 1/1$





\subsection{Ejercicio 5}
\textbf{Enunciado}

Estudiar la convergencia puntual, absoluta y uniforme de la serie de funciones
$\sum_{n\geq 0} e^{-nz^2}$

\textbf{Solución}

La serie converge puntualmente si, y sólo si, $| \frac{1}{e^{z^2}} | <1 \Longleftrightarrow 1<|e^{z^2}|  \Longleftrightarrow 0<Rez^2 = (Rez)^2-(Imz)^2 $
$\Longleftrightarrow |Rez| > |Imz|$

donde en la última implicación hemos usado
$e^{z^2} = e^{Rez^2} e^{Imz^2}$

$Rez^2 = (Rez)^2 - (Imz)^2$,
$Rez^2>0 \Longleftrightarrow |Rez| > |Imz|$


Vemos ahora la convergencia uniforme

$A = \{ z\in\mathbb{C} : (Rez)^2>(Imz)^2 \}$
Si $B\subset A$ y satisface que $\inf_{z\in B} [ (Rez)^2-(Imz)^2 ] > 0$, entonces hay convergencia uniforme en $B$




\subsection{Ejercicio 6} % ----
\textbf{Enunciado}

Probar que $a, b, c \in\mathbb{T}$ son vértices de un triángulo equilátero si, y sólo si, $a+b+c = 0$ .


\textbf{Solución}

$\{ a,b,c \} = \{ e^(\lambda - 2/3\pi), e^{\lambda i}, e^{(\lambda + 2/3\pi)i} \}$

$\Rightarrow$

$e^{2/3\pi i} (a+b+c) = e^{2/3\pi i} a + e^{2/3\pi i}b + e^{2/3\pi i}c = a+b+c \implies a+b+c=0$

$\Leftarrow$

$a'=\frac{a}{a} = 1, b'=\frac{b}{a}, c'=\frac{c}{a}$
$b'=e^{\theta i}$, $c' = e^{\gamma i} = -b-1$


$a+b+c = 0 \implies a'+b'+c' = 0 \implies 1+b'+c' = 0 \implies c'=-b'-1$ con $\gamma,\theta\in ]-\pi,\pi[$

De lo que deducimos que 
$(-\cos(\theta)-1)-i\sin(\theta) = (\cos(\gamma)) + i(\sin(\gamma))$

$-\in(\theta) = \sin(\gamma) \implies \theta = -\gamma $
$-\cos(\theta) -1 =  \cos(\gamma)$
por tanto
$\theta = \pm\frac{2\pi}{3} = -\gamma$



\subsection{Ejercicio 7}




\textbf{Solución}

$a\in\Omega$ y vemos que es derivable por la definición
$\lim_{z\rightarrow a} \frac{\phi(z)-\phi(a)}{z-a} \frac{\phi(z)-\phi(a)}{\phi(z)+\phi(a)} = \lim_{z\rightarrow a} \frac{z-a}{z-a} \frac{1}{\phi(z)+\phi(a)} = \frac{1}{2\phi(a)} = \phi '(a)$
donde hemos usado que $\phi(a)\not = 0$




\subsection{Ejercicio 8}
\textbf{Enunciado}

Probar que, para todo $z \in D(0, 1)$ se tiene:

a) $\sum_{n=1}{\infty} \frac{(-1)^{n+1}}{n} z^n = \log(1+z)$


\textbf{Solución}
a)
$\log (1+z) \in\mathbb{H}(D(0,1))$ y $(\log (1+z))' = \frac{1}{1+z}$ $\forall z\in D(0,1)$

$\frac{1}{1+z} = \frac{1}{1-(-z)} = \sum_{n=0}^{\infty} (-1)^n z^n$
por otra parte la serie de potencias 

$\sum_{n\geq 1} \frac{(-1)^{n+1}}{n} z^n$ tiene radio de convergencia $1$
y su suma $\sum_{n=1}^{\infty} \frac{(-1)^{n+1}}{n} z^n = f(z)$ es holomorfa en $D(0,1)$ y su derivada se calcula término a término

$f'(z) = \sum_{n=1}^{\infty} \frac{(-1)^{n+1}}{n} n z^{n-1} = \sum_{n=0}^{\infty} (-1)^{n}z^{n}$

Entonces $f'(z) = g'(z)$ $\forall z\in D(0,1)$, por tanto $f$ y $g$ difieren en una constante.

Como $g(0) = \log(1) = 0 = f(0)$
con lo que tenemos que $f$ y $g$ son iguales en $D(0,1)$



\subsection{Ejercicio 9}


\textbf{Solución}

$f(z) = \log(\frac{1+z}{1-z})$
la función es holomorfa en $\mathbb{C}\backslash \{1\}$
sabemos que $\log \in \mathbb{H}(\mathbb{C}^{\ast}\backslash\mathbb{R}^-)$
y vemos cuando la función logaritmo cae dentro de dicho conjunto

de forma intuitiva
$\frac{1+z}{1-z} \in\mathbb{R}^- \Longleftrightarrow \exists r>0 : z\not=1, \frac{1+z}{1-z} = -r \Longleftrightarrow 1+z = rz-r \Longleftrightarrow z(r-1)=1+r \Longleftrightarrow z = \frac{1+r}{r-1}$

Viendo que
$g(z) = \frac{1+z}{1-z} \in\mathbb{H}(\Omega)$ y $g(\Omega) \subseteq \mathbb{C}^{\ast}\backslash \mathbb{R}^-$
podemos asegurar que $f\in\mathbb{H}(\mathbb{C})$ por composición.

$f'(z) = \frac{ \frac{1-z+(1+z)}{(1-z)^2} }{ \frac{1+z}{1-z} } = \frac{2}{1-z^2}$

Siendo $\xi\in\mathbb{H}(\Omega)$, entonces
$\xi '(z) = \frac{1}{1+z} + \frac{1}{1-z} = \frac{2}{1-z^2}$



\textbf{Pista}
hacer $\frac{2}{1-z^2}$ en serie de potencias





\subsection{Ejercicio 10}
\textbf{Pista}
$z^{\phi} = e^{ \phi \log z}$ con $\Omega,\Omega_{\phi} \subset \mathbb{C}^{\ast}\backslash \mathbb{R^-}$

$f^{-1}(z) = \phi^{\frac{1}{\phi\log z}} = z^{1/{phi}}$


\subsection{Ejericicio 12}
\textbf{Pista}

$\sin(nz) = \frac{e^{inz}-e^{-inz}}{2i}$, entonces
$\frac{1}{2i} \sum_{n\geq 0} \frac{e^{inz}-e^{-inz}}{2^n}$
Tenemos que ver cuando, para un $z\in\mathbb{C}$ fijo, estudiar la convergencia de las series
$\sum_{n\geq 0} \frac{e^{inz}}{2^n}$ y $\sum_{n\geq 0} \frac{e^{-inz}}{2^n}$

$e^{inz} = e^{in(Rez+iImz)} = e^{-nImz+inRez} = e^{-nImz}e^{inRez}$ con $|e^{inRez}|=1$

$|\frac{e^{inz}}{2^n}| = \frac{e^{-nImz}}{2^n} = (\frac{e^{-Imz}}{2})^n$
entonces

$\sum_{n\geq 0} |\frac{e^{inz}}{2^n}|$ converge $\Longleftrightarrow e^{-Imz} < 2 \Longleftrightarrow -Imz < \ln 2 \Longleftrightarrow -\ln 2 < Imz$

y tenemos convergencia uniforme en $B\subset A= \{ z\in\mathbb{C} : |Imz|<\ln 2 \}$ tal que $\sup_{z\in B} |Imz| < \ln 2$