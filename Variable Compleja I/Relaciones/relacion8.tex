\subsection{Ejercicio 1}

\textbf{Solución}

$\phi (w,z) = \frac{\varphi}{w-z}$, 

$\phi : \gamma^{\ast}x\mathbb{C}\backslash \gamma^{\ast} \rightarrow \mathbb{C}$

$\phi$ es continua

$\phi_w(z) = \phi(w,z)$

$z\rightarrow \phi_w(z)$

$\phi_w \in \mathcal{H}(\mathbb{C}\backslash \{w\})$, $\phi_w \in \mathcal{H}(\mathbb{C} \backslash \gamma^{\ast}  )$

Por tanto  por el teorema de holomorfía de una integral dependiente de $1$ parámetro $f\in\mathcal{H}(\mathbb{C}\backslash \gamma^{\ast})$

Podemos proceder también de otra forma:


Fijo $a\in\mathbb{C}\backslash \gamma^{\ast}$
$\lim_{z\rightarrow a} \frac{f(z)-f(a)}{z-a}$
$= \lim_{z\rightarrow a}  \frac{ \int_{\gamma} \frac{\varphi (w)}{w-z}dw  -  \int_{\gamma} \frac{\varphi (w)}{w-a}dw }{z-a}$

$= \lim_{z\rightarrow a} \int_{\gamma} \varphi \frac{(w-a)-(w-z)}{(z-a)(w-z)(w-a)} dw$
$= \lim_{z\rightarrow a} \int_{\gamma} \varphi (w)\frac{dw}{(w-z)(w-a)}$

Sabemos que podemos intercambiar límite e integral

Nos podemos restringir a un compacto para ver $\psi : \phi_{| \gamma{\ast}  \times  \overline{D}(a,r)} \rightarrow \mathbb{C}$  es continua con $r$ suficientemente pequeño y $\psi (w,z) = \frac{\varphi (w)}{(w-z)(w-a)}$

tenemos que $\gamma^{\ast} \times \overline{D}(a,r)$ es compacto, por tanto $\phi$ es uniformemente continua.

La fórmula para las derivadas la podemos obtener por inducción:

$k=1$ lo hemos hecho, supuesto cierto para $k$, vemos para $k+1$



%$\frac{\phi (w)}{(w-z)^{k+1}} - \frac{\phi (w)}{(w-a)^{k+1}} = %\phi (w) = \left[ \frac{(w-a)^{k+1} - %(w-z)^{k+1}}{(w-z)^{k+1}(w-a)^{k+1}} \right]$






\subsection{Ejercicio 2}

\textbf{Solución}

$(1+z)^{\alpha} = e^{ \alpha \log(1+z) } \in\mathcal{H}(D(0,1))$ por tanto es analítica en ese disco.

Por el Teorema de desarrollo de Taylor
$\sum_{n\geq 0} \frac{f^{n)}(0)}{n!}z^n$


\subsection{Ejercicio 3}

\textbf{Solución}

\subsection{a} 
$f'(z) = \frac{2z-3}{z^2-3z+2}$

$\frac{1}{z^2-3z+2} = \frac{A}{2-z} + \frac{B}{1-z}$

Por tanto el desarrollo en serie de potencias de la original se puede conseguir a partir de desarrollo de las dos partes por separado.

$f'(z) = (2z-3) \sum_{n\geq 0} (\frac{A}{2} (\frac{z}{2})^n + Bz^n) = (2z-3)\sum_{n\geq 0} (\frac{A}{2^{n+1}} + B)z^n $

$= \sum_{n\geq 0} (\frac{A}{2^{n}} + 2B)z^{n+1} - \sum_{n\geq 0} 3 (\frac{A}{2^{n+1}}+B)z^n$

que nos da como resultado 

$$ f'(z) = -\left(\frac{3A}{2}+B\right) + \sum_{n\geq 1} \left(\frac{A}{2^{n+1}} - B \right)z^n $$