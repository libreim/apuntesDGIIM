    %----------------------------
    %   Introducción.
    %----------------------------

    \section{Relación 1}
    \subsection{Ejercicio 2}

    Sea la ecuación:
    \[
        x _{n+1} =  1/3 x_n +2^n
    \]
    Hallar una solución del tipo $x_n=c2^n$.

    Llo primero es ver que esta ecuación es una ecuación en diferencias de primer orden lineal no homogénea. Vamos a hallar el $c$ que nos piden. Entonces:
    \[
        c2^{n+1}= \dfrac{1}{3}c2^n+2^n  \implies 2^n[2c-c\dfrac{1}{3}-1] = 0
    \]
    \[
        2c-\dfrac{1}{3}c-1 = 0 \implies c = \dfrac{3}{5}
    \]
    Por tanto, la solución es: $x_n' =\dfrac{3}{5}2^n $, pero esta es una solución particular.

    Ahora, probaremos que $x_n$ es solución de la ecuación inicial $\iff z_n = x_n - x_n'$ donde $x_n'$ es una solución particular, es solución de:
    $z _{n+1} = \dfrac{1}{3} z_n$.

    Lo probamos:
    Que $x_n$ es solución de la inicial implica que $x _{n+1} =  \dfrac{1}{3} x_n +2^n$.
    Que $x_n'$ es solución de la inicial implica que $x _{n+1}' = \dfrac{1}{3}x_n' 2^n$.

    Si restamos estas dos obtenemos que:
    \[
        x _{n+1} - x _{n+1}' = \dfrac{1}{3}(x_n- x_n')
    \]
    Lo que implica que $z _{n+1} = x _{n+1} - x _{n+1}'$ y $z_n =(x_n- x_n') $

    Ahora, continuamos la resolución del ejercicio:
    \begin{enumerate}
        \item Encontrar una solución particular de la ecuación inicial, $x_n' = \dfrac{3}{5}2^n$
        \item Resolvemos la ecuación homogénea asociada
            \[
                z _{n+1} = \dfrac{1}{3}z_n  \implies z_n = K(\dfrac{1}{3})^n
            \]
        \item Usando el resultado:
            \[
                x_n = x_n' + z_n = \dfrac{3}{5}2^n+K(\dfrac{1}{3})^n
            \]
        \item Aplicamos la condición inicial $x_0 = 1$.


    \end{enumerate}
    Así, la solución es $x_n = \dfrac{3}{5}2^n +\dfrac{2}{5}(\dfrac{1}{3})^n$ con $n\geq 0$

    \subsection{Ejercicio 3}
    Primero calculamos $KerL$.
    $$ x \in KerL \Leftrightarrow L(x) = 0$$
    $$x_n^{*} = 0 \Leftrightarrow x_{n+1} - (2 + i ) x_n = 0 \Leftrightarrow x_{n+1} = (2+i)x_n$$
    Solución general: $$ x_n = K (2+i)^{n} \ , \ K \in C$$
    Por lo tanto los elementos de $X$ que pertenecen al $KerL$ son de la forma que acabamos de indicar. Veamos ahora los elementos de $X$ tales que $L(x) = b$ siendo b = \{ 1,1,1,1.. \}.
    $$x_{n+1} -(2+i)x_n = 1 \Leftrightarrow x_{n+1} = (2+i)x_n + 1$$

    \[
        \begin{rcases}
            x _{n+1} = (2+i) x_n + 1 \\
            x _{n+1}^* = (2+i) x_n^* +  1
        \end{rcases}
    \]
    Restamos y tenemos que:
    $$ x _{n+1}  - x _{n+1}^* = (2 + i) (x_n - x_n^*)$$
    LLamamos $y_n = x_n^* - x_n$, y tenemos que $ y_{n+1} = (2+i) y_n $. Por lo tanto la solucion general de $y_n$ es: $y_n = K (2+i)^n \ , \ k \in C $.
    Ahora calculamos la solucion constante $x^*$.
    $$ x^* = (2+i)x^* +1  \Leftrightarrow x^* (1-2-i) = 1 \Leftrightarrow x^* = \frac{1}{-1-i} \Leftrightarrow x^* = \frac{1(-1+i)}{(-1-i)(-1+i)} \Leftrightarrow x^* = \frac{-1+i}{2} $$
    Y ya podemos calcular la solución general de $x_n$ que es:
    $$ x_n = y_n + x^* = K (2+i)^n + \frac{-1+i}{2}$$
    Es decir los elementos de $X$ que cumplen que $L(x) = b$ son de esa forma.

    \subsection{Ejercicio 4}
    Llamaremos primero $x_n$ el número de clientes de $Paga^+$ en el año n.
    Llamaremos $y_n$ el número de clientes de $Paga^-$ en el año n.
    Ahora, tenemos que:
    \[
        \begin{rcases}
            x _{n+1} =0.5 x_n + 0.25 y_n \\
            y _{n+1} = 0.75 y_n +  0.5 x_n
        \end{rcases}
    \]
    Sabemos que $x_n+y_n = 1$, que es el $100\%$ de los clientes. Podemos despejar así una en función de la otra y nos queda:
    \[
        x _{n+1} = 0.5 x_n + 0.25(1-x_n)
    \]
    Esta es una ecuación en diferencias de primer orden lineal no homogénea. Despejando, tenemos:
    \[
        x _{n+1} = 0.25 x_n +0.25 \implies x_n =  x^* + z_n \implies x_n =  \dfrac{0.25}{1-0.25} + K(0.25)^n
    \]
    Ahora, tenemos que tomar el límite pues nos piden el mercado a largo plazo. Esto es:
    \[
        \lim_{n\to \infty}x_n = \dfrac{0.25}{0.75} = \dfrac{1}{3}
    \]
    Podemos afirmar ahora que, asintóticamente , $\dfrac{1}{3}$ de los clientes estarán en $Paga^+$ y $\dfrac{2}{3}$ en $Paga^-$

    \subsection{Ejercicio 6}
    Vamos a llamar $A_n$ al número de empleados en el departamento $A$ en el año $n$. Del mismo modo, llamaremos $B_n$ a los del departamento $B$ en el año $n$. Si $0 \leq p \leq 1$  y $0 \leq q \leq 1$, tenemos que:
    \[
        \begin{rcases}
            A _{n+1} = (1-p)A_n + qB_n\\
            B _{n+1}= pA_n + (1-q)B_n
        \end{rcases}
    \]
    Y tenemos ahora que $A_n + B_n = M$, (aunque se podría tomar como 1) por lo que el sistema lo reducimos a una ecuación teniendo:
    \[
        A _{n+1} = (1-p)A_n +q(M-A_n)\implies A _{n+1} =(1-p-1)A_n +qM
    \]
    Teniendo de nuevo una ecuación en diferencias de primer orden lineal no homogénea. Así, su solución es:
    \[
        A_n = \dfrac{qM}{1-(1-p-q)} + K(1-p-q)^n
    \]
    Estudiamos primero la solución constante.
    \[
        A_* = \dfrac{qM}{1-(1-p-q)}= \dfrac{qM}{p+q}
    \]
    Impondremos que $p+q\ne 0$, al menos 1 trabajador cambiará de departamento.
    Nuestro término general es:
    \[
        A_n = \dfrac{qM}{p+q} + K(1-p-q)^n
    \]
    Estudiaremos esto a largo plazo. Tenemos que $-1 \leq 1-p-q < 1$. Tenemos que $|1-p-q| < 1$, por lo que:
    \[ \begin{cases}
            \lim_{n \to \infty} A_n = \dfrac{qM}{p+q} = M \dfrac{q}{p+q}\\
            \lim_{n \to \infty}B_n = \lim_{n\to \infty}(M-A_n) = M- M \dfrac{q}{p+q} = \dfrac{p}{p+q}M
        \end{cases}
    \]

    \subsection{Ejercicio 8}
    \begin{enumerate}
        \item $X_n$ es el número de árboles en el año n.
            \[
                x _{n+1} = 0.9 x_n +K
            \]
            Esto tiende a la solución constante $x_* = \dfrac{K}{0.1}$
    \end{enumerate}

    \subsection{Ejercicio 12}
    La ecuación logística de Pielou es una ecuación en diferencias no lineal de la forma:
    \[
        x_{n+1}= \frac{\alpha x_n}{1+ \beta x_n} \quad \quad \alpha > 1 \quad \beta > 0
    \]

    \begin{enumerate}
        \item Demuestre que posee un punto de equilibrio positivo.
            Tenemos que buscar un $\bar{\alpha}\in \R$ tal que $\bar{\alpha} = f(\bar{\alpha})$ donde  $f(x)= \frac{a x}{1+bx}$. Por tanto, buscamos la solución de:
            \[
                \alpha = \frac{a x}{1+bx} \implies b\alpha^2 - a\alpha + \alpha = 0 \implies \alpha[b\alpha +(1-a)]=0 \implies \alpha = 0
            \]
            Es un punto de equilibrio, así que tenemos que:
            \[
                b\alpha +(1-a) = 0 \implies \alpha =  \frac{a-1}{b} > 0
            \]

        \item Tomando $\alpha = 2$ y $\beta = 1$, probar que el punto de equilibrio es Asintóticamente Estable. Para ello , trabajaremos ahora en $\R^+$.
            Tenemos que $f(x) = \dfrac{2x}{1+x} \in \mathcal{C}^\infty$ y que su derivada es:
            \[
                f'(x) = \frac{2(1+x)-2x}{(1+x)^2} = \frac{2}{(1+x)^2}\implies |f'(1)|= \frac{1}{2} < 1
            \]
            Por lo que sí es A.E.

        \item Demostrar que el cambio de variable $x_n = \dfrac{1}{z_n}$ trasnforma esa ecuación en una ecuación lineal del primer orden.
            Tenemos ahora por tanto:
            \[
                \frac{1}{z_{n+1}} =  \frac{a \dfrac{1}{z_n}}{1+b \dfrac{1}{z_n}} \implies z_{n+1}= \dfrac{1}{a}z_n + \dfrac{b}{a} \quad (2)
            \]
            Que es una ecuación lineal de primer orden

        \item A partir del resultado anterior, determinar el comportamiento asintótico de las soluciones de la ecuación logística.

            Esta es una ecuación en diferencias lineal no homogénea, por tanto tendremos una solución $z_n = z_* +y_n$ donde $z_*$ es la solución constante y $y_n$ es la solución de la homogénea asociada.

            Tenemos que $z_*=\frac{b/a}{1- (1/a)} = \frac{b}{a-1}> 0$. Ahora, calculamos la ecuación homogénea asociada a $(2)$, calculado en el apartado anterior:
            \[
                y_n = \mathcal{C} + (\dfrac{1}{a})^n
            \]
            Ya tenemos la solución, por tanto la solucion de $z_n$ es:
            \[
                z_n =  \dfrac{b}{a-1} + \mathcal{C}(\dfrac{1}{a})^n \quad n \geq 0
            \]
            El comportamiento asintótico de esta es convergente a $\dfrac{b}{a-1}$ pues , como $a > 1$, entonces $(\frac{1}{a})^n \to 0$

            Ahora, volviendo al cambio de variable, como $z_n$ tiene límite distinto de cero, entonces:
            \[
                \lim_{n\to \infty} x_n = \frac{1}{\lim_{n\to \infty}z_n} = \frac{1}{b/(a-1)}= \frac{a-1}{b}
            \]
            Que es el punto de equilibrio de $x_n$, luego $\alpha= \frac{a-1}{b}$ es Asintóticamente Estable.

    \end{enumerate}

    \subsection{Ejercicio 15}

    \begin{enumerate}
        \item Hecho en clase
        \item Para conseguir un equilibrio poblacional asintóticamente estable (a.e.), se propone vender una fracción $\alpha$$(0 < \alpha < 1)$ de la población en cada periodo de tiempo dando lugar al modelo:
            \[
                p_{n+1} = 10(1 - \alpha)p_n e^{(1-\alpha)p_n}
            \]

            i) Encuentre el intervalo abierto (de amplitud máxima) donde elegir $\alpha$ para que esté asegurada la estabilidad
            asintótica del equilibrio positivo.


            Si :
            \[
                \mu = \frac{ln(10(1-\alpha)}{(1-\alpha)}
            \]
            Vamos a comprobar que es L.A.E.
            \[
                f'(x) = 10(1-\alpha)e^{-(1-\alpha)x}(1+x(-(1-\alpha))
            \]
            \[
                f'(\mu) 0 1-ln(10(1-\alpha)) \quad y \quad 0 < ln(10(1-\alpha))< 2
            \]
            Y ahora, si:
            \begin{itemize}
                \item $\alpha < 9/10$
                \item $\alpha > \frac{10-e^2}{10}$
            \end{itemize}
            Por tanto $\mu$ es L.A.E si $\frac{10-e^2}{10} < \alpha < 9/10$

            ii) Calcule el valor de $\alpha$ para el que la población de equilibrio alcanza su valor máximo y es a.e.

            Ahora, si $g(x) = \frac{ln(10(1-x))}{1-x}\implies g'(x)= \frac{-1+ln(10(1-x))}{(1-x)^2}$.

            Ahora, $g'(x) = 0\implies 1-ln(10(1-x)) = 0 \implies x= \frac{10-e}{10}$ es donde la poblaciñon alcanza su máximo.

            Ahora, si comprobamos: $x_0 > \mu \implies x_1 = 10(1-\alpha)x_0 e^{-(1-\alpha)x_0} = x_0 \frac{10(1-\alpha)}{e^{(1-\alpha)x_0}} < x_0 \implies x _{n+1} < x_n$. Y Si miramos ahora si $x_0 < \mu$, veremos que $x _{n+1} > x_n$
    \end{enumerate}

    \subsection{Ejercicio 17}
    En cierto mercado, los precios de determinado producto siguen una din\'amica basada en los postulados del modelo de la telaraña pero se ha observado que las funciones de oferta y demanda vienen dadas por:

    \[
        O(p) = 1 + p^{2}, \ D(p) = c - dp
    \]\

    donde c \textgreater 1 y d \textgreater 0. Suponemos, adem\'as, que el equilibrio del mercado se da si la Oferta iguala a la Demanda y que la oferta en el periodo (n+1)-\'esimo depende del precio del periodo n-\'esimo.

    a) Deduzca la ED, $p_{n+1} = F(p_n)$, que describe la din\'amica planteada y calcule el precio de mercado $p^{*}$ (punto de equilibrio econ\'omicamente factible).

    b) Deduzca las condiciones sobre c y d que aseguran la estabilidad asit\'otica de $p^{*}$. ¿Qu\'e ocurre si d = 2 y c = 4 ?

    c) Para c = 3 y d = 2, use el diagrama de Cobweb para trazar los valores $p_1$ y $p_2$ a partir de $p_0$ = 1. A largo plazo, ¿c\'omo se comportar\'an los precios en el caso contemplado?

    a)

    Deducimos la ED, sabemos que el equilibrio del mercado se da si la oferta es igual a la demanda por lo que :

    \[
        O(p_{n-1}) = D (p_n) \Rightarrow
    \]

    \[
        \Rightarrow 1 + p_{n-1}^{2} = c - dp_n \Rightarrow
    \]

    \[
        \Rightarrow p_n = \frac{c-1}{d} - \frac{p_{n-1}^{2}}{d} \Rightarrow
    \]

    puesto que $d>0$

    \[
        \Rightarrow p_{n+1} = \frac{c-1}{d} - \frac{p_{n}^{2}}{d} = F(p_n)
    \]

    Para calcular el punto de equilibrio ($\alpha$) tenemos que resolver la ecuaci\'on

    \[
        \alpha = F(\alpha) \Rightarrow
    \]

    \[
        \Rightarrow F(\alpha) = \frac{c-1}{d} - \frac{\alpha^{2}}{d} = \alpha \Rightarrow
    \]

    \[
        \Rightarrow \alpha^{2} + d\alpha -c+1 = 0 \Rightarrow
    \]

    \[
        \Rightarrow \alpha = \frac{-d\pm\sqrt{d^{2} - 4(-c+1)}}{2} = \frac{-d\pm\sqrt{d^{2} +4c-4}}{2}
    \]

    Dado que $\alpha$ es el precio de equilibrio y \'este es un n\'umero real entonces se tiene que dar que el discriminante sea mayor que 0 lo cual es cierto ya que c$>$1 y d$>$0. Adem\'as se tiene que el precio de equilibrio es un n\'umero positivo así que debemos escoger aquellos $\alpha$s que sean positivos. As\'i se verifica que :

    \[
        d^{2} +4c-4 > d^{2} \Rightarrow
    \]

    \[
        -d + \sqrt{d^2 +4(c-1)} > -d + \sqrt{d^{2}} = 0
    \]

    Se puede ver que el precio de equilibrio que buscamos es:

    \[
        \alpha = \frac{-d+\sqrt{d^{2} +4c-4}}{2}
    \]

    b)
    Recordemos que para que un punto de equilibrio sea localmente asintóticamente estable (L.A.E.) se tienen que dar dos condiciones: la primera es que la funci\'on sea de clase 1 y segundo que el valor absoluto de la derivada evaluada en el punto de equilibrio sea menor que 1. Como podemos ver $F \in C^{\infty}$ luego, en particular, $F \in C^{1}$, basta con ver para que valores de c y d se tiene que la mencionada derivada sea menor que 1.

    \[
        F(x) = \frac{c-1}{d} - \frac{1}{d}x^{2} \Rightarrow
    \]

    \[
        \Rightarrow |F'(x)| = |\frac{-2}{d}x|
    \]

    \[
        \alpha = \frac{-d+\sqrt{d^{2} +4c-4}}{2} \Rightarrow
    \]

    \[
        \Rightarrow |F'(\alpha)| = |\frac{d-\sqrt{d^{2}+4c-4}}{d}| = | 1 - \frac {\sqrt{d^{2}+4c-4}}{d}| < 1
    \]

    \[
        \Rightarrow -1 <  1 - \frac {\sqrt{d^{2}+4c-4}}{d} < 1 \Rightarrow
    \]

    \[
        \Rightarrow -2 < - \frac {\sqrt{d^{2}+4c-4}}{d} < 0 \Rightarrow
    \]

    \[
        \Rightarrow 2 > \frac {\sqrt{d^{2}+4c-4}}{d} > 0 \Rightarrow
    \]

    \[
        \Rightarrow 2d > \sqrt{d^{2}+4c-4} > 0 \Rightarrow
    \]

    \[
        \Rightarrow 4d^{2} > d^{2}+4c-4 \Rightarrow
    \]

    \[
        \Rightarrow 3d^{2}-4c+4 > 0
    \]

    Si d = 2 y c = 4

    \[
        \Rightarrow 3*4 - 4*4 +4 = 0
    \]

    Luego la condici\'on anterior no nos garantiza que ese punto sea L.A.E.
    \

    \

    \

    Sustituyendo en $F(x)$:

    \[
        F(x) = \frac{4-1}{2} - \frac{1}{2}x^{2} = \frac{3}{2} - \frac{1}{2}x^{2}\Rightarrow
    \]

    \[
        |F(x)|' = |-x| = x
    \]

    Sustituyendo $\alpha$ por x:

    \[
        x = \alpha = \frac{-2+\sqrt{2^{2} +4*4-4}}{2} = 1
    \]

    El resultado como podemos ver es 1, lo cual no nos dice que sea L.A.E. ni inestable. Por tanto tendremos que comprobar que ocurre con la segunda derivada (si es distinta o mayor que 0 entonces es inestable y si es menor que 0 es L.A.E.) :

    \[
        F(x)' = -x \Rightarrow
    \]

    \[
        \Rightarrow F(x)'' = -1 \neq 0
    \]

    \[
        \Rightarrow F(\alpha)'' = -1
    \]

    Por tanto, es inestable.

    c) Si c = 3 , d = 2 y $p_0 = 1$ , entonces usando que :

    \[
        p_{n+1} = \frac{c-1}{d} - \frac{p_{n}^{2}}{d} = F(p_n) \Rightarrow
    \]

    \[
        \Rightarrow p_{n+1} = 1 - \frac{1}{2}p_n^{2} = F(p_n)
    \]

    \[
        \Rightarrow p_1 = 1 - \frac{p_0^{2}}{2} = \frac{1}{2}
    \]

    \[
        \Rightarrow p_2 = 1 - \frac{p_1^{2}}{2} = \frac{7}{8}
    \]

    %\includegraphics[scale=0.32]{/home/usuario/Escritorio/diagrama(ej17).png}
    \

    Podemos ver que el punto de equilibrio es un atractor y que la funci\'on converge a dicho punto:
    \[
        \alpha = \frac{-d+\sqrt{d^{2} +4c-4}}{2} = \frac{-2+\sqrt{4 +12-4}}{2} = \frac{-2+2\sqrt{3}}{2} = -1+\sqrt{3}
    \]

    \

    Para realizar este gr\'afico he usado wxmaxima, en concreto, la orden

    \[
        staircase(1-(1/2)*x^2,1,100,[x,0,1.5],[y,0,1.5]);
    \]

    donde el primer parámetro es la función obtenida antes $(F_n)$ representada por al curva roja, el segundo parámetro es el punto inicial, el tercero es el número de iteraciones o valores que calculamos (n=100) y los últimos parámetros son opciones para representar el gráfico.

    \subsection{Ejercicio 19}

    Determine razonadamente la veracidad o falsedad de las siguientes afirmaciones.
    a) Sea la ecuación en diferencias de primer orden, $x _{n+1} = f (x_n )$, donde $f(x)$ es suficientemente derivable en
    un entorno del punto de equilibrio, s. Si $f'(s) = 1 $y $f'' (s) = 0$, entonces el punto de equilibrio siempre será
    inestable.

    El punto de equilibrio es siempre inestable


    b) Toda ecuación en diferencias $x _{n+1} = f (x_n)$, con un punto de equilibrio inestable en $x = s$ admite, al menos,
    un 2-ciclo o solución 2-periódica.

    Falso. Basta tomar la función $f(x) = ax+b$ tal que si $a>1$, entonces la derivada es mayor que 1 y la función no tiene ningún ciclo.

    c) Todas las soluciones no constantes de la ecuación en diferencias lineal $x_{n+2}+ x_{n-1} = 0$, son 3-periódicas.

    Si despejamos la ecuación, vemos que:
    \[
        x_{n+3} = -x_n
    \]
    Dados 3 datos iniciales, $x_0,x_1,x_2$, podemos escribir algunos términos de la recurrencia en función de los anteriores y escribimos su órbita, quedando:
    \[
        \{x_0,x_1,x_2,-x_0,-x_1,-x_2,x_0,\cdots\}
    \]

    Por lo que el periodo es 6, no 3 y por tanto es FALSO.

    d ) La dinámica de cierta población ovina se rige mediante un modelo logístico discreto. Si la tasa de crecimiento
    para una población de 10 cientos es de $1.5$ y para una de población de 20 cientos es de $1$; entonces, a largo plazo, la población tiende a un valor constante.

    Tenemos que $P_n=10$ y la tasa de $P_n$ es 1.5 . También que si $P_m = 20$, entonces su tasa es de 1.

    Ajustando por un modelo conveniente, tenemos que:
    \[
        tasa(P_n) = \frac{P _{n+1} - P_n}{P_n} = a -bP_n
    \]
    y si despejamos, obtenemos:
    \[
        P _{n+1} = P_n + P_n -b P_n^2 \implies P _{n+1} = (1+a)P_n(1-\frac{b}{1+a}P_n)
    \]
    ASí que si tomamos $\mu = (1+a)$, $x_n = \frac{b}{1+a}P_n$, entonces:
    \[
        x _{n+1} = \mu x_n (1-x_n)
    \]
    Ahora, debemos hallar a y b. planteamos el sistema:
    \[
        \begin{rcases}
            1.5 = a-10b\\
            1 = a-20b
        \end{rcases}
        \begin{cases}
            a = 2 \\
            b = 0.05
        \end{cases}
    \]
    y planteamos la ecuación:
    \[
        P _{n+1} = 3P_n - 0,05 P_n^2
    \]
    entonces:
    \[
        P_* = 3P_* - 0,05P_*^2
    \]
    Si igualamos $0,05P_*^2 -2P_* = 0 \implies P_* = 0$, pero también tenemos que tener en cuenta el caso $0,05P_* - 2 = 0 \implies P_* = 40$

    Ahora, si $f(x) = 3x - 0,05x^2$, entonces $f'(0) = 3$, por lo que es inestable y $f'(40) =3-4 = -1$ y $f''(40) = -0,05$
    pero como la derivada vale $-1$, no podemos determinar con claridad lo que ocurre en este caso. Sin embargo, si realizamos el dibujo en $wxmaxima$, veríamos que se realiza la tela de araña y así converge.



    %%%%%%%%%%%%%%%%%%%%%%%%%%%%%%%%%%%%%%%%%%%%%%%%%%%%%%%%%%%%%%%%%%%%%%%%%%%%%%%%%%%%%%%%%%%%%%%%%%%%%%%%%%%5

    \section{Relación 2}

    \subsection{Ejercicicio 2}
    Calcule una solución de la ecuación $x_ {n+3} -x_n = 0$ con condiciones iniciales $x_0 = 1$, $x_1 = 0$ y $x_2 = 0$.

    Lo primero es ver que su polinomio característico es:
    \[
        p(\lambda) = \lambda^3 -1
    \]
    y las raíces de este son:
    \[
        \lambda = \frac{-1 \pm \sqrt{-3}i}{2}
    \]
    y por tanto tenemos: $\lambda_1 = 1$, $\lambda_2 = cos(\dfrac{2\pi}{3})+ i sen(\dfrac{2\pi}{3})$ y $\lambda_3 = cos(\dfrac{2\pi}{3}) - i sen(\dfrac{2\pi}{3})$
    Así, la solución general es:
    \[
        x_n = c_1 * 1^n + c_2 cos(n(\dfrac{2\pi}{3})) + c_3 sen(n(\dfrac{2\pi}{3}))
    \]
    Solo faltaría resolver el sistema con las condiciones iniciales para obtener la solución particular del sistema.


    \subsection{Ejercicio 4}
    Dado un número real $\alpha$, determine una expresión para la sucesión $\{x_n\}$ que verifica:
    \[
        \begin{cases}
            x_{k+2} = x _{k+1} + x_k \ \ k \in \mathbb N
            x_1 = 1, \ \ \ x_2 = \alpha
        \end{cases}
    \]
    Su polinomio característico es $p(\lambda) = \lambda^2 - \lambda - 1$, cuyas soluciones son $\lambda_1 = \frac{1+\sqrt 5}{2}$ y $\lambda_2 = \frac{1+ \sqrt 5}{2}$. Con las condiciones iniciales, podemos obtener la expresión de la sucesión hallando $c_1$ y $c_2$.

    Apartado $b$): estudiar el comportamiento de la sucesión $\{y_n\}$ definida por:
    \[
        y_{k+1} = \frac{1}{1+y_k} \ \ \ k \geq 2 \quad y \quad y_2 = 1
    \]
    Hallando su polinomio característico, sus raíces son $y = \frac{-1 \pm \sqrt 5}{2}$

    \subsection{Ejercicio 9}

    La sucesión $\{\overline{x_n}\}_{n\ge0} = \{1, 2, 5, 12, 29, \dots\}$ es solución de cierta ecuación en diferencias lineal homogénea de orden 2 con coeficientes constantes.\\

    \textbf{a)} Deduzca dicha ecuación en diferencias.

    Sabemos que la ecuación en diferencias será de la forma
    \begin{equation} \label {1}
        x_{n+2} + a_1x_{n+1} + a_0x_n = 0, \quad n \ge 0
    \end{equation}

    Si sustuimos en (\ref 1) los valores $n=0$ y $n=1$ para la solución particular $\{\overline{x_n}\}_{n\ge0}$, obtenemos el siguiente sistema de ecuaciones:

$$\begin{cases} 5 + 2a_1 + a_0 = 0\\ 12 + 5a_1 + 2a_0 = 0 \end{cases}$$

    cuyas soluciones son $a_1 = -2$, $a_0 = -1$. Por tanto, la ecuación en diferencias es:
    \begin{equation} \label{2}
        x_{n+2} - 2x_{n+1} - x_n = 0, \quad n \ge 0
    \end{equation}

    \textbf{b)} Dé una expresión de la solución general de la ecuación en diferencias.

    Resolvemos la ecuación (\ref 2) como ya sabemos. Comenzamos hallando las raíces del polinomio característico:

$$p(\lambda) =  \lambda^2 - 2\lambda - 1 = 0 \iff \lambda = \frac{2 \pm \sqrt{8}}{2} \iff \begin{cases} \lambda_1 = 1 + \sqrt{2}\\ \lambda_2 = 1 - \sqrt{2} \end{cases}$$

    Como $\lambda_1$ y $\lambda_2$ son dos raíces distintas de $p(\lambda)$, sabemos que $\{ X_{\lambda_1}, X_{\lambda_2} \}$ es una base del espacio de soluciones $\Sigma$, donde

    $$X_{\lambda_1} = \{ \lambda_1^n \}_{n\ge 0} \quad ; \quad X_{\lambda_2} = \{ \lambda_2^n \}_{n\ge0}$$
    Por tanto, la solución general $\{x_n\}_{n\ge0}$ será una combinación lineal de la base $\{ X_{\lambda_1}, X_{\lambda_2} \}$, es decir:
    \begin{equation} \label{3}
        x_n = c_1\lambda_1^n + c_2\lambda_{2}^n, \quad c_1,c_2 \in \mathbb C, \quad n\ge 0
    \end{equation}

    Para determinar las constantes $c_1$ y $c_2$, sustituimos los valores $n=0$, y $n=1$ en (\ref 3) para la solución particular $\{\overline{x_n}\}_{n\ge0}$, obteniendo el siguiente sistema de ecuaciones:

$$\begin{cases} \overline{x_0} = c_1 + c_2 = 1\\ \overline{x_1} = c_1(1+\sqrt{2}) + c_2(1-\sqrt{2}) =2 \end{cases}$$

    cuyas soluciones son $c_1 = \frac{2+\sqrt{2}}{4}$, $c_2 = \frac{2 - \sqrt{2}}{4}$. Por tanto, la expresión de la solución general es:

    \begin{equation} \label{4}
        x_n = \frac{2+\sqrt{2}}{4}\left(1+\sqrt 2 \right)^n + \frac{2 - \sqrt 2}{4}\left( 1-\sqrt 2\right)^n, \quad n \ge 0
    \end{equation}

    \textbf{c)} Deduzca de forma razonada el valor, si existe, de $\displaystyle \lim_{n\to\infty} \frac{x_{n+1}}{x_n}$.

    Como $\sqrt 2 < 2$, se tiene que $\left| 1 - \sqrt 2 \right| < 1$, y por tanto $(1-\sqrt2)^n \to 0$ cuando $n \to \infty$. Utilizando esa información, y atendiendo a (\ref 4), podemos expresar el límite de la siguiente forma:

    $$\lim_{n\to\infty} \ddfrac{x_{n+1}}{x_n} = \lim_{n\to\infty} \ddfrac{\frac{2+\sqrt{2}}{4}\left(1+\sqrt 2 \right)^{n+1} + \frac{2 - \sqrt 2}{4}\left( 1-\sqrt 2\right)^{n+1}}{\frac{2+\sqrt{2}}{4}\left(1+\sqrt 2 \right)^n + \frac{2 - \sqrt 2}{4}\left( 1-\sqrt 2\right)^n} =$$

    $$= \lim_{n\to\infty} \frac{\left( 1 + \sqrt 2 \right)^{n+1}}{\left(1+\sqrt 2 \right)^n} = \lim_{n\to\infty} \left(1 + \sqrt 2 \right) = 1 + \sqrt 2$$
    \vspace{0.2em}

    Por tanto, $\displaystyle \exists \lim_{n\to\infty} \frac{x_{n+1}}{x_n} = 1 + \sqrt 2.$
    \newpage
    \subsection{Ejercicio 13}

    Presentamos el modelo de Samuelson modificado:
    $$Y_n = C_n + I_n$$
    $$C_n=b\cdot I_{n-1}$$
    $$I_n = C_n-kC_{n-1}+G$$

    Donde $Y_n$ es la renta nacional, $C_n$ es el consumo e $I_n$ es la inversión. G es el gasto público que suponemos constante.

    Debemos escribir en primer lugar la ley de recurrencia para las inversiones anuales $I_n$ y posteriormente calcular las condiciones sobre los parámetros k y b para que haya convergencia.

    Para comenzar tomamos la expresión de $I_n$ e intentamos expresarla exclusivamente con términos de $I_n$. Para ello sustituimos en la tercera ecuación los $C_n$ por la expresión correspondiente que observamos en la segunda ecuación.

    $$I_n = b\cdot I_{n-1}-k\cdot (b\cdot I_{n-2}) +G\quad \quad n \ge 2$$

    Como nos interesa tener la ecuación en recurrencia para valores mayores o iguales que 0 hacemos una pequeña transformación en los índices (le sumamos 2 a todos)

    $$I_{n+2} = b\cdot I_{n+1}-k\cdot (b\cdot I_{n}) +G\quad \quad n \ge 0$$

    Y reordenamos para obtener la \textbf{ley de recurrencia:}

    $$I_{n+2} - b\cdot I_{n+1} + k\cdot b\cdot I_{n} = G$$

    \newpage
    Ahora procederemos a calcular las condiciones sobre k y b para que haya convergencia.

    Sabemos que la $I_n = I_n^h+I_n^{(1)}$, es decir, que podemos expresar $I_n$ como suma de la homogénea asociada más una solución particular. Y para que se produzca esta situación de convergencia la homogénea asociada debe converger a cero ( $\lim \limits_{n \to +\infty} X_n = 0 $ ).

    $$I_n^h:\ I_{n+2} - b\cdot I_{n+1} + k\cdot b\cdot I_{n} = 0 $$

    Cuyo polinomio característico es:
    $$p(\lambda) = \lambda^2 - b\lambda + kb$$

    Al estar ante una ecuación en diferencias lineal homogénea de orden dos, para que converja a 0 tiene que cumplirse lo siguiente:

    \[
        \begin{rcases}
            (1)\ \quad p(1) = 1-b + kb\ \textgreater\ 0\\
            (2)\ p(-1) = 1+b + kb\ \textgreater\ 0\\
            (3)\ \quad\quad \quad\quad p(0) = kb\ \textless\ 1\\
        \end{rcases}
    \]

    Y a partir de esto obtenemos las condiciones:
    Partimos de $1 \textgreater b \textgreater 0$ y de $k\textgreater 0$.

    Probamos que la condición (1) se cumple siempre.
    Restamos b  en la primera expresión:

    $$1-b > b-b$$
    $$1-b > 0$$

    Y de la segunda multiplicamos por b, un número positivo y la expresión se mantiene:

    $$k\cdot b > 0$$

    Si sumamos estas dos expresiones obtenemos:

    $$1-b +k\cdot b > 0$$

    La condición (2) es trivial. Suma de números positivos es positiva.

    Y de la condición (3) obtenemos:
    $$k < \frac{1}{b}$$

    Con esto deducimos finalmente que converge si:
    \[
        \begin{rcases}
            0 < k < \frac{1}{b}\\
            0 < b < 1
        \end{rcases}
    \]

    \section{Relación 3}

    \begin{ejer}[1]
        Demuestre el Teorema de Cayley-Hamilton.
    \end{ejer}

    \begin{sol}
        Sea $A \in \mathcal M_{r}(\mathbb C)$ y sea $p(\lambda) = |A - \lambda I| = (-1)^r(\lambda^r + c_{r-1}\lambda^{r-1} + \hdots + c_1\lambda + c_0)$ su polinomio característico, con $c_i \in \mathbb{C}$. Llamemos $C = A - \lambda I$. Sabemos que: $$C^{-1} = \frac{1}{|C|} \cdot Adj(C)^T \implies |C|I = C \ Adj(C)^T$$

        Entonces, tendremos que
        \begin{equation} \label {th_cayley}
            p(\lambda)I = (A - \lambda I) \ Adj(A-\lambda I)^T
        \end{equation}

        Consideremos la matriz $B = Adj(A - \lambda I)$. Como las entradas de $B$ son polinomios en $\lambda$, podemos separar para cada $0 \le i \le r$ los coeficientes del término $\lambda^i$, y formar una matriz de números $B_i$ tal que:
        \begin{align*}
            Adj(A - \lambda I)^T &= B_{r-1}\lambda^{r-1} + B_{r-2}\lambda^{r-2} + \hdots + B_0
        \end{align*}

        Ahora, sustituyendo en \eqref{th_cayley} nos queda que
        \begin{align*}
            (-1)^r(\lambda^r + c_{r-1}\lambda^{r-1} + \hdots + c_0)I &= (A -\lambda I)(B_{r-1}\lambda^{r-1} + \hdots + B_0) \\
                                                                     &= -(B_{r-1}\lambda^r + \hdots + B_0\lambda) + (AB_{r-1}\lambda^{r-1}+\hdots + AB_0) \\
                                                                     &= -B_{r-1}\lambda^r + (-B_{r-2}+AB_{r-1})\lambda^{r-1} + \hdots + (-B_0 + AB_1)\lambda + AB_0
        \end{align*}

        Identificando término a término en la igualdad anterior, notamos que se verifica lo siguiente:
        \begin{align*}
            \label{}
            (-1)^rI &= -B_{r-1} \\
            (-1)^rc_{r-1}I &= -B_{r-2}+AB_{r-1} \\
                           &\ \ \vdots \\
            (-1)^rc_1I &= -B_0 + AB_1 \\
            (-1)^rc_0I &= AB_0
        \end{align*}
        Por tanto, concluimos que
        \begin{align*}
            \label{}
            p(A) &= (-1)^r \left[ A^r + c_{r-1}A^{r-1} + c_{r-2}A^{r-2} + \hdots + c_1A + c_0I \right] \\
                 &= (-1)^r \left[ IA^r + (c_{r-1}I)A^{r-1} + \hdots + c_1IA + c_0I \right] \\
                 &= -A^{r}B_{r-1}+A^{r-1}(-B_{r-2}+AB_{r-1}) + \hdots + A(-B_0 + AB_1)A + AB_0 \\
                 &= 0
        \end{align*}
    \end{sol}

    \begin{ejer}[3]
        Sea $A$ una matriz $4 \times 4$ con un valor propio $\lambda = 3$ de
        multiplicidad algebraica 4. Escriba todas las formas canónicas de Jordan
        asociadas a $A$ posibles.
    \end{ejer}

    \begin{sol}
        $$J_1 =
        \begin{pmatrix}
            3 & & & \\
              & 3 & & \\
              & & 3 & \\
              & & & 3
        \end{pmatrix} \ \ \
        J_2 =
        \begin{pmatrix}
            3 & 1 & & \\
              & 3 & 1 & \\
              & & 3 & 1 \\
              & & & 3
        \end{pmatrix} \ \ \
        J_3 =
        \begin{pmatrix}
            3 & 1 & & \\
              & 3 & & \\
              & & 3 & \\
              & & & 3
        \end{pmatrix}$$

        $$J_4 =
        \begin{pmatrix}
            3 & 1 & & \\
              & 3 & 1 & \\
              & & 3 & \\
              & & & 3
        \end{pmatrix} \ \ \
        J_5 =
        \begin{pmatrix}
            3 & 1 & & \\
              & 3 & & \\
              & & 3 & 1 \\
              & & & 3
        \end{pmatrix}$$
    \end{sol}

    \begin{ejer}[5]
        Resuelva la ecuación matricial $X^2 = I$, donde $X$ es una matriz real $2
        \times 2$.
    \end{ejer}

    \begin{sol} Sabemos que $X^2 = X \cdot X = I \iff X^{-1} = X$. Además, por el \textit{teorema de la forma canónica de Jordan}, sabemos que $X = PJP^{-1}$, y $J$ puede ser una de estas tres matrices, donde $\lambda_i \in \sigma(X)$:
        \begin{nlist}
        \item $J =
            \begin{pmatrix}
                \lambda_1 & 0\\
                0 & \lambda_2
            \end{pmatrix},\ $ con $\lambda_1 \neq \lambda_2$.
        \item  $J =
            \begin{pmatrix}
                \lambda_1 & 0\\
                0 & \lambda_1
            \end{pmatrix}$
        \item  $J =
            \begin{pmatrix}
                \lambda_1 & 1 \\
                0 & \lambda_1
            \end{pmatrix}$
        \end{nlist}

        Ahora, realizando manipulaciones algebraicas, tenemos que
        $$X^2 = X\cdot X = (PJP^{-1})(PJP^{-1}) = PJP^{-1}PJP^{-1} = P^{-1}J^2P$$

        Entonces, se verifica la siguiente equivalencia: $$X^2 = I \iff PJ^2P^{-1} = I \iff P^{-1}(PJ^2P^{-1})P = P I P^{-1} \iff J^2 = I$$

        Es decir, para que se verifique la ecuación del enunciado, necesariamente debe ser $J^2 = I$. Evaluemos ahora los casos que teníamos:
        \begin{nlist}
        \item $J^2 =
            \begin{pmatrix}
                \lambda_1^2 & 0 \\
                0 & \lambda_2^2
            \end{pmatrix} =
            \begin{pmatrix}
                1 & 0 \\
                0 & 1
            \end{pmatrix} \Rightarrow \lambda_1^2 = \lambda_2^2 = 1 \Rightarrow
            \lambda_1 = \pm 1, \lambda_2 = \pm 1$, con $\lambda_1 \ne \lambda_2$. Supongamos $\lambda_1 = 1,\ \lambda_2 = -1$. Entonces, la matriz $X$ que verifica $X^2 = I$ será de la forma
            $$X = P
            \begin{pmatrix}
                1 &  0\\
                0& -1
            \end{pmatrix}P^{-1},\quad P \in M_2(\R)\ \text{regular}$$

        \item $J^2 =
            \begin{pmatrix}
                \lambda_1^2 & \\
                            & \lambda_1^2
            \end{pmatrix} =  \begin{pmatrix}
                1 & 0 \\
                0 & 1
            \end{pmatrix} \Rightarrow J = I$ ó $J = -I$. Luego tenemos que $X = PJP^{-1} \Rightarrow X= I$ ó
            $X = -I$.
        \item $J^2 =
            \begin{pmatrix}
                \lambda_1 & 1 \\
                0 & \lambda_1
            \end{pmatrix}
            \begin{pmatrix}
                \lambda_1 & 1 \\
                0 & \lambda_1
            \end{pmatrix} =
            \begin{pmatrix}
                \lambda_1^2 & 2\lambda_1 \\
                0 & \lambda_1^2
            \end{pmatrix} = \begin{pmatrix}
                1 & 0 \\
                0 & 1
            \end{pmatrix} \ $.
            Vemos que no puede darse este caso.
        \end{nlist}
    \end{sol}

    \begin{ejer}[7] \hfill
        \begin{nlist}
        \item Demuestre que una matriz $A \in \mathbb C^{d\times d}$ es nilpotente si
            y sólo si $\sigma(A) = \{0\}$.
        \item Describa todas las matrices nilpotentes $2 \times 2$.
        \end{nlist}
    \end{ejer}

    \begin{sol}
        Decimos que una matriz $A$ es \textbf{nilpotente} si $\exists n \in \mathbb{N}$ tal
        que $A^n = 0$. Al menor $n$ tal que $A$ es nilpotente se le llama \textit{orden de nilpotencia}. Por ejemplo, las matrices
        $$ N = \begin{pmatrix}
            0 & 1 \\
            0 & 0
        \end{pmatrix}, \quad M =
        \begin{pmatrix}
            0 & 1 & 0 \\
            0 & 0 & 1\\
            0 & 0 & 0
        \end{pmatrix}$$

        son nilpotentes de orden $2$ y $3$, respectivamente. En general, si $J$ es un bloque de Jordan con todos los elementos diagonales nulos, entonces $J$ es nilpotente de orden $\dim J$. Esto es una consecuencia directa del \textit{Teorema de Cayley-Hamilton}.

        Resolveremos ahora los dos apartados del ejercicio.

        \underline{Comenzamos probando $(i)$}. Sea $A \in \mathbb{C}^{d\times d}$. Aplicando el \textit{Teorema de la forma canónica de Jordan}, sabemos que $A$ es semejante a una matriz diagonal por bloques $J$, y también sabemos que $A^n = 0 \iff J^n = 0$. Además, es obvio que $J^n = 0$ si y solo si cada uno de los bloques que la forman es nulo.



        Consideremos un bloque arbitrario de $J$, digamos $$J_\lambda = \begin{pmatrix}
            \lambda & 1\\
                    & \lambda & 1 \\
                    & & \ddots & \ddots \\
                    & & & \lambda & 1 \\
                    & & & & \lambda
        \end{pmatrix}, \quad \lambda \in \sigma(A)$$


        Tenemos que $J_\lambda = \lambda I + F$, donde $$F = \begin{pmatrix}
            0 & 1\\
              & 0 & 1 \\
              & & \ddots & \ddots \\
              & & & 0 & 1 \\
              & & & & 0
        \end{pmatrix}$$

        Esta matriz recibe el nombre de \textit{shift matrix}, y tiene la propiedad de que sus sucesivas potencias desplazan una posición hacia la derecha la diagonal con $1$, dejando el resto de entradas a $0$. Como $F$ es un bloque de Jordan con todos los elementos diagonales nulos, es nilpotente de orden $\dim F = \dim J_\lambda = p$. Se puede demostrar por inducción la siguiente fórmula para la potencia $n$-ésima de la matriz $J_\lambda$: $$ J_\lambda^n = (\lambda I + F)^n = \sum^{n}_{k=0} \begin{pmatrix}
            n \\
            k
        \end{pmatrix} \lambda^{n-k} F^k$$
        %%% TODO: Demostrar la fórmula por inducción
        Ahora, como $F$ es nilpotente de orden $p$, tenemos que $$ J_\lambda^n = \sum^{p-1}_{k=0} \begin{pmatrix}
            n \\
            k
        \end{pmatrix} \lambda^{n-k} F^k, \quad  \forall n \geq p $$

        \begin{align*}
            \label{}
            J^n_\lambda =
            \begin{pmatrix}
                \lambda^n & n\lambda^{n-1} & \begin{pmatrix}
                    n\\
                    2
                \end{pmatrix} \lambda^{n-2} & \hdots & \begin{pmatrix}
                    n \\
                    p-1
                \end{pmatrix} \lambda^{n-p +1} \\
                & \lambda^n & n\lambda^{n-1} & \hdots & \begin{pmatrix}
                n\\
                p-2
            \end{pmatrix} \lambda^{n-p+2}\\
            &  & \ddots & \ddots & \vdots \\
            & & & \lambda^n & n\lambda^{n-1}\\
            & & & & \lambda^n
        \end{pmatrix},\quad n \ge p
    \end{align*}

    Teniendo en cuenta todo el desarrollo anterior, concluimos que $J$ es nilpotente de orden $p$ si y solo si $\lambda = 0$.

    \textit{Demostración alternativa}. En primer lugar, sea $A \in \mathbb{C}^{d\times d}$ tal que su único valor propio es $0$. Entonces, el polinomio característico asociado será $p(\lambda) = (-1)^d\lambda^d$. Por el \textit{Teorema de Cayley-Hamilton}, tenemos que $p(A) = (-1)^dA^d = 0$, por lo que $A$ es nilpotente de orden $d$.

    Supongamos ahora que $A$ es nilpotente de orden $p$. Si $\lambda$ es un valor propio de $A$, ya sabemos que $\lambda^p$ es valor propio de $A^p$. Por tanto, existirá un $v \neq 0$ tal que $$ 0 = A^p v =  \lambda^p v,$$ lo cual es absurdo salvo en el caso $\lambda = 0$.

    \underline{Probamos ahora $(ii)$}. Consideremos $A \in \mathbb{C}^{2\times 2}$. Ya sabemos por $(i)$ que $A$ es nilpotente si y solo si su único valor propio es el $0$. $$A = \begin{pmatrix}
        a & b\\
        c & d
    \end{pmatrix} \implies p(\lambda) = |A - \lambda I| = \lambda^2 - (a+d)\lambda - bc + ad$$

    Como queremos que $0$ sea la única raíz de $p(\lambda)$, imponemos las siguientes condiciones: $$\begin{cases}
        a + d = 0\\
        bc = 0\\
        ad = 0
    \end{cases} \implies
    \begin{cases}
        a = d = 0\\
        b = 0 \text{ ó } c = 0
    \end{cases}$$

    Es decir, las matrices nilpotentes de orden $2$ son de la forma: $$\begin{pmatrix}
        0 & 0\\
        c & 0
    \end{pmatrix} \quad \text{ó} \quad
    \begin{pmatrix}
        0 & c\\
        0 & 0
    \end{pmatrix}, \quad c \in \mathbb{C}.$$
\end{sol}

\begin{ejer}[9]
    Supongamos que el polinomio $p(\lambda) = \lambda^k + a_{k-1}\lambda^{k-1} + \hdots + a_1\lambda + a_0$ tiene $k$ raíces distintas $\lambda_1, \lambda_2, \hdots, \lambda_k$. Sea $A$ la matriz \textit{compañera} de $p(\lambda)$ definida en el problema anterior y sea $V$ la matriz de \textit{Vandermonde} $$V = \begin{pmatrix}
        1 & 1 & \hdots & 1 \\
        \lambda_1 & \lambda_2 & \hdots & \lambda_k \\
        \vdots & \vdots & & \vdots \\
        \lambda_1^{k-1} & \lambda_2^{k-1} & \hdots & \lambda_k^{k-1} \\
    \end{pmatrix}$$
    Demuestre que la matriz $V^{-1}AV$ es diagonal.
\end{ejer}

\begin{sol}
    Vamos a probar que: $V^{-1}AV = D = \begin{pmatrix}
        \lambda_1 & & & \\
                  & \lambda_2 & & \\
                  & & \ddots & \\
                  & & & \lambda_k
    \end{pmatrix}$. En realidad, lo que vamos a demostrar es que $AV = VD$.

    Vamos a multiplicar $A$ por $V$.

    \begin{align*}
        \label{}
        \begin{pmatrix}
            0 & 1 & 0 & \hdots & 0 \\
            0 & 0 & 1 & \hdots & 0 \\
            \vdots & \vdots & \vdots & & \vdots \\
            0 & 0 & 0 & \hdots & 1 \\
            -a_0 & -a_1 & -a_2 & \hdots & -a_{k-1}
        \end{pmatrix} \cdot \begin{pmatrix}
            1 & \hdots & \hdots & \hdots & 1 \\
            \lambda_1 & \hdots & \hdots & \hdots & \lambda_k \\
            \vdots & & & & \vdots \\
            \vdots & & & & \vdots \\
            \lambda_1^{k-1} & \hdots & \hdots & \hdots & \lambda_k^{k-1}
        \end{pmatrix} &= \begin{pmatrix}
            \lambda_1 & \lambda_2 & \hdots & \hdots & \lambda_k \\
            \lambda_1 & \lambda_2 & \hdots & \hdots & \lambda_k \\
            \vdots & & & & \vdots \\
            \lambda_1^{k-1} & \lambda_2^{k-1} & \hdots & \hdots & \lambda_k^{k-1} \\
            \lambda_1^k & \lambda_2^k & \hdots & \hdots & \lambda_k^k
        \end{pmatrix}\\
        &= VD \\
        &= \begin{pmatrix}
        1 & 1 & \hdots & 1 \\
        \lambda_1 & \lambda_2 & \hdots & \lambda_k \\
        \vdots & \vdots & & \vdots \\
        \lambda_1^{k-1} & \lambda_2^{k-1} & \hdots & \lambda_k^{k-1} \\
    \end{pmatrix} \\
        &\cdot \begin{pmatrix}
        \lambda_1 & & & \\
                  & \lambda_2 & & \\
                  & & \ddots & \\
                  & & & \lambda_k
    \end{pmatrix}
    \end{align*}
\end{sol}

\begin{ejer}[11]
    Sea $A$ una matriz $k \times k$ tal que $|\lambda| < 1$ para cualquier valor propio $\lambda$ de $A$. Demuestre que:
\end{ejer}

\begin{sol}\hfill\\
    \begin{itemize}
        \item Si $|A-I| = 0$ entonces $\lambda = 1$ es valor propio de $A$.
        \item $I-A$ es regular, luego $\exists (I-A)^{-1}$.
            $$\rho(A) < 1 \Leftrightarrow \{ A^{k}\}_{k \geq 0} \rightarrow 0$$
            \begin{align*}
                \label{}
                s_n &= \sum^{n}_{k=0} A^{k} = I + A + A^2 + \hdots + A^n \\
                (I - A)s_n &= (I-A) + (A-A^2) + (A^2-A^3) + \hdots + (A^n - A^{n+1}) \\
                           &= I - A^{n+1} \\
                s_n &= (I-A)^{-1}(I-A^{n+1}) \\
                \sum^{\infty}_{k=0} A^k &= \lim_{n \to \infty} s_n = (I -A)^{-1}
            \end{align*}
    \end{itemize}
\end{sol}

\begin{ejer}[15]
    Se considera la matriz de Leslie $$ \begin{pmatrix}
        \frac{1}{2} & 2 & 1 \\
        \alpha & 0 & 0 \\
        0 & \beta & 0
    \end{pmatrix},$$
    donde $\alpha, \beta \in ]0, 1[$. Encuentra las regiones donde se produce extinción, crecimiento ilimitado o convergencia a equilibrio en el plano de parámetros $(\alpha, \beta)$.
\end{ejer}

\begin{sol}
    La tasa de fertilidad del primer grupo es $\frac{1}{2}$, la del segundo $2$, y la del tercero, $1$.
    \begin{align*}
        \label{}
        X_n = L^nX_0 \\
        \left\{ \frac{1}{\lambda_1^n} L^nX_0 \right\}_{n \geq 0} \rightarrow V \ ; \ \ LV = \lambda_1V
        \end{align*}

        Entonces \begin{itemize}
            \item $0 < \lambda_1 < 1 \Rightarrow$ extinción.
            \item $\lambda_1 = 1 \Rightarrow$ equilibrio.
            \item $\lambda_1 > 1 \Rightarrow$ crecimiento ilimitado.
        \end{itemize}
\end{sol}

\begin{ejer}[17]
    Se considera la matriz de Leslie $$\begin{pmatrix}
        \frac{1}{2} & \delta & \gamma \\
        \frac{1}{2} & 0 & 0 \\
        0 & \frac{1}{2} & 0
    \end{pmatrix}$$ donde $\delta, \gamma \geq 0$. \begin{enumerate}[label=\alph*)]
        \item Encuentre las regiones donde se produce extinción, crecimiento ilimitado o convergencia a equilibrio en el plano de parámetros $(\delta, \gamma)$.
        \item Describa la pirámide de edad a largo plazo correspondiente a los valores $\delta = \frac{1}{2}, \gamma = 1$.

    \end{enumerate}
\end{ejer}

\begin{sol}\hfill
    \begin{enumerate}[label=\alph*)]
        \item $R = \frac{1}{2} + \frac{1}{2}\delta + \frac{1}{4}\gamma$
            \begin{align*}
                \label{}
                \frac{1}{2} + \frac{1}{2}\delta + \frac{1}{4}\gamma = 1 \\
                \frac{1}{2}\delta + \frac{1}{4}\gamma = \frac{1}{2} \\
                2\delta + \gamma = 2 \\
                \gamma = 2 - 2\delta
            \end{align*}
        \item $\delta = \frac{1}{2}, \gamma = 1$

            $$L = \begin{pmatrix}
                \frac{1}{2} & \frac{1}{2} & 1 \\
                \frac{1}{2} & 0 & 0 \\
                0 & \frac{1}{2} & 0
            \end{pmatrix}$$

            $\exists \lambda_1$ valor propio dominante (único positivo). ¿Es $\lambda_1 = 1$?

            \begin{align*}
                \label{}
                |L-\lambda I| = \begin{vmatrix}
                    \frac{1}{2} - \lambda & \frac{1}{2} & 1 \\
                    \frac{1}{2} & -\lambda & 0 \\
                    0 & \frac{1}{2} & -\lambda
                \end{vmatrix} &= \left( \frac{1}{2} - \lambda \right)\lambda^2 - \frac{1}{2}\left( \frac{-\lambda}{2} - \frac{1}{2} \right) \\
                              &= -\lambda^3 + \frac{1}{2}\lambda^2 + \frac{1}{4}\lambda + \frac{1}{4}
            \end{align*}

            $p(1) = 0 \Rightarrow \lambda_1 = 1$

            Buscamos $v_1 : (L - I)v_1 = 0$
            \begin{align*}
                \label{eq:}
                \begin{pmatrix}
                    -\frac{1}{2} & \frac{1}{2} & 1 \\
                    \frac{1}{2} & -1 & 0 \\
                    0 & \frac{1}{2} & -1
                \end{pmatrix} \begin{pmatrix}
                    v_1 \\
                    v_2 \\
                    v_3
                \end{pmatrix} = \begin{pmatrix}
                    0 \\
                    0 \\
                    0
                \end{pmatrix}
            \end{align*}

            $$ \begin{rcases}
                \frac{1}{2}v_1 - v_2 = 0 \\
                \frac{1}{2}v_2 - v_3 = 0
            \end{rcases} \Rightarrow \begin{rcases}
                v_2 = \frac{1}{2}v_1 \\
                v_3 = \frac{1}{2}v_2
            \end{rcases}$$

            Por ejemplo $v_1 = 4 \Rightarrow \begin{cases}
                v_2 = 2 \\
                v_3 = 1
            \end{cases}$

            $$V = \begin{pmatrix}
                4 \\
                2 \\
                1
            \end{pmatrix} \ \ \ \ \frac{V}{||V||_1} = \begin{pmatrix}
                \frac{4}{7} \\
                \frac{2}{7} \\
                \frac{1}{7}
            \end{pmatrix}$$
     \end{enumerate}
\end{sol}

\begin{ejer}[20]
  Una determinada población está estructurada en base a tres grupos diferentes de edad: crías (hasta los 3 años),
jóvenes (de 3 a 6 años) y adultos (de 6 a 9 años). Es conocido que cada cría engendra en media una nueva cría, cada
joven engendra en media 1,5 crías y cada adulto engendra en media 0,5 crías. Además, las observaciones arrojan el
dato de que la mitad de las crías llegan a jóvenes, en tanto que sólo el 20 \% de los jóvenes sobrevive.

\begin{itemize}
\item Construya la matriz del modelo.
\item Si la distribución de tamaños iniciales es $P_0 = (3, 1, 0)^t$ (en las unidades adecuadas), calcule cuál será la distribución de tamaños al cabo de seis años.
\item Explique el comportamiento a largo plazo de la población (incluyendo su distribución porcentual por grupos
\end{itemize}
de edad).

\end{ejer}

\begin{sol}\hfill\\
  \begin{enumerate}[label=\alph*)]
  \item    La matriz de Leslie del modelo queda como sigue:

    \begin{center}
      $\begin{pmatrix}

        1 & 1.5 & 0.5\\
        0.5 & 0 & 0\\
        0 & 0.2 & 0

      \end{pmatrix}$
    \end{center}


 \item    Al cabo de seis años, la distribución será


    \[
    % P_6 = A^6 P_0 = \left(36.183125,\,11.930625,\,1.57375\right)
    P_6 = A^2 P_0 = \left(6.85,\ 2.25\ ,\ 0.3\right)
    \]

  \item    Para calcular la distribución porcentual por grupos de edad a largo plazo, primero vamos a comprobar la
    \emph{tasa de natalidad}:

    \[
    R = 1 + 1.5\times 0.5 + 0.5\times 0.5 \times 0.2 = 1.8 > 1
    \]

    Por tanto, la población no se extinguirá a largo plazo, sino que crecerá indefinidamente.
    Calculemos la pirámide de población. Para ello, tenemos que calcular el valor propio dominante de
    la matriz del modelo. Un breve análisis del polinomio característico basta para comprobar que no podemos calcular
    sus raíces mediante procedimientos algebraicos. El método de la potencia nos da el
    valor propio $\lambda_1 \approx 1.51$, y el vector propio asociado $v_{\lambda_1} \approx \left(1.000000,\,0.3297385,\,0.04349099\right)$.

    La pirámide de población a largo plazo será, aproximadamente:

    \[
    \frac{v_{\lambda_1}}{\|v_{\lambda_1}\|_{1}} \approx (0.72821, 0.24011, 0.03167)
    \]
  \end{enumerate}
\end{sol}
