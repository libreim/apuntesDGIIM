%%%%%%%%%%%%%%%%%%%%%%%%%%%%%%%%%%%%%%%%%%%%%%%%%%%%%%%%%%%%%%%%%%%%%%%%%%%%%%%%%
%%                                                                             %%
%% Este fichero contiene ejercicios tipo con sus respectivas soluciones.       %%
%% Autor: Ignacio Aguilera Martos                                              %%
%% https://github.com/nacheteam                                                %%
%%                                                                             %%
%%%%%%%%%%%%%%%%%%%%%%%%%%%%%%%%%%%%%%%%%%%%%%%%%%%%%%%%%%%%%%%%%%%%%%%%%%%%%%%%%

%%%%%%%%%%%%%%%%%%%%%%%%%%%%% Desintegración radioactiva %%%%%%%%%%%%%%%%%%%%%%%%%%%%%

\section{Desintegración radioactiva}
\begin{ejer}
	Un reactor transforma plutonio 239 en uranio 238 que es relativamente estable para
	uso industrial. Después de 15 años se determina que 0.0043 por ciento de la cantidad
	inicial A 0 de plutonio se ha desintegrado. Determina la semivida (tiempo necesario
	para que la cantidad inicial de los átomos se reduzca a la mitad) de este isótopo si
	la rapidez de desintegración es proporcional a la cantidad restante.
\end{ejer}
\begin{sol}
	Se toma como ecuación diferencial para la desintegración radioactiva la $m'(t)=-\lambda \cdot m(t)$ siendo $m(t)$ la masa en cada instante t.  
	Sabemos que después de 15 años hay $0.9957 \cdot A_0$ de masa siendo $A_0$ la masa que había inicialmente.  
	Integrando la ecuación diferencial dada obtenemos que $m(t) = c\cdot e^{-\lambda \cdot t}$. Con esto procedemos a obtener la constante $\lambda$.  
	$0.9957\cdot A_0 = A_0\cdot e^{-\lambda \cdot 15}$ con lo que $0.9957 = e^{-\lambda \cdot 15}$ de donde sacamos $ln(0.9957) = -15\cdot \lambda$ y por lo tanto obtenemos como constante $\lambda = \frac{ln(0.9957)}{-15}$.  
	Para calcular el tiempo de semivida tenemos ahora que ver en qué instante t obtenemos la mitad de la cantidad inicial con la constante que hemos despejado.  
	$\frac{1}{2}\cdot A_0 = A_0\cdot e^{\frac{ln(0.9957)}{15}\cdot t} \rightarrow \frac{ln(\frac{1}{2})}{\frac{ln(0.9957)}{15}}=t \rightarrow t = \frac{15\cdot ln(\frac{1}{2})}{ln(0.9957)} = 2412.753$ años.  
	La solución es que el tiempo de semivida es de 2412.753 años.
\end{sol}

%%%%%%%%%%%%%%%%%%%%%%%%%%%%% Poblaciones con ley de Malthus %%%%%%%%%%%%%%%%%%%%%%%%%%%%%

\section{Poblaciones}
\begin{ejer}
	La población de Malthusilandia (país cuyo crecimiento sigue la ley de Malthus) era de 20 millones en 1980 y se había duplicado en 1990. ¿Qué población tendrá en el año 2000?
\end{ejer}
\begin{sol}
	Para este tipo de problemas usamos la ecuación diferencial $m'(t) = \lambda \cdot m(t)$ de donde obtenemos integrando $m(t) = c\cdot e^{\lambda \cdot t}$.  
	Sabemos que en el instante t=0 la población es de 20 millones, por lo tanto $20M = c\cdot e^{\lambda \cdot 0} = c$. Por lo tanto c=20M.  
	Sabemos que la población 10 años después es de 40 millones, por lo tanto $m(10) = 40M \Rightarrow 40M = 20M \cdot e^{\lambda \cdot 10} \Rightarrow 2 = e^{\lambda \cdot 10} \Rightarrow \lambda = \frac{ln(2)}{10}$.  
	Si queremos saber que población habrá en el 2000, es decir, en t = 20 sólo tenemos que sustituir en la fórmula.  
	$m(20) = 20M\cdot e^{\frac{ln(2)}{10}\cdot 20} = 80M$.  
	La población en el año 2000 será de 80 millones.
\end{sol}

