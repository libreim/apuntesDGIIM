\section{Ecuaciones Lineales: Teoremas de Existencia y Unicidad}


\begin{ndef}[Ecuación diferencial lineal.]
Sea $(\alpha,\beta)\subseteq \R^N$ un intervalo abierto y sean $A:(\alpha,\beta)\to \M (\R)$ y $b:(\alpha,\beta)\to \R^d$ funciones continuas. Entonces una ecuación diferencial lineal es de la forma:

\begin{equation}
x'=A(t)x+b(t) \tag{C} \label{completa}
\end{equation}
\end{ndef}

\begin{nth}[Teorema de existencia y unicidad de la solución.]
Dados $t_0\in(\alpha,\beta)$ y $x_0 \in \R^d$ y consideramos el Problema de Valores Intermedios (PVI):
\begin{equation}
\left\{\begin{array}{lcl}
x'=A(t)x+b(t) \\
x(t_0)=x_0
\end{array}\right.
\tag{PVI}\label{PVI}
\end{equation}
Entonces $\exists ! \varphi : (\alpha,\beta)\to\R^d \; \; \varphi \in \C ^ 1(\R)$ que verifica
$$\varphi ' (t)=A(t)x+ b(t) \qquad \forall t \in (\alpha,\beta)$$
y además $\varphi(t_0)=x_0$.
\end{nth}
