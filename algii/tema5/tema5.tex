\subsection{Grupos simples y series normales}

\begin{ndef}[Serie normal y refinamiento]
Dado un grupo G, una serie normal de G es una sucesión finita de grupos de la forma $$\{1\} = H_0 \trianglelefteq H_1 \trianglelefteq ... \trianglelefteq H_n = G \;\;\;(1)$$ A los grupos $H_i$ se les llama términos de la serie y a los cocientes $H_i/H_{i-1}$ se les llama factores de G.

La serie se dice propia si las inclusiones son todas estrictas, esto es, si $H_i \propernormal H_{i+1} \, i = 0,1,...,n-1$. En este caso diremos que la longitud  de la serie es n (esto es, $G$ no se cuenta).

Supongamos una serie normal $$\{1\} = K_0 \trianglelefteq K_1 \trianglelefteq ... \trianglelefteq K_m = G\;\;\;(2)$$ Diremos que la serie (1) es un refinamiento de la serie (2) si $m \le n$ y además todos los grupos de (2) aparecen en (1) (esto es, si la serie (1) es más larga que la serie (2)). El refinamiento se dirá propio si $m < n$.
\end{ndef}

\begin{ejemplo}
1. $\{1\} \trianglelefteq A_4 \trianglelefteq S_4$\\
2. $\{1\} \trianglelefteq K \trianglelefteq S_4$\\
3. $\{1\} \trianglelefteq K \trianglelefteq A_4 \le S_4$ refina a las dos series anteriores.\\
4. $\{1\} \trianglelefteq C_2 \trianglelefteq K \trianglelefteq A_4 \trianglelefteq S_4$ donde $C_2$ no es normal en $A_4$ ni en $S_4$. De hecho, no hay más refinamientos de esta.
\end{ejemplo}

\begin{ndef}[Serie de composición]
Una serie de composición de un grupo G es una serie normal, propia y que no admite refinamientos propios. Los factores de una serie de composición se llaman factores de composición de G (ya que sólo dependen de G).
\end{ndef}

(Motivar la definición de grupo simple como en 168 del libro de la UNAM).

\begin{ndef}[Grupo simple]
Un grupo G es simple si $G \neq \{1\}$ y no tiene subgrupos normales propios.
\end{ndef}

Claramente, la noción de grupo simple se mantiene por isomorfismo.

\begin{ejemplo}
1. Si $|G| = p$ primo entonces G es simple ya que ni siquiera tiene subgrupos propios. \\
2. Si G es abeliano, G es simple $\iff G \cong C_p$ con p un número primo.

En efecto, 

$\Leftarrow$ Si $G \cong C_p$ con p primo los subgrupos de G son los impropios. \\
$\Rightarrow$ Si G es abeliano y simple para empezar G no es trivial por definición y entonces podemos tomar $x \in G \setminus \{1\}$ y como G es abeliano el subgrupo $<x>$ será normal en G. Como G es simple, necesariamente $G = <x>$, luego G es cíclico.

Faltaría por ver que G es de orden finito. Para ello consideremos el orden de x. Si fuera infinito entonces $G \cong \mathbb{Z}$ pero $\mathbb{Z}$ no es simple porque tiene infinitos subgrupos y la simplicidad es un invariante por isomorfismo.

Además necesariamente el orden del grupo debe ser primo. Pues si no fuera primo existirían subgrupo propios según el teorema que describe el retículo de subgrupos de un grupo cíclico y como el grupo es abeliano se tendría que serían normales.
\end{ejemplo}

El siguiente teorema caracteriza a los grupos simples como los ladrillos básicos de construcción de las series de composición al modo de los números primos en $\mathbb{Z}$.

\begin{nth}[Condición de factores simples]
Sea $\{1\} = H_0 \trianglelefteq H_1 \trianglelefteq ... \trianglelefteq H_n = G$ una serie normal. 

La serie es de composición $\iff H_i/H_{i-1}$ son simples para todo $i = 1,...,n$.
\end{nth}
\begin{proof}
$\Rightarrow$ Supóngase que la serie es de composición y supongamos por reducción al absurdo que existe un i tal que $H_i/H_{i-1}$ no es simple. Por tanto existiría un subgrupo normal propio $K/H_{i-1}$ que por el segundo teorema isomorfismo implicaría que $H_{i-1} \propernormal K \propernormal H_{i}$. Ahora bien, por ser una serie de composición no puede admitir refinamientos propios y llegamos a una contradicción.

$\Leftarrow$ Para empezar el hecho de que los cocientes sean simples $H_i/H_{i-1}$ me dice que la serie es propia ya que $H_i/H_{i-1} \neq \{1\} \implies H_{i-1} \propernormal H_i$.

Supongamos ahora que $\{1\} = K_0 \trianglelefteq K_1 \trianglelefteq ... \trianglelefteq K_m = G$ es un refinamiento propio de la serie (con lo que en particular $m > n$). Sea $K_l$ el mayor de los que no aparecen en la serie original de modo que $l < m$ y además $K_{l+1}$ aparece en la original. Sea $K_{l+1} = H_r$. 

Se tiene que $H_{r-1} \propernormal K_l \propernormal K_{l+1} = H_r$ de donde $K_l/H_{r-1} \propernormal H_r/H_{r-1}$ y $K_l/H_{r-1}$ es no trivial. Como $H_r/H_{r-1}$ es simple llegamos a una contradicción.
\end{proof}

\begin{ejemplo}
1. Probemos que en efecto la serie $\{1\} \trianglelefteq C_2 \trianglelefteq K \trianglelefteq A_4 \trianglelefteq S_4$ no admite refinamientos propios. Para ello calculamos los cocientes sucesivos. 

$C_2/\{1\} \cong C_2$ que es simple puesto que es abeliano.\\
$K/C_2 \cong C_2$ ya que $|K/C_2| = 2$.\\
$A_4/K \cong C_3$ ya que $|A_4/K| = 3$.\\
$S_4/A_4 \cong C_2$ ya que $|S_4/A_4| = 2$.
 
2.$\mathbb{Z}$ no tiene series de composición.

Si tuviera una serie de composición $\{0\} = t_1\mathbb{Z} \trianglelefteq ... \trianglelefteq t_{n-1}\mathbb{Z} \trianglelefteq \mathbb{Z}$, en particular $(t_1\mathbb{Z})/\{0\} \cong t_1\mathbb{Z}$ debería ser simple, pero se trata de un grupo abeliano que tiene como subgrupos los múltiplos de $t_1$ y los subgrupos de un grupo abeliano son todos normales.
\end{ejemplo}

\begin{nth}[Existencia de una serie de composición para grupos finitos]
Si G es un grupo finito entonces tiene al menos una serie de composición.
\end{nth}
\begin{proof}
Procedamos por inducción fuerte sobre $|G|$. 

Si $|G| = 2$ entonces la serie $\{1\} \trianglelefteq G$ es una serie de composición.\\
Si $|G| > 2$ entonces tomamos $K$ el mayor subgrupo normal contenido propiamente en $G$. Obsérvese que al menos $\{1\}$ es normal en $G$ y que existirá un mayor subgrupo normal ya que el retículo de subgrupos es finito. 

Por hipótesis de inducción, dicho subgrupo $K$ admite una serie de composición $$\{1\} = K_0 \trianglelefteq K_1 \trianglelefteq ... \trianglelefteq K_r = K$$ de modo que la serie $$\{1\} = K_0 \trianglelefteq K_1 \trianglelefteq ... \trianglelefteq K \trianglelefteq G$$ es una serie de composición para $G$.
\end{proof}

\begin{ejemplo}
La serie de composición del grupo no tiene por qué ser única. Así $\{1\} \propernormal <r^2> \propernormal <r> \propernormal D_4$ y si $K = <r^2,s>$ es uno de los subgrupos de Klein tenemos $\{1\} \propernormal <s> \propernormal K \propernormal D_4$. Y observamos que los factores en ambos casos son isomorfos a $C_2$.
\end{ejemplo}

La pregunta es si dadas dos series de composición existirá alguna relación entre ellas.

\begin{ndef}[Series equivalentes]
Dadas dos series normales de G, $$\{1\} = G_0 \trianglelefteq G_1 \trianglelefteq ... \trianglelefteq G_n = G$$ y $$\{1\} = H_0 \trianglelefteq H_1 \trianglelefteq ... \trianglelefteq H_n = G$$ Diremos que estas dos series son equivalentes o isomorfas si:

1) m = n\\
2) $\exists \sigma \in S_n$ tal que $G_i/G_{i-1} \cong H_{\sigma(i)}/H_{\sigma(i)-1}$ con $i=1,...,n$. 

Esto es, tienen la misma longitud y factores isomorfos salvo el orden en que ocurren.
\end{ndef}

Probamos a continuación el Cuarto teorema de isomorfía, que nos será útil para demostrar para demostrar el lema de Schreier para lo cual usaremos dos lemas previos.

\begin{nprop}[Ley Modular o Regla de Dedekind]
Sea $G$ un grupo. $A,B,C \le G$ con $A \le C$. Entonces $A(B \cap C) = (AB) \cap C$
\end{nprop}
\begin{proof}
Procedemos por doble inclusión

$\subseteq$ Si tomo un elemento de $A(B \cap C)$ entonces es producto de un elemento $a \in A$ y de un elemento de $x \in B \cap C$. Como $x \in B$ se tiene que $ax \in AB$ y como $A \le C$ se tiene que $ax \in C$ de done $ax \in (AB) \cap C$.

$\supseteq$ Recíprocamente si tomo un elemento de $(AB) \cap C$ entonces será de la forma $z = ab$ con $z \in C$ y $a \in A$,$b \in B$. Pero $a$ también está en $C$ y eso nos permite afirmar que $b = a^{-1}ab \in C$ de donde $z \in A(B \cap C)$. Como se quería.
\end{proof}

\begin{nprop}[Consecuencia del tercer teorema de isomorfía]\label{lemma:consecuencia-tercer-teorema-isomorfia}
Sea $G$ un grupo. $A,B,C \le G$ y $B \trianglelefteq A$. Entonces se verifica:\\
i) $B \cap C \cong A \cap C$ y $\frac{A \cap C}{B \cap C} \cong \frac{B(A \cap C)}{B}$\\
ii) Si además $C \trianglelefteq G$ entonces $BC \trianglelefteq AC$ y $\frac{AC}{BC} \cong \frac{A}{B(A \cap C)}$
\end{nprop}
\begin{proof}
i) Basta aplicar el tercer teorema de isomorfismo con $H = A \cap C$ y $K = B$.\\
ii) Puesto que $C \trianglelefteq G$ por la proposición \ref{theorem:teorema-producto} se tendrá que $BC,AC \le G$ y como $B \trianglelefteq A$ claramente $BC \le AC$.

Veamos que $BC \trianglelefteq AC$. Sea $ac \in AC$ y $bc' \in BC$, entonces $(ac)(bc')(ac)^{-1} = (aca^{-1})(aba^{-1})(ac'c^{-1}a^{-1}) \in CBC = BCC = BC$ por ser $B \trianglelefteq A$, $C \trianglelefteq G$ y $BC =CB$.

El isomorfismo se deduce al aplicar el tercer teorema de isomorfismo para $H = A$ y $K = BC$ teniendo en cuenta que por la regla de Dedekind $(BC) \cap A \cong B(A \cap C)$ y que $HK = ABC = AC$ ya que $B \le A$.
\end{proof}

\begin{nprop}[Cuarto teorema de isomorfía]
Sea G un grupo y $C_1,A_1,C_2,A_2$ subgrupos de G tales que $C_1 \trianglelefteq A_1$ y $C_2 \trianglelefteq A_2$. Entonces:

1. $(A_1 \cap C_2)C_1 \trianglelefteq (A_1 \cap A_2)C_1$.\\
2. $(C_1 \cap A_2)C_2 \trianglelefteq (A_1 \cap A_2)C_2$.\\
3. $(A_1 \cap A_2)C_1/(A_1 \cap C_2)C_1 \cong (A_1 \cap A_2)C_2/(A_2 \cap C_1)C_2$.
\end{nprop}
\begin{proof}
Considérese como espacio ambiente $A_2$ y apliquemos el tercer teorema de isomorfismo para $K = C_2$ y $H = A_1 \cap A_2$. Entonces $H \cap K = A_1 \cap A_2 \cap C_2 = A_1 \cap C_2 \trianglelefteq A_1 \cap A_2$ y análogamente $A_2 \cap C_1 \trianglelefteq A_1 \cap A_2$. Entonces es fácil razonar que su producto $B = (A_1 \cap C_2)(A_2 \cap C_1) \trianglelefteq A_1 \cap A_2$ (para ello utilícese la proposición \ref{theorem:teorema-producto} y el teorema \ref{theorem:criterio-normalidad}). 

Aplicamos el apartado ii) del lema \ref{lemma:consecuencia-tercer-teorema-isomorfia} para el producto $B$, $A = A_1 \cap A_2$ y $C = C_1$ entonces $$BC = BC_1 = (A_1 \cap C_2)(A_2 \cap C_1)C_1 = (A_1 \cap C_2)C_1 \trianglelefteq (A_1 \cap A_2)C_1$$ y $$\frac{AC}{BC} \cong \frac{(A_1 \cap A_2)C_1}{(A_1 \cap C_2)C_1} \cong \frac{A}{B(A \cap C)} = \frac{A_1 \cap A_2}{B(A_1 \cap A_2 \cap C_1)} = \frac{A_1 \cap A_2}{(A_1 \cap C_2)(A_2 \cap C_1)}$$. Por simetría se obtendría el punto 2 y el segundo isomorfismo.
\end{proof}

\begin{nth}[Lema de refinamiento de Schreier (1928)]
Dos series de un grupo G admiten refinamientos equivalentes.
\end{nth}
\begin{proof}
Sean $$\{1\} = G_0 \trianglelefteq G_1 \trianglelefteq ... \trianglelefteq G_n = G$$ y $$\{1\} = H_0 \trianglelefteq H_1 \trianglelefteq ... \trianglelefteq H_m = G$$ dos series normales para el grupo $G$. 

Para $i \in I_n$ y $j \in I_m$ notaremos $G_{ij} = (G_i \cap H_j)G_{i-1}$ $(i \neq 0)$ y $H_{ij} = (H_{j} \cap G_i) H_{j-1}$ $(j \neq 0)$. Se tiene que $$G_{i-1} = G_{i0} \le G_{i1} \le ... \le G_{im} = G_i \;\;\; (1)$$ y $$H_{j-1} = H_{0j} \le H_{1j} \le ... \le H_{nj} = H_j \;\;\; (2)$$

Usando el lema con $C_1 = G_{i-1} \trianglelefteq A_1 = G_i$ y $C_2 = H_{j-1} \trianglelefteq A_2 = H_j$ de aquí se obtiene que $G_{ij-1} \trianglelefteq G_{ij}$ y que $H_{i-1j} \trianglelefteq H_{ij}$ y se obtiene que en (1) y (2) las series son normales. Además, por el cuarto teorema de isomorfía $G_{ij}/G_{ij-1} \cong H_{ij}/H_{i-1j}$.

Para ver que las series obtenidas son equivalentes basta ver que tienen la misma longitud. 

$1 = G_0 \trianglelefteq G_{10} \trianglelefteq G_{11} \trianglelefteq ... \trianglelefteq G_{1m} = G_1 = G_{20} \trianglelefteq G_{21} \trianglelefteq ... \trianglelefteq G_{2m} = G_2 \trianglelefteq ... \trianglelefteq G_{nm} = G_n = G$

Cuya longitud es $n+(m-1)n = nm$.

$1 = H_0 \trianglelefteq H_{01} \trianglelefteq H_{02} \trianglelefteq ... \trianglelefteq G_{n1} = H_1 = H_{02} \trianglelefteq H_{12} \trianglelefteq ... \trianglelefteq H_{n2} = H_2 \trianglelefteq ... \trianglelefteq H_{nm} = H_m = G$

Cuya longitud es $nm$.
\end{proof}


\begin{nth}[Teorema de Jordan-Hölder]
Sea G un grupo finito, entonces:\\
1) Toda serie normal de G admite un refinamiento que es una serie de composición de G.\\
2) Cualesquiera dos series de composición de G son equivalentes.
\end{nth}
\begin{proof}
1. Denotemos por $S_1$ a una serie normal de $G$ y por $S_2$ a una serie de composición de $G$, que existe puesto que $G$ es finito. Por el lema de refinamiento de Schreier ambas series admiten refinamientos equivalentes. 

$S_1$ admite un refinamiento equivalente a un refinamiento de $S_2$ y como $S_2$ es de composición el refinamiento de $S_1$ es equivalente a la serie $S_2$.

Pero una serie que sea equivalente a una serie de composición necesariamente ha de ser una serie de composición ya que los factores de ambas series salvo el orden son isomorfos y como los de la serie de composición son simples por la condición de factores simples se deduce que tenemos una serie de composición.

2. Por el lema de refinamiento de Schreier las series de composición $S_1$ y $S_2$ admiten refinamientos equivalentes. Pero como $S_1$ y $S_2$ son series de composición sus refinamientos coinciden con $S_1$ y $S_2$ luego son equivalentes.
\end{proof}

\begin{ndef}[Longitud y factores de un grupo finito]
Sea G un grupo finito.

La longitud de G es la longitud de cualquier serie de composición de G. Lo denotaremos por L(G).\\
Los factores de composición de G son los factores de cualquiera de sus series de composición. Al conjunto de los factores de composición lo denotaremos por Fact(G).
\end{ndef}

\subsection{Grupos solubles}

\begin{ndef}[Grupo soluble]
Un grupo G es soluble si tiene una serie normal: $$\{1\} = G_0 \trianglelefteq G_1 \trianglelefteq ... \trianglelefteq G_n = G$$ cuyos factores $G_i/G_{i-1}$ con $i=1,...,n$ son abelianos.
\end{ndef}

Notemos que la solubilidad de un grupo es invariante por homomorfismo.

\begin{ejemplo}
Todo grupo abeliano es soluble ya que $\{1\} \trianglelefteq G$ es una serie normal y además $G/\{1\} = G$ es abeliano.
\end{ejemplo}

\begin{nth}[Caracterización por factores de un grupo finito soluble]
Sea G un grupo finito. Entonces equivalen:

1) Los factores de composición de G son cíclicos de orden primo.\\
2) G tiene una serie normal con factores cíclicos.\\
3) G es soluble.
\end{nth}
\begin{proof}
Claramente $1) \implies 2) \implies 3)$.

Veamos que $3) \implies 1)$. En efecto, sea $\{1\} = G_0 \trianglelefteq G_1 \trianglelefteq ... \trianglelefteq G_n = G$ una serie normal para $G$ cuyos cocientes $G_i/G_{i-1}$ son abelianos con $i = 1, ... ,n$.

Por el teorema de Jordan-Hölder dicha serie admite un refinamiento que es una serie de composición $\{1\} = H_0 \trianglelefteq H_1 \trianglelefteq ... \trianglelefteq H_m = G$. Si demostramos que los cocientes además de ser simples son abelianos habríamos terminado ya que un grupo abeliano y simple es cíclico de orden primo. 

Consideremos $G_{j-1} \trianglelefteq H_{r-1} \propernormal H_{r} \trianglelefteq G_{j}$ y por el segundo teorema de isomorfismo tenemos que $H_r/H_{r-1} \cong (H_r/G_{j-1})/(H_{r-1}/G_{j-1})$ y dado que $H_r/G_{j-1} \le G_j/G_{j-1}$ que es abeliano, es también él mismo abeliano. En consecuencia el cociente es abeliano y por isomorfismo hemos acabado.
\end{proof}

\begin{ejemplo}
$S_2$ es un grupo abeliano y por tanto es soluble. Se verifica que $Fact(S_2) = \{C_2\}$.\\
En $S_3$ tenemos la serie de composición $\{1\} \trianglelefteq A_3 \trianglelefteq S_3$ de donde $Fact(S_3) = \{C_2,C_3\}$ y por tanto $S_3$ es un grupo soluble.\\
En $S_4$ consideramos la serie de composición $\{1\} \trianglelefteq C \trianglelefteq K \trianglelefteq A_4 \trianglelefteq S_4$ con factores $Fact(S_4) = \{C_2,C_3,C_2,C_2\}$ y por tanto $S_4$ es un grupo soluble.
\end{ejemplo}

\begin{nprop}[Relación de la solubilidad con cocientes y subgrupos]
Sea $G$ un grupo:

1. Si $G$ es soluble y $H \le G \implies H$ es soluble.\\
2. Si $G$ es soluble y $N \trianglelefteq G \implies G/N$ es soluble.\\
3. Si $N \trianglelefteq G$ tal que $N$ y $G/N$ son solubles $\implies$ G es soluble.
\end{nprop}
\begin{proof}
1. Sea $\{1\} = G_0 \trianglelefteq G_1 \trianglelefteq ... \trianglelefteq G_n = G$ con $G_i/G_{i-1}$ abeliano. Vamos a ver que $\{1\} = G_0 \cap H \trianglelefteq G_1 \cap H \trianglelefteq ... \trianglelefteq G_n \cap H= G$ hace soluble a $H$.

Considerando como espacio ambiente $H_i$ tenemos la normalidad de $G_{i-1} \trianglelefteq G_i$ y podemos aplicar el tercer teorema de isomorfía a $K = G_{i-1}$ y $H = H \cap H_i$ de modo que $\frac{H \cap G_i}{H \cap G_{i-1}} = \frac{H \cap G_i}{H \cap G_{i} \cap G_{i-1}} \cong \frac{(H \cap G_i)G_{i-1}}{G_{i-1}} \le \frac{G_i}{G_{i-1}}$ que es un grupo abeliano. Se concluye ya que los subgrupos de grupos abelianos son abelianos.

2. Como en 1. sea $\{1\} = G_0 \trianglelefteq G_1 \trianglelefteq ... \trianglelefteq G_n = G$ con $G_i/G_{i-1}$ abeliano. Entonces claramente (revisar notas) $G_iN \trianglelefteq G_{i+1}N$ y la serie $\{1\} = G_0N/N \trianglelefteq G_1N/N \trianglelefteq ... \trianglelefteq G_nN/N = G/N$ nos dirá que $G/N$ es abeliano.

En efecto, $\frac{G_iN/N}{G_{i-1}N/N} \cong \frac{G_iN}{G_{i-1}N} \cong \frac{G_i}{G_i \cap G_{i-1}N} \cong \frac{G_i/G_{i-1}}{\frac{G_i \cap G_{i-1}N}{G_{i-1}}}$. Donde en la primera igualdad se utiliza el segundo teorema de isomorfismo gracias a la normalidad de $N$ en $G$ y en la segunda igualdad se utiliza el tercer teorema de isomorfismo con $H = G_{i}$ y $K = G_{i-1}N$ y en la última igualdad se vuelve a usar el segundo teorema de isomorfismo con la normalidad de $G_{i-1}$ en $G_i$. Por ser el numerador abeliano se deduce que los cocientes de la nueva serie son abelianos y hemos acabado.

3. De la solubilidad de $N$ deducimos que existe una serie $\{1\} = N_0 \trianglelefteq N_1 \trianglelefteq ... \trianglelefteq N_r = N$ con $N_i/N_{i-1}$ abeliano. Como $G/N$ es soluble existe otra serie $\{1\} = G_0/N \trianglelefteq G_1/N \trianglelefteq ... \trianglelefteq G_n/N = G/N$ con $(G_i/N)/(G_{i-1}/N) \cong G_i/G_{i-1}$ abeliano.

Por tanto la serie $\{1\} = N_0 \trianglelefteq N_1 \trianglelefteq ... \trianglelefteq N_r = N = G_0 \trianglelefteq G_1 \trianglelefteq ... \trianglelefteq G_n = G$ hace al grupo $G$ soluble.
\end{proof}

\begin{ncor}[Solubilidad de los grupos diédricos]
$D_n$ es un grupo soluble $\forall n \ge 3$.
\end{ncor}
\begin{proof}
Consideremos $N = <r>$. Como $[D_n:N] = 2$ sabemos que $N \trianglelefteq D_n$. Como $N$ es abeliano, es soluble y como $|D_n/N| = 2$ entonces $D_n/N \cong C_2$ que es un grupo abeliano y por tanto es soluble. Finalmente, aplicando 3. se llega a que $D_n$ es soluble.
\end{proof}

\begin{nth}[Teorema de Abel]
$A_n$ es un grupo simple $\forall n \ge 5$.
\end{nth}

\begin{ncor}[Solubilidad de los grupos de permutaciones]
$S_n$ es soluble $\iff n \le 4$.
\end{ncor}
\begin{proof}
En efecto, si $n \le 4$ por el ejemplo anterior anterior sabemos que $S_n$ es soluble. Ahora, si $n > 4$ sabemos que $A_n$ es simple por el teorema de Abel y por tanto la serie $\{1\} \trianglelefteq A_n \trianglelefteq S_n$ es una serie de composición ya que sus factores son $\{C_2,A_n\}$ que son simples y ya que $A_n$ no es cíclico ni de orden primo por la caracterización por factores de un grupo finito soluble se tiene que $S_n$ no es soluble.
\end{proof}

\begin{ndef}[Subgrupo conmutador]
Dado un grupo $G$ y $x,y \in G$, su conmutador es $[x,y] := xyx^{-1}y^{-1}$. Es claro que $xy = [x,y]yx$ y es por ello que se llama conmutador. El subgrupo conmutador o primer derivado de $G$ es $$[G:G] := <\{[x,y]:x,y \in G\}>$$
\end{ndef}

\begin{nprop}[Propiedades del subgrupo conmutador]
1. $G$ es abeliano $\iff [G,G] = 1$.\\
2. $[G,G] \trianglelefteq G$.\\
3. $G/[G,G]$ es abeliano y se le llama el abelianizado del grupo $G$, $G_{ab}$.\\
4. Si $f:G \rightarrow A$ es un homomorfismo y $A$ es abeliano entonces $[G,G] \le Ker(f)$.\\
5. Si $N \trianglelefteq G$ entonces $G/N$ es abeliano $\iff [G,G] \le N$. En otras palabras, el conmutador es el subgrupo más pequeño que hace abeliano al cociente.
\end{nprop}
\begin{proof}
1. Es evidente.\\
2. Basta usar la condición de normalidad $a[x,y]a^{-1} = [axa^{-1},aya^{-1}] \in [G:G]$.\\
3. $(x[G:G])(y[G:G]) = (xy)[G:G] = (yx[y^{-1}x^{-1}])[G:G] = yx[G:G] = (y[G:G])(x[G:G])$.\\
4. $f([x,y]) = f(x)f(y)f(x)^{-1}f(y)^{-1} = f(x)f(x)^{-1}f(y)f(y)^{-1} = 1$ de donde $[x,y] \in Ker(f)$.\\
5. $\Leftarrow$ Si $[G:G] \triangleleft N$ entonces $(xN)(yN)=(xy)N=[x,y]yxN=yxN = (yN)(xN)$.\\
$\Rightarrow$ Si $G/N$ es abeliano entonces $[G:G] \le Ker(p) = N$ donde $p$ es la proyección canónica.
\end{proof}

\begin{ndef}[Derivado y serie derivada de un grupo]
Dado un grupo $G$, el n-ésimo derivado de $G$ es por recurrencia:

$G^{0}:= G$\\
$G^{1}:= [G,G]$\\
$G^{n+1}:=[G^{n},G^{n}]$ $n \ge 1$\\

En estas condiciones tenemos una serie normal, en general no finita de la forma $$G^{n+1} \trianglelefteq G^{n} \trianglelefteq ... \trianglelefteq G^{1} \trianglelefteq G$$ llamada serie derivada del grupo $G$ cuyos factores son abelianos.
\end{ndef}

\begin{nth}[Caracterización por derivados de un grupo soluble]
Sea $G$ un grupo:

$G$ es soluble $\iff \exists n $ tal que $G^{n} = \{1\}$.

En otras palabras, $G$ es soluble si y sólo si la serie derivada es finita.
\end{nth}
\begin{proof}
$\Rightarrow$ Si $G$ es soluble y $\{1\} = G_0 \trianglelefteq G_1 \trianglelefteq ... \trianglelefteq G_n = G$ es una serie tal que $G_i/G_{i-1}$ es abeliano entonces vamos a probar que para todo $k$ se verifica que $G^{k)} \le H_{n-k}$. De donde para $k = n$ se tendría que $G^{n)} \le H_{n-n} = H_0 = \{1\}$ de donde $G^{n)} = \{1\}$.

Para $k = 1$ se verifica que como $G/G_{n-1}$ es abeliano entonces $[G,G]  \le G_{n-1}$. 

Para $k > 1$ tomemos como hipótesis de inducción que $G^{k)} \le G_{n-k}$. Por tanto como $G_{n-k}/G_{n-(k+1)}$ es abeliano entonces $G^{k+1)} \le [G_{n-k},G_{n-k}] \le G_{n-(k+1)}$. Y hemos acabado.

$\Leftarrow$ En efecto, si $\exists n $ tal que $G^{n} = \{1\}$ entonces la serie derivada es finita y sus cocientes son abelianos por la propiedad del abelianizado.
\end{proof}
