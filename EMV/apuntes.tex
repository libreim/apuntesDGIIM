\section{Fundamentación probabilística de vectores aleatorios}

\begin{ndef}[Variable Aleatoria]
    Sea $(\Omega, \mathcal{F}, P)$ un espacio de probabilidad. Una \textit{variable aleatoria} es una función medible
    \begin{align*}
    X:(\Omega, \mathcal{F}) &\rightarrow (\mathbb{R}, \mathbb{B})\\
    \omega &\mapsto X(\omega)
    \end{align*}
    cuya condición de medibilidad es \(X^{-1}(B) \in \mathcal{F}\) para todo \(B \in \mathbb{B}\).
\end{ndef}

\begin{ndef}[Probabilidad inducida]
    La \textit{probabilidad inducida} por una variable aleatoria $X$ a partir de $P$ se define como
    \begin{align*}
    P_{X}:\mathbb{B} &\rightarrow [0,1] \\
    B &\mapsto P[X^{-1}(B)].
    \end{align*}
\end{ndef}

\begin{ndef}[Vector Aleatorio]
    Sea $(\Omega, \mathcal{F}, P)$ un espacio de probabilidad. Un \textit{vector aleatorio}, $X = (X_1, \dots, X_p)$ es una función medible
    \begin{align*}
    X:(\Omega, \mathcal{F}) &\rightarrow (\mathbb{R}^p, \mathbb{B}^p) \\
    \omega &\mapsto X(\omega) = (X_1(\omega), \dots, X_p(\omega))
    \end{align*}
    cuya condición de medibilidad es \(X^{-1}(B) \in \mathcal{F}\) para todo \(B \in \mathbb{B}^p\).
\end{ndef}

\begin{ndef}[Probabilidad inducida]
    La \textit{probabilidad inducida} por un vector aleatorio $X = (X_1, \dots, X_p)$ a partir de $P$ se define como
    \[
    P_X[B] := P[X^{-1}(B)]
    .\] para todo \(B \in \mathbb{B}^p\).
\end{ndef}

\begin{ndef}[Función de distribución]
    Se define la \textit{función de distribución} asociada a $P_X$ como
    \begin{align*}
    F_X:\mathbb{R}^p &\rightarrow [0,1] \\
    x=(x_1,\dots,x_p) &\mapsto P_X[X_1 \leq x_1, \dots, X_p \leq x_p]
    .\end{align*}
\end{ndef}

\begin{ndef}[Función de densidad]
    Si existe una función $f_X$ que sea integrable en el sentido de Lebesgue y tal que
    \[
    F_X(x) = \int^{x_1}_{-\infty} \dots \int^{x_p}_{-\infty} f_X(u_1, \dots,  u_p) du_1 \dots du_p \quad \text{para todo } x \in \mathbb{R}^p
    ,\]
    diremos que $f_X$ es la \textit{función de densidad} asociada a  $F_X$. En el caso en que $f_X$ sea continua, se puede escribir como:
    \[
    % TODO Añadir lo de las derivadas
    f_X(x) = \frac{\partial^p}{\partial x_1 \dots \partial x_p} F_X(x) \quad \text{para todo } x \in \mathbb{R}^p
    .\]
\end{ndef}

En general, dado un conjunto producto (también llamado conjunto rectangular) $B \in \mathbb{B}^p$ con
\[
    B = B_1 \times \dots \times B_p,\quad B_j \in \mathbb{B}, \quad j = 1, \dots, p
.\]
se tiene que
\[
P_X[B] \neq P_{X_1}[B_1] \dots  P_{X_p}[B_p].
\]

\begin{ndef}
    Si en las condiciones anteriores se da la igualdad para todo conjunto producto en \(\mathbb{B}^p\), se dice que \(X_1, \dots, X_p\) son \textit{independientes}.

    Equivalentemente, esto ocurre si y solo si \(
    F_X (x) = F_{X_1} (x_1) \cdots F_{X_p}(x_p)\,\) para todo \(x \in \mathbb{R}^p\).

    Por último, esto ocurre en el caso continuo si y solo si \(f_X (x) = f_{X_1}(x_1) \cdots f_{X_p}(x_p)\,\) para todo \(x \in \mathbb{R}^p\), salvo, a lo sumo, en un conjunto de medida de Lebesgue nula.
\end{ndef}

\begin{ndef}[Función característica]
    La \textit{función característica} de un vector aleatorio \(X = (X_1,\dots,X_p)\) se define como \[\psi_X(t)=E[e^{it^TX}], \quad t\in \mathbb{R}^p.\]
\end{ndef}

\begin{nth}[Unicidad]
  La función característica de un vector aleatorio determina de forma única su distribución.
\end{nth}

\begin{nprop}
  Las componentes de \(X=(X_1,\dots,X_p)^T\) son independientes si, y solo si \[\psi_X(t)=E[e^{it^TX}] = \prod_{k=1}^p\psi_{X_k}(t_k).\]
\end{nprop}

\begin{nprop}
  Si las componentes de \(X=(X_1,\dots, X_p)\) son independientes, entonces la función característica de la variable \(Y=\sum_{k=1}^p X_k\) es \[\psi_Y(t)=E[e^{it^TY}] = \prod_{k=1}^p\psi_{X_k}(t).\]
\end{nprop}
