
\section{Espacios métricos}
Sea $X$ un conjunto no vacío.
\begin{ndef}[Distancia]
  Una \textbf{distancia} en $X$ es una aplicación $X \times X \to \R$ tal que:
  \begin{enumerate}[label=D{{\arabic*}}]
    \item $d(x,y)=0 \Leftrightarrow x=y \quad \forall x,y \in X$
    \item (Simetría) $d(x,y)=d(y,x) \quad \forall x,y \in X$
    \item (Desigualdad triangular) $d(x,y) \le d(x,z)+d(z,y) \quad \forall x,y,z \in X$
  \end{enumerate}
\end{ndef}
\begin{note}
  La propiedad $d(x,y) \geq 0 \ \forall x,y \in X$ se obtiene a partir de la definición dada de distancia.
\end{note}
\begin{proof}
  $0 \stackrel{D1}{=} d(x,x) \stackrel{D3}{\leq} d(x,y)+d(y,x) \stackrel{D2}{=} d(x,y)+d(x,y)=2d(x,y) \implies \implies d(x,y) \geq 0$
\end{proof}

\begin{exmp}
  Si $d$ es distancia en $X$ y $r>0$, entonces $rd: X \times X \to \R$ tal que $(rd)(x,y) = r \cdot d(x,y)$ es una distancia en $X$.
\end{exmp}
\begin{exmp}
  Tomemos $X=\R$. Ahora definimos $d(x,y)=|x-y| \ \forall x,y \in \R$. Entonces $d$ es una distancia en $\R$.
\end{exmp}
\begin{proof}
  Para demostrarlo, veamos que se cumplen las tres condiciones necesarias de una distancia:
  \begin{enumerate}[label=D{{\arabic*}}]
    \item $d(x,y)=0 \Leftrightarrow |x-y|=0 \Leftrightarrow x=y \quad \forall x,y \in \R$
    \item $d(x,y)=|x-y|=|0-(x-y)|=|y-x|=d(y,x) \quad \forall x,y \in \R$
    \item $d(x,y)=|x-y|=|x-z+z-y| \le |x-z|+|z-y|=d(x,z)+d(z,y) \quad \forall x,y,z \in \R$
  \end{enumerate}
\end{proof}
\begin{exmp}
  Sea $X$ un conjunto no vacío. La \textbf{distancia discreta} en $X$ es la aplicación $d:X \times X \to \R$ definida por:
  \[ d(x,y)=
    \begin{cases}
      0, & \text{si } x=y \\ 1,& \text{si } x\neq y
    \end{cases}\]
\end{exmp}
\begin{proof}
  Verifiquemos que cumple las tres propiedades:
  \begin{enumerate}[label=D{{\arabic*}}]
    \item $d(x,y)=0 \Leftrightarrow x=y \ \forall x,y \in X$ por la propia definición de $d$.
    \item $d(x,y)=d(y,x) \ \forall x,y \in X$ también por la definición dada, pues sólo depende de  que $x=y$ o $x \neq y$.
    \item Queremos probar que $d(x,y) \leq d(x,z)+d(z,y)$. Luego distinguimos dos casos:
          \begin{itemize}
            \item Si $d(x,y)=0 \implies 0 \leq d(x,z)+d(z,y)$, pues los valores que toman dichas distancias sólo pueden ser 0 o 1.
            \item Si $d(x,y)=1$, entonces $x \neq y$. Queremos ver que $1 \leq d(x,z)+d(z,y)$. Esta desigualdad se cumple si $d(x,z)=1$ o $d(z,y)=1$. Si no se cumpliera alguna de estas desigualdades, entonces $d(x,z)=d(z,y)=0$ y se tiene que $x=z=y$. Por tanto $x=y!!$, lo cual es una contradicción porque habíamos supuesto  que $x \neq y$.
          \end{itemize}
  \end{enumerate}
\end{proof}

Sea $\R^n$ el conjunto definido tal que $\R^n=\{(x_1,\cdots,x_n):x_i \in \R, \ i=1,\cdots,n\}$
\begin{ndef}[Norma]
  Una \textbf{norma} en $\R^n$ es una aplicación $||\cdot||: \R^n \to \R$ con las siguientes propiedades:
  \begin{enumerate}[label=N{{\arabic*}}]
    \item $||x||=0 \Leftrightarrow x=0 \quad \forall x \in \R^n$
    \item $||\lambda x||=|\lambda| \cdot ||x||\quad \forall x \in \R^n, \forall \lambda \in \R$
    \item $||x+y|| \le ||x||+||y|| \quad \forall x,y \in \R^n$
  \end{enumerate}
\end{ndef}
\begin{exmp}
  Tomando el producto escalar usual de $\R^n$, con $x,y \in \R^n$ \[\langle (x_1, \cdots, x_n), (y_1,\cdots,y_n)\rangle=x_1y_1+\cdots+x_ny_n\] su norma asociada es \[||x||=\langle x,x\rangle^{1/2}= \left(\sum^n_{i=1}x^2_i\right)^{1/2}=(x^2_1+\cdots+x^2_n)^{1/2}\]
  a la que llamamos \textbf{norma Euclídea} y denotaremos $||\cdot||$ o $||\cdot||_2$. Otras normas en $\R^n$ que no proceden del producto escalar son la \textbf{norma 1} $||x||_1=|x_1|+\cdots+|x_n|$ o la \textbf{norma infinito} $||x||_\infty = \max_{1 \leq i \leq n} \{|x_i|\}$.
\end{exmp}

\begin{nprop}
  Toda norma $||\cdot||$ en $\R^n$ induce una distancia $d$ tal que $d(x,y)=||x-y|| \ \forall x,y \in \R^n$.
\end{nprop}
\begin{proof} Probemos que se cumplen la propiedades de una distancia:
  \begin{enumerate}[label=D{{\arabic*}}]
    \item $d(x,y)=0 \Leftrightarrow ||x-y||=0 \stackrel{N1}{\Leftrightarrow} x-y=0 \Leftrightarrow x=y \quad \forall x,y \in \R^n$
    \item $d(x,y)=||x-y||=||(-1)\cdot(x-y)||\stackrel{N2}{=} |-1|\cdot||y-x||=||y-x||=d(y,x) \quad \forall x,y \in \R^n$
    \item $d(x,y)=||x-y||=||x-z+z-y|| \le ||x-z||+||z-y||=d(x,z)+d(z,y) \quad \forall x,y,z \in \R^n$
  \end{enumerate}
  Por tanto, $d$ es una distancia en $\R^n$.
\end{proof}
\begin{exmp}
  En el caso del ejemplo anterior, $||\cdot||_2$ induce a $d_2$, $||\cdot||_1$ a $d_1$ y $||\cdot||_\infty$ a $d_\infty$, tal y como indica la proposición.
\end{exmp}

\begin{properties}
  La distancia discreta en $\R^n$ no procede de ninguna norma, es decir, no existe ninguna norma en $\R^n$ cuya distancia asociada sea la discreta.
\end{properties}
\begin{proof}
  Sea $d$ una distancia que procede de una norma $||\cdot||$ en $\R^n$. Tomamos $x=0$, $y\neq0$. Como $y \neq 0 \implies ||y|| \neq 0$. Ahora, tomemos $\lambda\in\R$ y vemos que $d(x,\lambda y) = ||\lambda y|| = |\lambda|\cdot||y||\neq 0$.
  Si $|\lambda| > \frac{1}{||y||}$, entonces $d(x,\lambda y) = |\lambda| \cdot ||y|| > \frac{1}{||y||} \cdot ||y||=1$. Luego $d(x,\lambda y) > 1$. Por tanto, $d$ no puede ser la distancia discreta, ya que en la distancia discreta, dos puntos están a lo sumo a distancia 1.
\end{proof}
Con esto queda probado que si una distancia $d$ en $\R^n$ procede de una norma, existen puntos a distancia arbitrariamente grande.

\begin{ndef}
  Un \textbf{espacio métrico} es un par $(X,d)$ formado por un conjunto no vacío $X$ y una distancia $d$ en $X$.
\end{ndef}
\begin{ndef}
  Diremos que un espacio métrico $(X,d)$ es \textbf{acotado} si existe $M>0$ tal que $d(x,y)\leq M \ \forall x,y \in X$.
\end{ndef}
\begin{exmp}
  Un espacio métrico cuya distancia asociada es la discreta es un espacio métrico acotado, mientras que un espacio métrico sobre $\R^n$ y una distancia inducida de una norma, es un espacio métrico no acotado.
\end{exmp}

\begin{ndef}[Bolas en un espacio métrico]
  Sea $(X,d)$ un espacio métrico, con $a \in X$,
  \begin{enumerate}
    \item La bola abierta de centro $a$ y radio $r$ es el conjunto $B(a,r)= \{x \in X : d(x,a) \leq r\}$
    \item La bola cerrada de centro $a$ y radio $r$ es el conjunto $\overline{B}(a,r)= \{x \in X : d(  x,a) \leq r\}$
    \item La esfera
  \end{enumerate}

\end{ndef}
\begin{exmp}
  Ni idea
\end{exmp}
\begin{exmp}
  En $\R^2$ con la distancia Euclídea, las bolas descritas serían de la siguiente forma en el espacio métrico $(\R^2,d_2)$:
  DIBUJO
\end{exmp}
\begin{exmp}
  Sea $(X,d)$ un espacio métrico con $d$ la distancia discreta. Sea $a \in X,\ r>0$. La bola abierta $B$ sería
  \[B(a,r)=
    \begin{cases}
      \{a\}, & r \leq 1 \\ X,& r>1
    \end{cases}\]
  SEGUIR EL EJEMPLO
\end{exmp}

\begin{properties} Estas son algunas propiedades de las bolas:
  \begin{enumerate}
    \item $a \in B(a,r),\ a \in \overline{B}(a,r) \quad \forall a \in X,\ \forall r>0$
    \item $B(a,r) \subset \overline{B}(a,r) \quad \forall a \in X,\ \forall r>0$
    \item Si $r \leq s$, entonces $B(a,r) \subset B(a,s)$ (también $\overline{B}(a,r) \subset \overline{B}(a,s)$) $\quad \forall a \in X$
    \item Si $0<\lambda<1$, entonces $B(a,\lambda r) \subset B(a,r) \quad \forall a \in X,\ \forall r>0$
  \end{enumerate}
\end{properties}

\begin{ndef}
  E
\end{ndef}
\begin{properties}
  Las bolas abiertas en un espacio métrico son conjuntos abiertos.
\end{properties}
\begin{proof}
  Sea $U=B(a,r),\ a \in X,\ r>0$. Sea $x \in B(a,r) \implies d(x,a)<r \implies s=r-d(x,a)>0$. Vamos a ver que $B(x,s) \subset B(a,r)$ usando la desigualdad triangular. Entonces tenemos que $z \in B(x,s) \implies d(x,z)<s \implies d(a,z) \leq d(a,x) + d(x,z)<d(a,x)+s=d(a,x)+(r-d(x,a))=r \implies z \in B(a,r) \implies B(x,s) \subset B(a,r)$. Queda probado que para todo punto $x \in B(a,r)$, existe $s>0$ que depende de $x$ al que $B(x,s) \subset B(a,r)$. Por tanto, $B(a,r)$ es un conjunto abierto.
\end{proof}
\begin{exmp}
  En un espacio métrico discreto, los puntos son conjuntos abiertos puesto que $B(a,r)=\{a\}$ si $r \geq 1$.
\end{exmp}
\begin{exmp}
  $X$ es un conjunto abierto, puesto que si $x \in X$ y $r>0$ es arbitrario, entonces $B(x,r) \subset X$.
\end{exmp}
\begin{nth}
  Sea $(X,d)$ un espacio métrico, y sea $T=$. Entonces:
  \begin{enumerate}
    \item $\varnothing ,X$
    \item 2
    \item $U_1, \cdots, U_k \in T \implies U_1 \cap \cdots \cap U_k \in T$
  \end{enumerate}
\end{nth}
\begin{proof}
  Dem
\end{proof}
D
\\ $\ldots$
\begin{exmp}
  Ejemplo intersección de abiertos no  tiene por qué ser abierto:
  Sea $(\R,d)$ un espacio métrico de los reales con la distancia usual, fijamos $x=0,\ \forall r>0,\ (-r,+r) \in T$. Si tomamos $\cap_{r>0}(-r,r)=\{0\}$. El $\{0\}$ no es abierto porque no existe $s>0$ tal que $B(0,s)=(-s,s) \subset \{0\}$.
\end{exmp}

\begin{ndef}
  Decimos que dos distancias son \textbf{métricamente equivalentes} para un conjunto $X$ y $d,\ d'$ distancias en $X$ si existe $\alpha,\beta>0\ /\ \alpha d \leq d' \leq \beta d$.
\end{ndef}
\begin{properties}
  Dos distancias métricas equivalentes tienen los mismos conjuntos abiertos.
\end{properties}
\begin{properties}
  Con dos distancias métricamente equivalentes, las sucesiones convergentes son las mismas y tienen los mismos límites. Es decir la noción de sucesión convergente no depende de la distancia. Matemáticamente,
  $x_i \to x \Leftrightarrow \forall U \in T\ /\ x\in U,\ \exists b \in \N\ /\ x_i \in U\ \forall i \geq i_0$.
\end{properties}
\begin{exmp}
  En $\R^n$, $d_1,\ d_2$ y $d_\infty$ son distancias métricamente equivalentes.
\end{exmp}
DIBUJO

\begin{nprop}
  Sea $(X,d)$ un espacio métrico. Entonces todo conjunto abierto no vacío es unión de bolas abiertas.
\end{nprop}
\begin{proof}
  Sea $U \in T$. Por definición de conjunto abierto, para todo $x \in U,\ \exists r_x>0\ /\ B(x,r_x) \subset U$. Veamos que $U=\cup_{x \in U}B(x,r_x).$ \\ $B(x,r_x) \subset U\ \forall x\in U \implies \cup_{x\in U}B(x,r_x) \subset U$, quedando demostrada la primera implicación. Por otro lado, tomamos $z \in U \implies B(z,r_z) \subset U \implies z \in B(z,r_z) \subset \cup_{x \in U}B(x,r_x) \implies U \subset \cup_{x \in U}B(x,r_x)$.
\end{proof}

Ahora nos preguntamos, ¿puede ser un abierto $U \neq \varnothing $ unión de bolas cerradas? La respuesta es sí,  veámoslo: \\

Si $x \in U \ \exists r_x>0\ /\ B(x,r_x) \subset U$. Pero $\overline{B}(x, \frac{r_x}{2}) \subset B(x,r_x)$. Luego tomando $z \in \overline{B}(x,\frac{r_x}{2}) \implies d(x,z) \le \frac{r_x}{2} \le r_x \implies < \in B(x,r_x)$. Razonando como antes llegamos a que $U = \bigcup_{x \in U} \overline{B}(x,\frac{r_x}{2})$.

\begin{ndef}
  Sea $(X,d)$ un espacio métrico. Diremos que $F \subset X$ es cerrado si su complementario $F^c$, $X \setminus F$, es un conjunto abierto. \[ X\setminus F = \{x \in X\ /\ x \notin F\}\]
  \begin{enumerate}
    \item $\varnothing ^c=X \in T$, $X^c=\varnothing \in T\implies \varnothing ,X$ son conjuntos cerrados.
    \item Sea $\{F_i\}_{i \in I}$ una familia de conjuntos cerrados, entonces $(\cap_{i \in I}F_i)^c = \cup_{i \in I}F_i^c \in T \implies \cap_{i \in I}F_i$ es cerrado.
    \item Sea $F_1,\ldots,F_k$ una familia finita de conjuntos cerrados, entonces $(F_1 \cup \ldots \cup F_k)^c=F_1^c \cap \ldots \cap F_k^c \in T \implies F_1 \cup \ldots \cup F_k$ es un conjunto cerrado.
  \end{enumerate}
  EJEMPLO
  \begin{exmp}
    Otro ejemplo sería $(-1,1)=\cup_{n \in \N}[-1+\frac{1}{n},1-\frac{1}{n}]$, donde la primera parte de la igualdad no es cerrada mientras que la segunda sería una familia de cerrados $(\overline{B}(0,1-\frac{1}{n}))$.
  \end{exmp}
\end{ndef}

\begin{properties}
  Las bolas cerradas en un espacio métrico son conjuntos cerrados.
\end{properties}
\begin{proof}
  Sean $x \in X$, $r>0$. Para que $\overline{B}(x,r)$ sea un conjunto cerrado, vamos a comprobar que $X \setminus \overline{B}(x,r)$ es un conjunto abierto. $z \in \overline{B}(x,r) \Leftrightarrow d(x,z) \leq r$. Ahora como $z \in X \setminus \overline{B}(x,r) \Leftrightarrow z \notin  \overline{B}(x,r) \Leftrightarrow d(z,x)>r$. Entonces $X \setminus \overline{B}(x,r) = \{z \in X\ /\ d(z,x)>r\}$. Veamos que es abierto. Tomamos $y \in X \setminus \overline{B}(x,r) \implies d(x,y)>r$. Definimos $s=d(x,y)-r>0$ y veamos que $B(y,s) \subset X \setminus \overline{B}(x,r)$. Sea $z \in B(y,s) \implies d(z,y)<s=d(x,y)-r$. Queremos ver que $z \in X \setminus \overline{B}(x,r)$. Es decir, que $d(x,z)>r$. Finalmente, \[d(x,y) \le d(x,z) + d(z,y) < d(x,z) + s= d(x,z)+d(x,y)-r \implies d(x,z)>r\]
\end{proof}

\begin{properties}
  Sea $(X,d)$ un espacio métrico y $x,y \in X$ con $x \neq y$. Existen entonces dos conjuntos abiertos $U_x,\ U_y$ tales que $x \in U_x,\ y \in U_y$ y $U_x \cap U_y \neq \varnothing $.
\end{properties}
\begin{ndef}[Hausdorff]
  Dicha propiedad se conoce como la propiedad $T_2$ o \textit{Hausdorff}.
\end{ndef}
\begin{proof}
  Sean $x \neq y$. Entonces $r=d(x,y)>0$ y tomamos $U_x = B(x,\frac{r}{2}),\ U_y=B(y,\frac{r}{2})$. Veamos que $U_x \cap U_y=\varnothing $. Para comprobar que $U_x \cap U_y=\varnothing $, razonemos por contradicción suponiendo que $U_x \cap U_y \neq \varnothing $. Tomemos $z \in U_x \cap U_y$. Entonces $d(x,y) \le  d(x,z) + d(z,y) < \frac{r}{2} + \frac{r}{2} = r$, ya que $z \in U_x=B(x,\frac{r}{2}) \implies d(x,z)<\frac{r}{2}$ y $z \in U_y=B(y,\frac{r}{2}) \implies d(z,y)<\frac{r}{2}$. Esta contradicción demuestra que no existe $z \in U_x \cap U_y$. Por tanto, $U_x \cap U_y = \varnothing $.
\end{proof}

\newpage
\section{Espacios Topológicos}

\begin{ndef}[Topología]
  Sea $X$ un conjunto no vacío. Una \textbf{topología} en $X$ es una familia de $T$ de subconjuntos de $X$ ($T \subset P(x)$) que verifica:
  \begin{enumerate}
    \item $\varnothing ,X \in T$
    \item $\{U_i\}_{i \in I} \implies \bigcup_{i \in I} U_i \in T$
    \item $U_1,\ldots,U_k \in T \implies U_1 \cap \ldots \cap U_k \in T$
  \end{enumerate}
  A los elementos de $T$ los llamamos conjuntos abiertos de la topología $T$.
\end{ndef}
\begin{ndef}[Espacio topológico]
  Un \textbf{espacio topológico} $(X,T)$ es un conjunto $X$ no vacío en una topología $T$ en $X$.
\end{ndef}
\begin{exmp}
  Sea $(X,d)$ un espacio métrico, $T=\{U \subset  X / \forall x \in U, \exists r>0$ tal que $B(x,r) \in U\} \cup \{\varnothing \}$. $T_d$ es una topología en $X$ y la llamaremos la topología asociada a la distancia $d$.
\end{exmp}
Sin embargo, ahora nos surge un problema. Dado un espacio topológico $(X;T)$, ¿existe una distancia en $d$ tal que $T_d=T$? Pues en general no.
\begin{ndef}[Espacio topológico metrizable]
  Un espacio topológico es \textbf{metrizable} si existe una distancia $d$ en $X$ tal que $T_d=T$.
\end{ndef}
\begin{exmp}
  Sea $X \neq \varnothing $, $T_t=\{\varnothing ,X\}$ la \textbf{topología trivial}, que es la topología con la menor cantidad posible de conjuntos. Si $T$ es otra topología en $X$, entonces $T_t \in T$ (ya que $U \in T_t \implies U \in T$).
\end{exmp}
\begin{exmp}
  Sea $X \neq \varnothing $, $T_D = P(X) = \{U / U \in X\}$ la \textbf{topología discreta}. Es la topología con el mayor número posible de conjuntos. Si $T$ es una topología cualquiera en $X$, entonces $T \subset P(X)=T_D$.
\end{exmp}
\begin{properties}
  Sea $X \neq \varnothing $, $d$ la distancia discreta y $T_D$ la topología discreta en $X$. Entonces $T_d=T_D$ y por tanto $(X,T_D)$ es metrizable.
\end{properties}
\begin{proof}
  Queremos ver que $T_d = T_D$. La inclusión $T_d \subset T_D$ se sigue de que toda topología de $X$ está contenida en $T_D$. Falta ver que $T_D \in T_d$. Sea $U \in T_D$. Si $U=\varnothing \implies U \in T_d$ porque $T_d$ es topología si $U \neq \varnothing $. \[U = \bigcup_{x \in U} \{x\} = \bigcup_{x \in U} B(x,\frac{1}{2}) \in T_d \]
\end{proof}
¿Cuándo coinciden $T_t$ y $T_D$? Solo si $X=\{p\}$.\\
¿Cuántas topologías hay en un conjunto con un elemento? $T=\{\varnothing ,X\}$

\begin{exmp}
  Sea $X \neq \varnothing $. Entonces llamaremos \textbf{topología de los complementos finitos} o topología cofinita a $T_{CF}=\{U \in X\ /\ U^c$ es finito $\} \cup \{\varnothing \}$.
\end{exmp}
\begin{proof}
  Veamos que cumple las tres propiedades de una topología:
  \begin{enumerate}
    \item $\varnothing \in T_{CF}$, $X^c = \varnothing $ (0 elementos) $\implies X \in T_{CF}$
    \item $\{U_i\}_{i \in I} \subset T_{CF}$. Si $U_i = \varnothing \ \forall i \in I \implies \bigcup_{i \in I} U_i = \varnothing \in T_{CF}$. Supongamos que $\exists i_0 \in I\ /\ U_{i_0} \neq \varnothing \implies U^c_{i_0}$ es finito. Entonces \[U_{i_0} \subset \bigcup_{i \in I} U_i \implies \left(\bigcup_{i \in I} U_i \right )^c \subset U_{i_0}^c \implies \left( \bigcup_{i \in I} U_i \right) \in T_{CF} \] El cual es finito por ser subconjunto de un conjunto finito.
    \item $U_1,\ldots,U_k \in T_{CF} \implies U_1 \cap\ldots\cap U_k \in T_{CF}$. Si algún $U_{i_0} = \varnothing \implies U_1 \cap \ldots \cap U_{i_0} \cap \ldots \cap U_k = \varnothing \in T_{CF}$. Si ningún $U_i \neq \varnothing \ \forall i \in I \implies (U_1 \cap \ldots \cap U_k)^c = U_1^c \cup \ldots \cup U_k^c \in  T_{CF}$ (la unión de conjuntos finitos es finito).
  \end{enumerate}
\end{proof}

Si $X$ es finito, $T_{CF}$ coincide con la topología discreta $T_D$, pues al ser $X$ finito, cualquier complementario es abierto en la topología por lo que $T_{CF} = P(X) = T_D$. \\

Si $X$ es infinito, $(X,T_{CF})$ no es metrizable, ya que no existe una distancia $d$ en $X$ tal que $T_d=T_{CF}$.
\begin{proof}
  Si $(Y,d)$ es un espacio métrico y $x \neq y$, $x,y \in Y \implies \exists U_x \in T_d, U_y \in T_d\ /\ U_x \cap U_y = \varnothing $. En un espacio métrico siempre podemos encontrar dos abiertos $U,V \in T_d$ tales que $U \cap V = \varnothing $.
\end{proof}

\begin{ndef}
  Un espacio topológico $(X,T)$ es \textbf{Hausdorff} (o verifica el axioma de separabilidad $T_2$ o en $T_2$) cuando, para todo par de puntos $x,y \in X$, con $x \neq y$, existen dos abiertos $U_x,U_y \in T$ tales que:
  \begin{enumerate}
    \item $x \in U_x, y \in U_y$
    \item $U_x \cap U_y = \varnothing $
  \end{enumerate}
\end{ndef}

\begin{exmp}
  Si $(X,d)$ es un espacio métrico, entonces $(X,T_d)$ es Hausdorff.
\end{exmp}
\begin{exmp}
  Si $X$ es infinito, $(X,T_{CF})$ no es Hausdorff.
\end{exmp}
\begin{exmp}
  Si $(X,T_D)$ es un espacio discreto, entonces es Hausdorff: si $x \neq y$, $U_x=\{x\}, U_y=\{y\} \in T_D$. Como $x \neq y \implies \{x\} \cap \{y\} = \varnothing \implies U_x \cap U_y = \varnothing $.
  Otra forma de ver que $(X,T_D)$ es Hausdorff es tener en cuenta que $(X,T_D)$ es metrizable. Si $d$ es la distancia discreta en $X$, entonces$T_d=P(X)=T_D$.
\end{exmp}

\begin{exmp}
  Sea $X \neq 0$. Supongamos que $X$ es, al menos, numerable. La \textbf{topología de los complementos numerables} es $T_{CN}=\{U \subset X / U^c$ es numerable$\} \cup \{\varnothing \}$.
\end{exmp}
Si $X$ es numerable, entonces $T_{CN}=T_D$ (Si $U \subset  X$ es cualquier conjunto, $U^c \subset X$. Como $X$ es numerable y $U^c \subset X / U^c$ también es numerable $\implies U \in T_{CN} \implies T_{CN} = T_D$).

Si $X$ es infinito, ¿es $(X, T_{CN})$ metrizable?
\begin{enumerate}
  \item Si $X$ es numerable $\implies T_{CN} = T_D \implies (X,T_{CN})$ es metrizable usando la distancia discreta.
  \item Si $X$ no es numerable $\implies (X,T_{CN})$ no es Hausdorff (Si $U,V \neq \varnothing \implies U \cap V \neq \varnothing $).
\end{enumerate}

\begin{ndef}
  Sean $T_1,T_2$ topologías en $X$. Diremos que $T_1$ es \textbf{más fina} que $T_2$ si $T_2 \subset T_1$. También diremos que $T_2$ es \textbf{más gruesa} que $T_1$.
\end{ndef}

\begin{ndef}
  Sea $(X,T)$ un espacio topológico. Diremos que $F \subset X$ es \textbf{cerrado} si $F^c$ es abierto. A la familia de todos los conjuntos cerrados de $(X,T)$ la llamaremos $C_T$.
\end{ndef}

\begin{properties}
  Sea $(X,T)$ un espacio topológico. Entonces:
  \begin{enumerate}
    \item $\varnothing ,X \in C_T$
    \item Si $\{F_i\}_{i \in I} \subset C_T \implies \bigcap_{i \in I} F_i \in C_T$
    \item Si $F_1,\ldots,F_k \in C_T \implies F_1 \cup \cdots \cup F_k \in C_T$
  \end{enumerate}
\end{properties}
\begin{proof}
  La demostración es idéntica a la dada para espacios métricos.
\end{proof}
\begin{note}
  Si tenemos un conjunto $X$ y una familia $C \subset P(X)$ que cumple
  \begin{enumerate}
    \item $\varnothing ,X \in C$
    \item Si $\{F_i\}_{i \in I} \subset C \implies \bigcap_{i \in I} F_i \in C$
    \item Si $F_1,\ldots,F_k \in C \implies F_1 \cup \cdots \cup F_k \in C$
  \end{enumerate}
  entonces existe una única topología $T$ en $X$ tal que $G=C$.
\end{note}
\begin{proof}
  Definimos $T=\{F^c / {F \in C}\}$. Como $C \subset P(X)$ verifica las tres propiedades, pasando al complementario, $T$ verifica las propiedades de los abiertos (habría que mostrarlo).
\end{proof}
\begin{exmp}
  Estos son algunos ejemplos:
  \begin{enumerate}
    \item $(X,T_{CF}) \quad C_{T_{CF}}=\{F \subset X / F$ es finito $\} \cup \{X\}$.
    \item $(X,T_D) \quad C_{T_D}= P(X)=T_D$.
    \item $(X,T_t) \quad C_{T_t}= \{\varnothing ,X\}=T_t$.
    \item Sea $(X,d)$ un espacio métrico, todos los puntos son conjuntos cerrados.
  \end{enumerate}
\end{exmp}

\begin{properties}
  En un espacio topológico Hausdorff todo punto es cerrado.
\end{properties}
\begin{proof}
  Sea $x \in X$. $\{x_0\} \in C_T \Leftrightarrow X \setminus \{x_0\}$ es abierto. Sea $y \in X \setminus \{x_0\} \implies y \neq x_0 \implies U_y,U_{x_0}^y = \varnothing \implies x_0 \not\in U_y \implies U_y \subset X \setminus \{x_0\}$. $X \setminus \{x_0\} = \bigcup_{y \neq x_0} U_y \in T \implies \{x_0\} \in C_T$.
\end{proof}
\begin{exmp}
  $X$ infinito con $(X,T_{CF})$ \textbf{no} es Hausdorff. Los puntos son conjuntos cerrados.
\end{exmp}

\begin{ndef}[Base]
  Sea $(X,T)$ un espacio topológico, una \textbf{base} de la topología $T$ es una familia $B \subset T$ (los elementos de $\mathcal{B}$ son conjuntos abiertos) con la propiedad de que todo conjunto abierto puede expresarse como unión de elementos de $\mathcal{B}$.
  \begin{itemize}
    \item $\forall U \in T,\ \exists \{B_i\}_{i \in I} \subset \mathcal{B}\ /\ U=\bigcup_{i \in I} B_i$
    \item $\mathcal{B} \subset T$
  \end{itemize}
\end{ndef}
\begin{exmp}
  Varios ejemplos de bases serían:
  \begin{enumerate}
    \item Sea $(X,d)$ un espacio métrico. $\mathcal{B}=\{B(x,r)\ /\ x \in X, r>0\}$ es una base de $T_d$.
    \item MAS EJEMPLOS
  \end{enumerate}
\end{exmp}

\begin{properties}
  Sea $(X,T)$ un espacio topológico, $\mathcal{B}$ base de $T$. Entonces:
  \begin{enumerate}
    \item $\forall x \in X, \exists B \in \mathcal{B}$ tal que $x \in B$.
    \item $\forall B_1,B_2 \in \mathcal{B},\ \forall x \in B_1 \cap B_2,\ \exists B_3 \in \mathcal{B}\ /\ x \in B_3 \subset B_1 \cap B_2$.
  \end{enumerate}
\end{properties}

\begin{nth}
  Sea $X$ un conjunto, $\mathcal{B} \subset P(X)$ una familia de subconjuntos de $X$ tal que
  \begin{enumerate}
    \item $\forall x \in X$, $\exists B \in \mathcal{B}$ tal que $x \in B$.
    \item $\forall B_1,B_2 \in \mathcal{B}$, $\forall x \in B_1 \cap B_2$, $\exists B_3 \in \mathcal{B}\ /\ x \in B_3 \subset B_1 \cap B_2$.
  \end{enumerate}
  Entonces existe en $X$ una única topología $T$ tal que $\mathcal{B}$ es una base de $T$.
\end{nth}
\begin{proof}
  Definimos $T=\{U \subset T / \forall x \in U, \exists B \in \mathcal{B}$ con $ x \in B \subset U\} \cup \{\varnothing \}$. Entonces:
  \begin{enumerate}
    \item $\varnothing \in T$ (por definición). $X \in T$ por (1) ($\forall x \in X\ \exists B \in \mathcal{B}$ con $x \in B \subset X$).
    \item Sea $\{U_i\}_{i \in I} \subset T$. ¿$\bigcup_{i \in I} U_i \in T$? Si $\bigcup_{i \in I} U_i = \varnothing , \varnothing \in T$. Si $x \in \bigcup_{i \in I} U_i \implies \exists i_0 \in I / x \in U_{i_0}$. Como $U_{i_0} \in T$, $\exists B \in \mathcal{B}$ ESTE Y OTRO PUNTO.
  \end{enumerate}
  Veamos ahora que $\mathcal{B}$ es base de $T$. Tenemos que comprobar que
  \begin{enumerate}
    \item $B \subset T$.
    \item Todo $U \in T \setminus \{\varnothing \}$ es unión de elementos de $\mathcal{B}$.
  \end{enumerate}
  Tomamos $T=\{U \subset T / \forall x \in U, \exists B \in \mathcal{B}$ con $ x \in B \subset U\} \cup \{\varnothing \}$.
  \begin{itemize}
    \item Sea $B \in \mathcal{B}$. $\forall x \in B,\ x \in B \subset B$ (tomando $B=U$) $\implies \mathcal{B} \subset T$.
    \item Sea $U \in T,\ U \neq \varnothing $. Sea $x \in U \implies \exists B_x \in \mathcal{B}x \in U \implies \exists B_x \in \mathcal{B}\ /\ x \in B_x \subset U \implies U = \bigcup_{x \in U} B_x \implies U$ es unión de elementos de $\mathcal{B}$.
  \end{itemize}
  Veamos por último la unicidad de $T$. Sea $T'$ otra topología en $X$ tal que $\mathcal{B}$ es base de $T'$. Veamo que $T=T'$. \[T \subset T' \implies \exists \{B_i\}_{i \in I} \subset \mathcal{B}\ /\ U = \bigcup_{i \in I} B_i \implies U = \bigcup_{i \in I} B_i \in T' \]
  $T \subset T'$ se demuestra igual (para probar $T \subset T'$ solo hemos usado que $\mathcal{B}$ es base de $T,T'$).
\end{proof}
\begin{exmp}
  Topología de \textbf{Sorgenfrey} o del \textbf{límite inferior}. $(\R,T_S)$, cuya base es $\mathcal{B}=\{[a,b)\ /\ a,b \in \R, a<b\} $.
\end{exmp}
PEDIR VIERNES

\begin{properties}
  La topología usual es distinta de la topología de Sorgenfrey y de la topología de Kuratowski, es decir, $T_u \neq T_S$, $T_u \neq T_K$.
\end{properties}
\begin{proof}
  Como $T_u \subset T_S$, buscamos $u \in T_S \setminus T_u$ tal que $u = [0,1) \in \mathcal{B} \subset T_S$. Veamos que $u \not\in T_u$. Supongamos que $u \in T_u \implies \exists \{B_i\}_{i \in J} \subset \mathcal{B}_u / 0 \in u = \bigcup_{i \in I}B_i \implies \exists i_0 \in I / 0 \in B_{i_0} \subset  \bigcup_{i \in I} B_i = u$. Luego $B_{i_0} = (a,b),\ a<b$. Entonces $0 \in (a,b) \subset [0,1)$, ya que $a<0$. Por último, tomamos $z \in (a,0) \implies z \in (a,b)$ y $z \not\in [0,1) \implies (a,b) \not\subset [0,1)!!$. Esto es una contradicción, pues suponíamos que $u \in T_u$. Luego $T_u \neq T_S$. \\
  Por otro lado, tenemos que encontrar $u \in T_K \setminus T_u$. Tomamos $u = (-1,1) \setminus K \in T_K$. Supongamos que $u \in T_u \implies \exists \{B_i\}_{i \in I} \subset \mathcal{B}_u\ /\ u = \bigcup_{i \in I}$ ACABAR
\end{proof}
\begin{ncor}
  Sean $\mathcal{B}, \mathcal{B}'$ bases de $T,T'$. Son equivalentes:
  \begin{enumerate}
    \item $T=T'$.
    \item \begin{itemize}
            \item $\forall B \in \mathcal{B}, \forall x \in B, \exists B' \in \mathcal{B}' / x \in B' \subset B \quad (T \subset T')$.
            \item $\forall B' \in \mathcal{B}', \forall x \in B', \exists B \in \mathcal{B} / x \in B \subset B' \quad (T' \subset T)$.
          \end{itemize}
  \end{enumerate}
\end{ncor}

\begin{lema}
  Si $X$ es un conjunto y $\{T_i\}_{i \in I}$ es una familia de topologías en $X$. Entonces \[T=\bigcap_{i \in I} T_i = \{U \subset X\ /\ U \in T_i \forall i \in I\} \] es una topología en $X$.
\end{lema}
\begin{proof}
  demostración
\end{proof}

\begin{nprop}
  Sea $S \subset P(X)$. Existe una topología $T(S)$ (topología generada por S) en $X$ tal que $S \subset T(S)$. Además, si $T'$ es otra topología que contiene a $S$, entonces $T(S) \subset T'$. Es decir, $T(S)$ es la topología mś gruesa que contiene a S).
\end{nprop}
\begin{proof}
  proof
  ppppp
\end{proof}

SEMANA ANTERIOR (PUNTOS INTERIORES Y ADHERENCIA)

\begin{ndef}
  $x \in X$ es un \textbf{punto frontera} de $A$ si $\forall U \in N_x$, se tiene $U \cap A \neq \varnothing $, $U \cap A^c \neq \varnothing $.
\end{ndef}
\begin{ndef}
  $x \in X$ es un \textbf{punto de acumulación} de $A$ si $\forall U \in N_x$, se tiene $(U\setminus \{x\}) \cap A \neq \varnothing $.
\end{ndef}
\begin{ndef}
  $x \in X$ es un \textbf{punto aislado} de $A$ si $\exists U \in N_x$, se tiene $U \cap A = \{x\} $.
\end{ndef}

Entonces para $A \subset X$, denotaremos:
\begin{itemize}
  \item $int(A)$ o $A^o$ a los puntos interiores de $A$, es decir, el interior de $A$.
  \item $cl(A)$ o $\overline{A}$ a los puntos adherentes de $A$, es decir, la clausura de $A$.
  \item $fr(A)$ o $\delta A$ a los puntos frontera de $A$, es decir, la frontera de $A$.
  \item $A'$ a los puntos de acumulación de $A$.
  \item $ais(A)$ a los puntos aislados de $A$, es decir, al conjunto de puntos aislados de $A$.
\end{itemize}

\begin{ndef}
  $x \in X$ es \textbf{punto exterior} de $A$ si $x$ es punto interior de $A^c$, es decir, $\exists U \in N_x\ /\ U \subset A^c$. El conjunto de puntos exteriores de $A$ se denota por $ext(A)$.
\end{ndef}

\begin{properties}
  Dado un conjunto $A$, se tiene que $A^o \subset A \subset \overline{A}$.
\end{properties}
\begin{proof}
  \begin{itemize}
    \item $A^o \subset A$. Sea $x \in A^o \implies \exists U \in N_x\ /\ x \exists $ ACABAR
  \end{itemize}
\end{proof}
\begin{properties}
  $A^o \in T$. Además, si $U \in T$ y $U \subset A$, entonces $U \subset A^o$, es decir, $A^o$ es el mayor conjunto abierto contenido en $A$.
\end{properties}
\begin{proof}
  $A^o \in T$. Si $A^o = \varnothing $, entonces $A^o \in T$. Si $A^o \neq \varnothing $, tomamos $x \in A^o \implies \exists U \in N_{x}\ /\ U \subset A \implies \exists V \in N_x$ con $V \subset U\ /\ U \in N_y \ \forall y \in V \implies y \in A^o \ \forall y \in V \implies V \subset A^o \implies \exists W_x \in T \ /\ x \in W_x \subset V \subset A^o \implies A^o = \bigcup_{x \in A^o} W_x \in T$. Ahora, sea $U \in T\ /\ U \subset A$. Si $x \in U$, como $U \in N_x \implies x \in A^o \implies U \subset A^o$.
\end{proof}

\begin{properties}
  $\overline{A} \in C_T =$ \{\text{conjuntos cerrados en }$(X,T)\}$. Además, si $F \in C_T$ y $A \subset F$, entonces $\overline{A} \subset F$. Es decir, la clausura de $A$ es el menor conjunto cerrado que contiene a $A$.
\end{properties}
\begin{proof}
  $\overline{A} \in C_T \Leftrightarrow X \setminus \overline{A} \in T$. Sea $x \in X \setminus \overline{A} \Leftrightarrow x \not\in \overline{A} \Leftrightarrow \exists U \in N_x\ /\ U \cap A = \varnothing \Leftrightarrow U \subset X \setminus A \Leftrightarrow x \in int(X \setminus A)$. Por tanto, $X \setminus \overline{A} = int(X \setminus A) \implies X \setminus \overline{A} \in T \implies \overline{A} \in C_T$.
\end{proof}

\begin{ncor}
  Sea $A \subset X$ se cumple que:
  \begin{enumerate}
    \item $A \in T \Leftrightarrow A=A^o$
    \item $A \in C_T \Leftrightarrow A=\overline{A}$
  \end{enumerate}
\end{ncor}
\begin{proof}
  Demostremos ambas dobles implicaciones:
  \begin{enumerate}
    \item \fbox{$\Leftarrow$} Si $A=A^o$, como $A^o \in T \implies A \in T$. \\ \fbox{$\Rightarrow$} $A \in T$. Siempre $A^o \subset A$. Además $A^o$ es el mayor conjunto abierto contenido en $A$. Luego $A \in T$, $A \subset A \implies A \subset A^o$. Por tanto $A=A^o$.
    \item \fbox{$\Leftarrow$} Si $A=\overline{A}$ ACABAR
  \end{enumerate}
\end{proof}

\begin{properties}
  De lo dicho anteriormente caben destacar las siguientes propiedades:
  \begin{enumerate}
    \item $\overline{A}=A^o \cup \delta A$, $A^o \cap \delta A = \varnothing$
    \item $\delta A = \overline{A} \cap \overline{(X \setminus A)}$, en particular, $\delta A \in C_T$.
    \item $\{A^o,\delta A, ext(A)\}$ es una partición de $X$.
  \end{enumerate}
\end{properties}

\begin{exmp}
  Sea $(\R^n, ||\cdot||)$ un espacio normado en $\R^n$ con la distancia asociada $d(x,y)=||x-y||$. Entonces se tiene que la clausura de la bola abierta es igual a la bola cerrada y el interior de la bola cerrada es igual a la bola abierta. Es decir, $\overline{B(x,r)}=\overline{B}(x,r)$ y $int(\overline{B}(x,r))=B(x,r)$.
\end{exmp}
Para que la propiedad del ejemplo se cumpla, la distancia debe proceder de una norma.
\begin{exmp}
  Sea $(X,d)$ un espacio métrico donde $d$ es la distancia discreta, tal que $\#X \ge 2$. Si tomamos $x \in X$, entonces se cumple que $B(x,1)=\{x\}$ y $\overline{B}(x,1)=X$. Aquí tenemos que $\overline{B(x,1)}$ es el menor conjunto cerrado que contiene a $B(x,1)$, luego $\overline{B(x,1)} \subset \overline{B}(x,1)$. Como $T_d$ es la topología discreta, $\{x\} \in C_{T_d} \implies \overline{\{x\}} = \{x\} \implies \overline{B(x,1)}=\{x\}$. Pero $\overline{B}(x,1)=X \neq \{x\} = \overline{B(x,1)} \implies \overline{B(x,1)} \subsetneq \overline{B}(x,1)$.
\end{exmp}
\begin{properties}
  Si $(X,d)$ es un espacio métrico, $x \in X$ y $r >0$, entonces $int(\overline{B}(x,r))$ es el mayor abierto contenido en $\overline{B}$. Pero $B(x,r) \subset \overline{B}(x,r) \implies B(x,r) \subset int(\overline{B}(x,r))$.
\end{properties}

\begin{ndef}[Separación]
  Diremos que un espacio topológico $(X,T)$ es $T_1$ si $\forall x,y \in X,\ x\neq y,\ \exists V_x \in N_x,\ V_y \in N_y$ tales que $x \not\in V_y,\ y \not\in V_x$.
\end{ndef}
\begin{properties}
  $(X,T)$ es $T_1 \Leftrightarrow$ todo punto de $X$ es cerrado.
\end{properties}
\begin{proof}
  \fbox{$\Rightarrow$} Supongamos que $(X,T)$ es $T_1$. Fijamos $x \in X$. Veamos que $\{x\}$ es cerrado comprobando que $X \setminus \{x\}$ es abierto. Para ello, vemos que $int(X \setminus \{x\})=X \setminus \{x\}$. Sea $y \in X \setminus \{x\} \implies y \neq x$. Como $(X,T)$ es $T_1,\ \exists V_x \in N_x,\ V_y \in N_y$ tales que $x \not\in V_y,\ y \not\in V_x$. $x \not\in V_y \implies \{x\} \cap V_y = \varnothing \implies V_y \subset X \setminus \{x\} \implies y \in int(X \setminus \{x\})$. Hemos probado que  $X \setminus \{x\} \subset int(X \setminus \{x\}) \implies X \setminus \{x\} = int(X \setminus \{x\}) \implies X \setminus \{x\} \in T \implies \{x\} \in C_T$. \\
  \fbox{$\Leftarrow$} Supongamos que todo punto de $X$ es cerrado. Veamos que $(X,T)$ es $T_1$. Para ello tomaremos $x,y \in X,\ x \neq y$. Por hipótesis, $\{x\}$ es cerrado $\implies X \setminus \{x\} \in T$. Como $y \neq x,\ y \in X \setminus \{x\}$ y $X \setminus \{x\}$ es abierto, coincide con su interior $\implies \exists V_y \in N_y$ tal que $V_y \subset X \setminus \{x\} \implies x \in V_y$. Razonando igual para $x$ encontramos $V_x \in N_x$ tal que $y \not\in  V_x$.
\end{proof}

\begin{ndef}
  Diremos que un espacio topológico $(X,T)$ es $T_2$ o \textbf{Hausdorff} si $\forall x,y \in X,\ x \neq y,\ \exists V_x \in N_x,\ V_y \in N_y$ tales que $V_x \cap V_y = \varnothing$.
\end{ndef}
ME FALTA UNA NOTA Y CONSECUENCIAS

\begin{exmp}
  $(\N,T_{CF})$ es $T_1$ pero no es $T_2$. Ya hemos visto en clase que $(\N,T_{CF})$ no es $T_2$. Sin embargo, dicho espacio topológico es $T_1$ porque todo punto es cerrado.
\end{exmp}

\subsection{Axiomas de numerabilidad}
\begin{ndef}
  Diremos que $(X,T)$ verifica el \textbf{el primer axioma de numerabilidad} o que es $AN-I$, si cada punto de $X$ admite una base de entornos numerable.
\end{ndef}

\begin{ndef}
  Diremos que $(X,T)$ verifica el \textbf{el segundo axioma de numerabilidad} o que es $AN-II$  , si $T$ admite una base numerable.
\end{ndef}

\begin{exmp}
  Sea $(X,T_D)$ un espacio topológico discreto. Entonces $\mathcal{B}_x=\{\{x\}\}$ es base de entornos $\implies (X,T_D)$ es $AN-I$. Si $X$ es no numerable, entonces  $(X,T_D)$ no es $AN-II$.\\

  Sea $\mathcal{B}$ una base de $T$. Si $x \in X,\ \{x\} \in T_D \implies \{x\} = \bigcup_{i \in I}B_i$ tal que $B_i \in \mathcal{B} \implies \exists B_{i_0}\ /\ x \in B_{i_0} \subset \bigcup_{i \in I} B_i = \{x\}$. ACABAR
\end{exmp}

\begin{lema}
  Si $\mathcal{B}$ es base de T. Entonces $\mathcal{B}(x) = \{B \in  \mathcal{B}\ /\ x \in B\}$ es base de entornos abiertos de $x$ para todo $x \in X$.
\end{lema}

\begin{ncor}
  Si $(X,T)$ es $AN-II$, entonces es $AN-I$.
\end{ncor}
\begin{proof}
  P
\end{proof}

\begin{ndef}[Subconjunto denso]
  Diremos que un subconjunto $D$ de un espacio topológico $(X,T)$ es denso si $\overline{D}=X$.
\end{ndef}

\begin{ndef}[Separable]
  Diremos que un espacio topológico es separable si contiene un subconjunto denso y numerable.
\end{ndef}


\newpage
\section{Continuidad en Espacios Topológicos}
MIERCOLES JUEVES Y VIERNES
\begin{exmp}
  Sea $(X,T)$ un espacio topológico tal que $A \subset X$, $A \neq \varnothing$. La \textbf{aplicación inclusión} $i_A \to X$ $i_A(a)=a \ \forall a \in A$. $i_A$ es continua, ya que $T_A=U \in T,\ i^{-1}_A(U)=\{a \in A / i_A(a) \in U\}=\{a \in A / a \in U\}=U \cap A \in T_A$.
\end{exmp}

\begin{exmp}
  $f:(X,T) \to (Y,T')$ continua, $A \subset X$, $A \neq \varnothing$. La restricción de $f$ al conjunto $A$ es la aplicación $f_{|A}: (A,T_A) \to(Y,T')$ es continua (si $f$ es continua). Para demostrarlo, vemos que $F_{|A} = f compuesto i_A$, con el mismo dominio de ($A$) y mismo codominio ($Y$). Para $a \in A, \ f_{|A}(a)=f(a)=f(i_A(A))=(f compuesto i_A)(a)$. Si $f:X \to Y$ es continua, como sabemos que $i_A:A \to X$ es continua, entonces la composición $f compuesto i_A:A \to Y$ es continua.
\end{exmp}

\begin{exmp}
  Sea $f: (X,T) \to (Y,T')$ una aplicación. Supongamos que $X = \bigcup_{\alpha \in I}U_{\alpha},\ U_{\alpha} \in T \ \forall \alpha \in I$. $f_{|U_{\alpha}}$ es continua $\forall \alpha \in I \Leftrightarrow f$ es continua.
\end{exmp}
En el anterior ejemplo, no podemos cambiar $U_{\alpha} \in T$ por la hipótesis $U_{\alpha} \in C_T$. Si eso fuera cierto y $f: \R \to \R$, $\R$

\begin{exmp}
  Sea $f:(X,T) \to (Y,T')$ una aplicación. Supongamos que $X=C_1 \cup \ldots \cup C_k,\ k \in \N,\ C_i \in C_T \ \forall i \in \{1,\ldots,k\}$. Entonces $f_{|C_i}$ es continua $\Leftrightarrow f$ es continua. \\
  \fbox{$\Leftarrow$} Cierta siempre. \\
  \fbox{$\Rightarrow$} $f$ continua $\Leftrightarrow f^{-1}(C') \in C_{T'}$. Sea $C' \in C_{T'},\ f^{-1}(C')=f^{-1}(C') \cap X = f^{-1}(C') \cap (\bigcup^k_{i=1}C_i)=\bigcup^k_{i=1} (C_if^{-1}(C')) = \bigcup_{i=1}^kf^{-1}_{|C_i}(C') \in C_{T_{C_i}}$ siGUE.
\end{exmp}


