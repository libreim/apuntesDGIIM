%%%%%%%%%%%%%%%%%%%%%%%%%%%%%%%%%%%%%%%%%%%%%%%%%%%%%%%%%%%%%%%%
%
% Ejercicios de la asignatura Álgebra I.
% Doble Grado de Informática y Matemáticas.
% Universidad de Granada.
% Curso 2016/17.
% 
% 
% Colaboradores:
% Javier Sáez (@fjsaezm)
% Daniel Pozo (@danipozodg)
% Pedro Bonilla (@pedrobn23)
% Guillermo Galindo
% Antonio Coín (@antcc)
%
% Agradecimientos:
% Andrés Herrera (@andreshp) y Mario Román (@M42) por
% las plantillas base.
%
% Sitio original:
% https://github.com/libreim/apuntesDGIIM/
%
% Licencia:
% CC BY 4.0 (https://creativecommons.org/licenses/by/4.0/)
%
%%%%%%%%%%%%%%%%%%%%%%%%%%%%%%%%%%%%%%%%%%%%%%%%%%%%%%%%%%%%%%%


%------------------------------------------------------------------------------
%   ACKNOWLEDGMENTS
%------------------------------------------------------------------------------

%%%%%%%%%%%%%%%%%%%%%%%%%%%%%%%%%%%%%%%%%%%%%%%%%%%%%%%%%%%%%%%%%%%%%%%%
% Plantilla básica de Latex en Español.
%
% Autor: Andrés Herrera Poyatos (https://github.com/andreshp) 
%
% Es una plantilla básica para redactar documentos. Utiliza el paquete  fancyhdr para darle un
% estilo moderno pero serio.
%
% La plantilla se encuentra adaptada al español.
%
%%%%%%%%%%%%%%%%%%%%%%%%%%%%%%%%%%%%%%%%%%%%%%%%%%%%%%%%%%%%%%%%%%%%%%%%%

%%%
% Plantilla de Trabajo
% Modificación de una plantilla de Latex de Frits Wenneker para adaptarla 
% al castellano y a las necesidades de escribir informática y matemáticas.
%
% Editada por: Mario Román
%
% License:
% CC BY-NC-SA 3.0 (http://creativecommons.org/licenses/by-nc-sa/3.0/)
%%%

%%%%%%%%%%%%%%%%%%%%%%%%%%%%%%%%%%%%%%%%
% Short Sectioned Assignment
% LaTeX Template
% Version 1.0 (5/5/12)
%
% This template has been downloaded from:
% http://www.LaTeXTemplates.com
%
% Original author:
% Frits Wenneker (http://www.howtotex.com)
%
% License:
% CC BY-NC-SA 3.0 (http://creativecommons.org/licenses/by-nc-sa/3.0/)
%
%%%%%%%%%%%%%%%%%%%%%%%%%%%%%%%%%%%%%%%%%


% Tipo de documento y opciones.
\documentclass[11pt, a4paper, titlepage]{article}


%---------------------------------------------------------------------------
%   PAQUETES
%---------------------------------------------------------------------------

% Idioma y codificación para Español.
\usepackage[utf8]{inputenc}
\usepackage[spanish, es-tabla, es-lcroman, es-noquoting]{babel}
\selectlanguage{spanish} 
%\usepackage[T1]{fontenc}

% Fuente utilizada.
\usepackage{courier}    % Fuente Courier.
\usepackage{microtype}  % Mejora la letra final de cara al lector.

% Diseño de página.
\usepackage{fancyhdr}   % Utilizado para hacer títulos propios.
\usepackage{lastpage}   % Referencia a la última página.
\usepackage{extramarks} % Marcas extras. Utilizado en pie de página y cabecera.
\usepackage[parfill]{parskip}    % Crea una nueva línea entre párrafos.
\usepackage{geometry}            % Geometría de las páginas.

% Símbolos y matemáticas.
\usepackage{amssymb, amsmath, amsthm, amsfonts, amscd}
\usepackage{upgreek}

% Otros.
\usepackage{enumitem}   % Listas mejoradas.
\usepackage[hidelinks]{hyperref}
\usepackage{float}

\usepackage{titlesec}
%---------------------------------------------------------------------------
%   OPCIONES PERSONALIZADAS
%---------------------------------------------------------------------------

% Redefinir letra griega épsilon.
\let\epsilon\upvarepsilon

% Formato de texto.
\linespread{1.1}            % Espaciado entre líneas.
\setlength\parindent{0pt}   % No indentar el texto por defecto.
\setlist{leftmargin=.5in}   % Indentación para las listas.

% Estilo de página.
\pagestyle{fancy}
\fancyhf{}
\geometry{left=3cm,right=3cm,top=3cm,bottom=3cm,headheight=1cm,headsep=0.5cm}   % Márgenes y cabecera.

% Redefinir entorno de demostración (reducir espacio superior)
\makeatletter
\renewenvironment{proof}[1][\proofname] {\vspace{-15pt}\par\pushQED{\qed}\normalfont\topsep6\p@\@plus6\p@\relax\trivlist\item[\hskip\labelsep\it#1\@addpunct{.}]\ignorespaces}{\popQED\endtrivlist\@endpefalse}
\makeatother

% Cambiar sections 


%---------------------------------------------------------------------------
%   COMANDOS PERSONALIZADOS
%---------------------------------------------------------------------------

% Números enteros: \ent
\providecommand{\ent}{\mathbb{Z}}

% Números racionales: \rac
\providecommand{\rac}{\mathbb{Q}}

% Números naturales: \nat
\providecommand{\nat}{\mathbb{N}}


% Valor absoluto: \abs{}
\providecommand{\abs}[1]{\lvert#1\rvert}    

% Fracción grande: \ddfrac{}{}
\newcommand\ddfrac[2]{\frac{\displaystyle #1}{\displaystyle #2}}

% Texto en negrita en modo matemática: \bm{}
\newcommand{\bm}[1]{\boldsymbol{#1}}

% Línea horizontal.
\newcommand{\horrule}[1]{\rule{\linewidth}{#1}}


%---------------------------------------------------------------------------
%   CABECERA Y PIE DE PÁGINA
%---------------------------------------------------------------------------

% Cabecera del documento.
\renewcommand\headrule{
	\begin{minipage}{1\textwidth}
		\hrule width \hsize 
	\end{minipage}
}

% Texto de la cabecera.
\lhead{\subject}  % Izquierda.
\chead{}            % Centro.
\rhead{\docauthor}    % Derecha.

% Pie de página del documento.
\renewcommand\footrule{                                 
	\begin{minipage}{1\textwidth}
		\hrule width \hsize   
	\end{minipage}\par
}

% Texto del pie de página.
\lfoot{}                                                 % Izquierda
\cfoot{}                                                 % Centro.
\rfoot{Página\ \thepage\ de\ \protect\pageref{LastPage}} % Derecha.


%---------------------------------------------------------------------------
%   ENTORNOS PARA MATEMÁTICAS
%---------------------------------------------------------------------------

% Listas ordenadas con números romanos (i), (ii), etc.
\newenvironment{nlist}
{\begin{enumerate}
\renewcommand\labelenumi{(\emph{\roman{enumi})}}}
{\end{enumerate}}

% División por casos con llave a la derecha.
\newenvironment{rcases}
  {\left.\begin{aligned}}
  {\end{aligned}\right\rbrace}



%---------------------------------------------------------------------------
%   PÁGINA DE TÍTULO
%---------------------------------------------------------------------------

% Título del documento.
\newcommand{\subject}{Ejercicios resueltos Álgebra I}

% Autor del documento.
\newcommand{\docauthor}{Doble Grado de Informática y Matemáticas}

% Título
\title{
  \normalfont \normalsize 
  \textsc{Universidad de Granada} \\ [25pt]    % Texto por encima.
  \horrule{0.5pt} \\[0.4cm] % Línea horizontal fina.
  \huge \subject\\ % Título.
  \horrule{2pt} \\[0.5cm] % Línea horizontal gruesa.
}

% Autor.
\author{\Large{\docauthor}}

% Fecha.
\date{\vspace{-1.5em} \normalsize Curso 2016/17}


%---------------------------------------------------------------------------
%   COMIENZO DEL DOCUMENTO
%---------------------------------------------------------------------------
\begin{document}
\maketitle

\section{\Huge{Relación 2}}

\subsection{\LARGE{Ejercicio 3 (2ª parte de la relación)}}

\textbf{Determinar los polinomios $f(x) \in \rac [x]$ de grado menor o igual que tres que satisfacen el sistema de congruencias}

\[
\begin{rcases}
	f(x) \equiv x - 1\ mod(x^2+1)\\
	f(x) \equiv x + 1\ mod(x^2+x+1)
\end{rcases}
\]

En primer lugar, reescribimos la primera ecuación para sustituirla en la segunda. 
$$f(x) \equiv x - 1\ mod(x^2+1) \implies f(x) = x-1 + (x^2+1)g(x)^\circledast$$

Reemplazando lo obtenido, tenemos:
$$x-1 + (x^2+1)g(x) \equiv x+1\ mod(x^2+x+1)$$ 
Pasamos el x-1 restando:
$$(x^2+1)g(x) \equiv 2\ mod(x^2+x+1)$$
Resolvemos esta ecuación, que tendrá solución si, y solo si, $(x^2+1,\ x^2+x+1)/2$. Así, hallamos el mcd a través de la tablita correspondiente:


\begin{center}
\begin{tabular}{c|cc}
$x^2+x+1$ & 1 & 0 \\
$x^2+1$ & 0 & 1 \\
$x$ & 1 & -1 \\
1 & $-x$ & $x+1$\\
0
\end{tabular}
\end{center}

Obtenemos de esta forma que $(x^2+1,\ x^2+x+1) = 1$, que divide a 2, por tanto, habrá solución. Partiendo de la identidad de Bezout: $1 = (x^2+x+1)(-x) + (x^2+1)(x+1)$, la transformamos en una ecuación en congruencia (si vemos la ecuación como $(x^2+x+1)(x) = (x^2+1)(x+1) - 1$, por la definición de congruencia) $(x^2+1)(x+1) \equiv 1\ mod(x^2+x+1)$, multiplicando por 2, encontramos la g(x) buscada: $(x^2+1)(2x+2) \equiv 2\ mod(x^2+x+1)$.
$g(x)=2x+2$.

Sustituyendo en $^\circledast$ el polinomio g(x) recién encontrado llegaremos a $f_0$(x), solución parcial del sistema. 

$$f_0(x) = x-1 + (x^2+1)g(x) = x-1 + (x^2+1)(2x+2) = 2x^3+2x^2+3x+1$$

La solución general será $$f(x) \equiv f_0(x) mod[x^2+1,\ x^2+x+1]$$
Calculamos el mcm:
\[
	[x^2+1,\ x^2+x+1] = \frac{(x^2+x+1)(x^2+1)}{1} =x^4+x^3+x^2+x+1
\]

Como el ejercicio pide aquellos polinomios de grado menor o igual que tres, nos basta con la solución parcial, ya que otras soluciones eran polinomios de grado superior al buscado.

\textbf{Solución:} $f_0(x) = 2x^3+2x^2+3x+1$


\section{\LARGE{Ejercicio 6}}\textbf{
En el anillo $\mathbb{Z}$[i], resolver el siguiente sistema de congruencias:} 

\[
	\begin{cases}
	x \equiv i\ \ \ \ \ \ \ \ mod\ (3)\\
	x \equiv 1+i \ \ \ mod\ (3+2i)\\
	x \equiv 3+2i \ \ mod\ (4+i)
\end{cases}
\]

\emph{Resolución. \\ }

Empezaremos resolviendo el sistema: 
\[
	\begin{cases}
	x \equiv i\ \ \ \ \ \ \ \ mod\ (3)\\
	x \equiv 1+i \ \ \ mod\ (3+2i)\\
\end{cases}
\]

 Y para ello hallaremos la solución particular de $x \equiv i\ mod\ (3)$. Como i es una unidad del anillo, entonces $\forall a \in \mathbb{Z}[i] \Rightarrow (a,i) = i.$ Y los coeficientes de bezout son $0*3 + 1*i = 1$ de manera trivial. Entonces la solución general de la primera ecuación sería $x = i + 3*k$. \\
 
 Ahora sustituimos x en la segunda ecuación, y nos queda la ecuación $i + 3k \equiv\ 1 +i\ mod (3+2i) $ de manera equivalente $3k\ \equiv\ 1\ mod\ (3+2i)$. Ahora sacaremos los coeficientes de bezout de 3 y 3+2i. Mira abajo. Entonces por ello sabemos que $3*(-1-i) \equiv -i\ mod (3+2i) $. Por ello una solución particular será $k=(-1-i)*-i=(i-1)$. Y la solución general para $k$ será por lo tanto $k = (i-1) + (3+2i)*k'$. \\
 
 %jalp! nid jalp! como coloco esto bien???
  \begin{table}[]
\centering
\caption{coeficientes de bezout}
\label{my-label}
\begin{tabular}{lllll}
     & 3+2i & 3    &  &  \\
3+2i & 1    & 0    &  &  \\
3    & 0    & 1    &  &  \\
-i   & 1    & -1-i &  & 
\end{tabular}
\end{table}
%hasta aquí 
 
Sustituimos la particular de k en la primera resolción y hayariamos $M = [3,3+2i]$ para hayar cada cuanto debemos hacer la repetición. Para calcular el m.c.m recordaremos que $(a,b)*[a,b]=ab \Rightarrow \frac{ab}{(a,b)} = [a,b].$ Entonces para nuestro caso particular $[3,3+2i] = \frac{9+6i}{-i}=9i-6$. Así pues la colución será:

$$x = i + 3 (i-1) + k''(9i-6) = 4i-3+ k''(9i-6) $$
Ahora cogemos el sistema:
\[
	\begin{cases}
	x \equiv 4i-3 \ \ mod\ (9i-6) \\
	x \equiv 3+2i \ \ mod\ (4+i)
\end{cases}
\]

 Para resolverlo y hayar (por fin) la solución final haremos lo mismo: hayar la solución general de la primera ecuación (ya resuelta) y sustituir en la segunda, m.c.m de los módulos, y terminamos.\\
 
 Solución primera ecuación: $4i-3+ k''(9i-6)$.\\
 
 Sustituimos segunda: $4i-3+ k''(9i-6) \equiv 3+2i \ \ mod\ (4+i) \rightarrow 
 (9i-6)k''\equiv 6-2i \ \ mod\ (4+i)$ \\
 
 Hayamos coeficientes de bezout y M.C.D. (tabla adjunta arriba):  \\ 
 
 Enunciamos solución particular y un indicio de la general: $Como\ (9i-6)(-2) \equiv i\ mod\ (4i-1) \Rightarrow $ 
 $(9i-6)(-2)(-6i-2) \equiv i(-6i-2)\ mod\ (4i-1) \Rightarrow (9i-6)(12i+4) \equiv (6-2i)\ mod\ (4i-1) \Rightarrow k'' = (12i+4) + [9i-6,4+i] k''' $\\
 
 Calculamos [9i-6,4+i]: $\frac{30i-33)}{i}=30-33i$\\ 
 
 Solución general: $(12i+4) + (30-33i) k'''$
\begin{table}[]
\centering
\caption{coficientes de bezout 2}
\label{my-label}
\begin{tabular}{lll}
     & 9i-6 & 4+i  \\
9i-6 & 1    & 0    \\
4+i  & 0    & 1    \\
2i   & 1    & 1-2i \\
i    & -2   & 4i-1
\end{tabular}
\end{table}
 

 
\end{document}

\end{document}
