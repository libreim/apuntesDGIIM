\section{\Huge{Relación 1}}

\subsection{\LARGE{Ejercicio 3}}

\textbf{¿Cuál de los siguientes conjuntos son subanillos de los anillos indicados?}

\begin{enumerate}
  \item ${a \in \mathbb{Q} | 3a \in \mathbb{Z}} \subseteq \mathbb{Q},$
  \item ${m + 2n\sqrt{3} | m, n \in \mathbb{Z}} \subseteq \mathbb{R},$
  \item $ {f(x) = \sum a_ix^i \in \mathbb{Z}[x] | a_1 es múltiplo de 2} \subseteq \mathbb{Z}[x],$
  \item $ {f(x) = \sum a_ix^1 \in \mathbb{Z}[x] | 2 / a_2} \subseteq \mathbb{Z}[x].$
\end{enumerate}

Comenzamos diciendo que para que un subconjunto B sea un subanillo de un anillo A se tiene que cumplir que:

\begin{itemize}
  \item $1, -1 \in B$ (contienen al elemento neutro para el producto y su opuesto).
  \item B es cerrado para la suma y el producto.
\end{itemize}

Pasemos ahora a analizar los casos descritos arriba.

\textit{i.} ${a \in \mathbb{Q} | 3a \in \mathbb{Z}} \subseteq \mathbb{Q}$

Este este subconjunto no es un anillo ya que $1, -1 \in \mathbb{Q}$ no están en él, ya que no son múltiplos de 3.

\textit{ii.} ${m + 2n\sqrt{3} | m, n \in \mathbb{Z}} \subseteq \mathbb{R}$

Con $n = 0$ y $m = 1, -1$ tenemos $1$ y $-1$ de $\mathbb{R}$. Además, sean cual sean $n, m$ de \mathbb{Z} este subconjunto es cerrado para la suma y el producto, dado que la suma de enteros es siempre un entero y lo mismo ocurre con el producto.

\textit{iii.} $ {f(x) = \sum a_ix^i \in \mathbb{Z}[x] | a_1 es múltiplo de 2} \subseteq \mathbb{Z}[x]$

Con $a_0 = 1, -1$ y $a_i = 0, i > 0$ tenemos el elemento neutro del producto y su opuesto. Además, sean cuales sean los $a_i$ este subconjunto es cerrado para la suma y el producto ya que para los $a_i, i \neq 1$ la suma y producto de números enteros es siempre un número entero y en el caso del $a_1$, la suma y producto de múltiplos de 2 es siempre un múltiplo de 2.

\textit{iv.} $ {f(x) = \sum a_ix^1 \in \mathbb{Z}[x] | 2 / a_2} \subseteq \mathbb{Z}[x]$

Este subconjunto no es un anillo ya que $1, -1 \in \mathbb{Z}[x]$ no están en él ya que todos sus elementos son monomios de la forma $a_ix^1$. 
