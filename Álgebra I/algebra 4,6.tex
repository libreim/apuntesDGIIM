
\documentclass[13pt]{article}
\usepackage[utf8]{inputenc}
\usepackage[margin=3cm]{geometry}
\usepackage{enumerate}
\usepackage{amsmath}
\usepackage{float}





\begin{document}
\textbf{Ejercicio 4:} Una banda de 13 piratas se reparten N monedas de oro, pero le sobran 8. Dos mueren, las vuelven a repartir y sobran 3. Luego se ahogan 3 y sobran 5. ¿Cuál es la mínima cantidad posible N de monedas?\\

\[
	\begin{cases}
	\text{N} \equiv 8\; \text{mod}(13)\\
	\text{N} \equiv 3\; \text{mod}(11)\\
	\text{N} \equiv 5\; \text{mod} (8)\\
	\end{cases}
\]

Resolvemos el sistema formado por la primera y la tercera ecuación:

N $\equiv$ 8 mod(13) $\rightarrow$ N = 8 + 13 $\cdot$ x

Sustituimos en la tercera ecuación

8 + 13 $\cdot$ x $\equiv$ 5 mod(8) $\rightarrow$ 5 $\cdot$ x $\equiv$ 5 mod(8) $\Rightarrow$ x $\equiv$ 1 mod(8) (Podemos simplificar el 5 por qué $\hspace*{0.5cm}$(5,8) = 1)

$x_{0}$ = 1 $\Rightarrow$ $N_{0}$ 8 + 13 $\cdot$ 1 = 21 (Solución óptima)

N $\equiv$ 21 mod(8 $\cdot$ 13) = 21 mod(104)\\

Ahora, resolvemos el sistema con la ecuación anterior y la segunda del sistema inicial

N $\equiv$ 21 mod(104) $\rightarrow$ 21 + 104 $\cdot$x

21 + 104 $\cdot$ x $\equiv$ 3 mod(11) $\Rightarrow$ 10 + 5 $\cdot$ x $\equiv$ 3 mod(11) $\Rightarrow$ 5 $\cdot$ x $\equiv$ -7 mod(11) $\Rightarrow$ 5 $\cdot$ x $\equiv$ 4 mod(11)\\

\begin{figure}[H]
\begin{center}
\caption{mcd}
\label{my-label}
\begin{tabular}[(b)]{|c|cc}
\cline{1-1}
\textbf{11} & 1                    & 0                    \\ \cline{1-1}
\textbf{5}  & 0                    & 1                    \\ \cline{1-1}
\textbf{1}           & 1                    & -2                   \\ \cline{1-1}
\textbf{0}           & \multicolumn{1}{l}{} & \multicolumn{1}{l}{} \\ \cline{1-1}
\end{tabular}
\end{center}
\end{figure}

1 = 1 $\cdot$ 11  + 5 $\cdot$ (-2) $\rightarrow$ 5 $\cdot$ (-2) $\equiv$1 mod(11)

Multiplicamos por 4 y así obtendremos una solución particular

5 $\cdot$ (-8) $\equiv$ 4 mod(11) $\Rightarrow$ $x_{0}$ = -8

Sustituimos y nos queda $N_{0}$ = -811 

N $\equiv$ -811 mod(1144) = 333 mod(1144) (donde 1144 = [11,104])

N = 333 + 1144 $\cdot$ k\\

Solución: La cantidad mínima de monedas sería cunado k vale 0. Entonces el número de monedas $\hspace*{0.5cm}$es 333\\



\textbf{Ejercicio 6:} Antonio, Pepe y Juan son tres campesinos que principalmente se dedican al cultivo de la aceituna. Este año la producción de los olivos de Antonio fue tres veces la de los de Juan y la de Pepe cinco veces la de los de Juan. Los molinos a los que estos campesinos llevan la aceituna, usan recipientes de 25 litros el de Juan, 7 litros el de Antonio y 16 litros el de Pepe. Al envasar el aceite producido por los olivos de Juan sobraron 21 litros, al envasar el producido por Antonio sobraron 3 litros y al envasar el producido por Pepe sobraron 11 litros. Sabiendo que la producción de Juan está entre 1000 y 2000 litros ¿cual fue la producción de cada uno de ellos?.\\

A $\rightarrow$ producción de Antonio 

P $\rightarrow$ producción de Pepe 

J $\rightarrow$ producción de Juan \\

J  $\equiv$ 21 mod (25) 

A $\equiv$ 3 mod(7) $\Rightarrow$ 3J $\equiv$ 3 mod(7) $\Rightarrow$ J $\equiv$ 1 mod(7) (Aquí podemos simplificar el 3 debido a que es \hspace*{0.5cm}primo con 7)

P $\equiv$ 11 mod (16) $\Rightarrow$ 5J $\equiv$ 11 mod (16) 


\begin{figure}[H]
\begin{center}
\caption{mcd}
\label{my-label}
\begin{tabular}[(b)]{|c|cc}
\cline{1-1}
\textbf{16} & 1                    & 0                    \\ \cline{1-1}
\textbf{5}  & 0                    & 1                    \\ \cline{1-1}
\textbf{1}           & 1                    & -3                   \\ \cline{1-1}
\textbf{0}           & \multicolumn{1}{l}{} & \multicolumn{1}{l}{} \\ \cline{1-1}
\end{tabular}
\end{center}
\end{figure}


1 = 1 $\cdot$1 6 + 5 $\cdot$ (-3) $\Rightarrow$ 5 $\cdot$ (-3) = 1 mod(16) $\rightarrow$ 5 $\cdot$ ((-3) $\cdot$ 11) $\equiv$ 1 mod(16)\\
\hspace*{0.5cm}J $\equiv$ 15 mod(10)
\\

Así nos queda e sistema de ecuaciones de congruencia
\[
	\begin{cases}
	\text{J} \equiv 21\; \text{mod}(25)\\
	\text{J} \equiv 1\; \text{mod}(7)\\
	\text{J} \equiv 15\; \text{mod}(16)\\
\end{cases}
\]
\\

Resolvemos el sistema formado por las dos primeras ecuaciones

J $\equiv$ 21 mod(25) $\Rightarrow$ J = 21 + 25y

Sustituimos en la segunda ecuación

21 + 25y $\equiv$ 1 mod(7) $\Rightarrow$ 4y $\equiv$ 1 mod(7) $\Rightarrow$ y $\equiv$ 2 mod(7) $\Rightarrow$ $y_{0}$ = 2

$J_{0}$ = 21 + 25 $\cdot$ 2 = 71  (solución óptima)

J $\equiv$ 71 mod(175)  (donde 175 es el mcm entre 7 y 25)
\\

Resolvemos el sistema formado por la tercera ecuación y el resultado anterior

J = 175y + 71

Sustituimos en la segunda tercera ecuación

175y + 71 $\equiv$ 15 mod(16) $\Rightarrow$ 175y $\equiv$ 56 mod(16) $\Rightarrow$ 15y $\equiv$ 8 mod(16)



\begin{figure}[H]
\begin{center}
\caption{mcd}
\label{my-label}
\begin{tabular}[(b)]{|c|cc}
\cline{1-1}
\textbf{16} & 1                    & 0                    \\ \cline{1-1}
\textbf{15}  & 0                    & 1                    \\ \cline{1-1}
\textbf{1}           & 1                    & -1                   \\ \cline{1-1}
\textbf{0}           & \multicolumn{1}{l}{} & \multicolumn{1}{l}{} \\ \cline{1-1}
\end{tabular}
\end{center}
\end{figure}

1 = 16 + (-1) $\cdot$ 15

1 = 16 + (-1)$\cdot$15 $\Rightarrow$ 15$\cdot$(-1) = 1 mod (16) $\rightarrow$ (-1) $\cdot$ (-1) $\equiv$ 1 mod(16) $\Rightarrow$ 1 $\equiv$ 1 mod(16)

Multiplicamos por 8, el resto, para encontrar una solución particular:

8 $\equiv$ 8 mod(16) $\Rightarrow$ $y_{0}$ = 8

$J_{0}$ = 71 + 8 $\cdot$ 175 = 1471 (solución óptima)

J = 1471 mod(16 $\cdot$ 175)= 1471 mod(2800)
\\

Los litros de la poducción anuales tienen que estar entre 1000 y 2000

1000 \textless $\;$ 1471 + 2800 $\cdot$ k \textless $\;$ 2000 $\Rightarrow$ $\;$ k = 0
\\

Solución: La producción anual de los campesinos fue:

Juan = 1471 litros

Antonio = 1471 $\cdot$ 3 = 4413

Pepe = 1471 $\cdot$ 5 = 7355


\end{document}