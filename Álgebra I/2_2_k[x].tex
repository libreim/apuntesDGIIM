\documentclass[10pt,a4paper]{article}
\usepackage[latin1]{inputenc}
\usepackage{amsmath}
\usepackage{amsfonts}
\usepackage{amssymb}
\usepackage{graphicx}
\usepackage[left=3.00cm, right=3.00cm, top=3.00cm, bottom=3.00cm]{geometry}

\begin{document}
	\begin{center}
		\textbf{Relaci�n 2 k[x]}
	\end{center}
	
	\textbf{2. Calcular el m�ximo com�n divisor y el m�nimo com�n m�ltiplo, en el anillo $Z_{3}$[x], de los polinomios $x^{4}+x^{3}-x-1$ y $x^{5}+x^{4} -x-1$. Encontrar todos los polinomios f (x) y g(x) en $Z_{3}$[x], con grado de g(x) igual a 7, tales que $(x^{4}+x^{3}-x-1)$ f(x)+$(x^{5}+x^{4} -x-1)$ g(x) = $x^{4}+x^{2}+1$ \\ }
	
	
     \underline{M�ximo com�n divisor} \\
     
     
     \noindent{$x^{5}+x^{4} -x-1$  $ | $  \hspace{0.5cm} 1 \hspace{0.5cm}0 \\}
     $x^{4}+x^{3}-x-1$ $ | $  \hspace{0.5cm} 0 \hspace{0.5cm}1 \\
     $x^{2}-1 $ \hspace{1.4cm}$ | $  \hspace{0.5cm} 1 \hspace{0.5cm}-x \\
     0
     
     Las divisiones realizadas han sido : \\
     
     $\bullet$ ($x^{5}+x^{4} -x-1$)/($x^{4}+x^{3}-x-1$) \\
     
     \noindent{Cociente: x} \\
     Resto:  $x^{2}-1 $\\
     
     $\bullet$ ($x^{4}+x^{3}-x-1$)/($x^{2}-1 $) \\
     
      \noindent{Cociente: $x^{2}+x+1$ } \\
      Resto:  0\\
      
      As�, el m�ximo com�n divisor es $x^{2}-1$ \\
     
     \underline{M�nimo com�n m�ltiplo} \\
     
     Usamos que [a,b]=$\displaystyle{\frac { ab }{ (a,b) } }$ \\
     
     ($x^{5}+x^{4} -x-1$)($x^{4}+x^{3}-x-1$) = $x^{9}+2x^{8}+x^{7}-x^{6}-x^{3}+x^{2}+2x+1$ \\
     
     ($x^{9}+2x^{8}+x^{7}-x^{6}-x^{3}+x^{2}+2x+1$)/($x^{2}-1$) = $x^{7}+2x^{6}+x^{4}+x^{2}-x+2$ \\
     
     Por lo tanto, el m�nimo com�n m�ltiplo es $x^{7}+2x^{6}+x^{4}+x^{2}-x+2$ \\
     
     \underline{Ecuaci�n diof�ntica} \\
     
     Dividiendo por el m�ximo com�n divisor la ecuaci�n para obtener al reducida (vemos que tiene soluci�n) queda: \\
     
     $(x^{2}+x+1)$ f(x)+$(x^{3}+x^{2}+x+1)$ g(x) = $x^{2}+2$ \\
     
     Aplicando la igualdad de Bezout (a partir de los c�lculos del m�ximo com�n divisor): \\
     
     $(x^{2}+x+1)(-x)$+$(x^{3}+x^{2}+x+1)$ (1) = 1 \\
     
     $(x^{2}+x+1)$$(-x^{3}-2x)$+$(x^{3}+x^{2}+x+1)$ $(x^{2}+2)$ = $x^{2}+2$ \\
     
     Obtenemos como soluci�n particular: \\
     
     $f_{0}$(x) = $-x^{3}-2x$ \\
     
     $g_{0}$(x) = $x^{2}+2$ \\
     
     La soluci�n general quedar�a: \\
     
     $f$(x) = $-x^{3}-2x +k(x)(x^{3}+x^{2}+x+1)$  \\
     
     $g$(x) = $x^{2}+2 -k(x)(x^{2}+x+1)$ \\
     
     Para cumplir la condici�n  g(x) igual a 7 observamos que el grado de k(x) tiene que ser igual a 5. De este modo, todas soluciones pedidas son las que se obtienen a partir de la general para todos los k(x) de $Z_{3}$[x] tal que gr(k(x)) = 5. 
     
     
     
     
	
\end{document}