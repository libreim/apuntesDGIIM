\documentclass[11pt]{article}

\usepackage[utf8]{inputenc}
\usepackage[spanish]{babel}
\usepackage[left=1.7cm,top=2cm,right=1.7cm,bottom=2.5cm]{geometry}
\usepackage{ upgreek }
\usepackage{ amssymb }
\usepackage{lmodern}
\usepackage[T1]{fontenc}


\title{\textbf{Algunos conceptos de Análisis Matemático I}}
\date{}
\begin{document}

\maketitle



\section*{Tema 1}

Debemos definir primeramente cómo es la distancia usual en $\mathbb{R}^n$:
\[
x,y \in \mathbb{R}^n ; d(x,y) = \sqrt{\sum_{i=1}^n (y_i - x_i)^2}
\]
Ahora, basándonos en esa distancia, daremos una primera noción de una función continua en $\mathbb{R}^n$


\subsection*{Continuidad en $\mathbb{R}^n$}
Sea $A \subset R, A \neq \emptyset$. Sea $f: A \to \mathbb{R}^m$. Decimos que $f$ es continua en $a \in A; a = (a_1,...,a_n)$ si:

\[
\forall \upvarepsilon > 0  \hspace{0.3cm}\exists \delta > 0 :\hspace{0.3cm} d_{\mathbb{R}^n}(x,a)< \delta \hspace{0.2cm} ,\hspace{0.2cm} x \in A\hspace{0.3cm} \Rightarrow \hspace{0.3cm} d_{\mathbb{R}^m}(f(x),f(a))< \upvarepsilon
\]


\subsection*{Bola abierta}

Sea $x\in \mathbb{R}^n$, $\upvarepsilon > 0$. Se define entonces la bola abierta de centro $x$ y radio $\upvarepsilon$ como:
\[
B(x,\upvarepsilon) = \{ y \in \mathbb{R}^n : d(x,y) < \upvarepsilon\}
\]



\subsection*{Bola cerrada}

Sea $x\in \mathbb{R}^n$, $\upvarepsilon > 0$. Se define entonces la bola cerrada de centro $x$ y radio $\upvarepsilon$ como:
\[
\bar{B}(x,\upvarepsilon) = \{ y \in \mathbb{R}^n : d(x,y) \leq \upvarepsilon\}
\]



\subsection*{Abierto}
Si $A\subset \mathbb{R}^n$, con la topología usual en $\mathbb{R}^n$ podemos afirmar:

\[
A \hspace{0.2cm} es \hspace{0.2cm}abierto \hspace{0.2cm}\Leftrightarrow \hspace{0.2cm} \forall a \in A, \exists \upvarepsilon > 0 : B(a,\upvarepsilon)\subset A
\]



\subsubsection*{Proposición}

En un espacio métrico $(X,d)$
\begin{enumerate}
	\item Si $\{ A_\lambda : \lambda \in \Lambda\}$ es una familia de subconjuntos con $A_\lambda \in X$. Si $A_\lambda$ es abierto $\forall \lambda \in \Lambda$ entonces $\bigcup_{\lambda \in \Lambda}A_\lambda$ es abierto

	\item Si $\{A_1,...,A_n\}$ es una familia finita de abiertos con $A_i \in X $ entonces $\bigcap_{i=1}^n A_i$ es un abierto en $X$

	\item $X, \emptyset$ son abiertos

\end{enumerate}

Esto define una topología en $\mathbb{R}^n$.

\subsection*{Punto Interior}

Sea $A\subset X, a \in A$. Decimos que $a$ es un punto interior de $A$ sii:
\[
\exists \upvarepsilon_0 > 0: B(a,\upvarepsilon_0) \subset A
\]
Definimos el conjunto de los puntos interiores de A como:
\[
Aº = \{ a\in A: a \hspace{0.2cm} es \hspace{0.2cm} interior\hspace{0.2cm}de \hspace{0.2cm}A \} = int(A)
\]



\subsection*{Conjunto cerrado}

Sea $F\subset X$. Decimos que F es cerrado $\Leftrightarrow$ $X-F$ es abierto.
Equivalentemente, podemos decir que existe $int(X-A)$, el interior del complementario.



\subsection*{Cierre de un $A \subset X$}

Sea $A \subset X$, $(X,d)$ un espacio métrico. Llamamos cierre de A a:

\[
\bar{A} = X \hspace{0.1cm}\textbackslash \hspace{0.1cm} int(X-A)
\]

Es fácil ver que $\bar{A}$ es cerrado y que otra forma de definirlo es:
\[
\bar{A} = \bigcap \{F\in X: F \hspace{0.2cm} cerrado \hspace{0.2cm}y \hspace{0.2cm} A\subset F\}
\]
Por tanto, El cierre de A es el conjunto cerrado más pequeño que contiene a todos los cerrados que contienen al conjunto. De esta forma, $A\subset X$ es cerrado $\Leftrightarrow$ $A = \bar{A}$



\subsection*{Frontera de un conjunto}

Sea $A\subset X$. Llamaremos frontera de A ($\partial A$)
\[
\partial A = \bar{A} \backslash Aº
\]

Es decir, al cierre de A sin los puntos interiores de A.


\subsection*{Punto de acumulación}

Sea $A \subset X, x \in A$

Se dice que $x$ es un punto de acumulación de A sii:

\[
\forall \upvarepsilon > 0 \hspace{1cm} B(x,\upvarepsilon)\bigcap(A\backslash \{x\}) \neq \emptyset
\]



\end{document}
