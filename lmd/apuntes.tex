\section{Inducción y Recurrencia}
\subsection{Axiomática de Peano}
Supongamos que existe un conjunto $\mathbb{N}$. Los elementos de este conjunto se llaman números naturales.
\begin{ndef}[Axiomas de Peano]
Los axiomas que definen a $\nat$ son los siguientes:
\begin{enumerate}[label=\emph{A\arabic*}]
\item\label{a1} El cero es un número natural. $0 \in \mathbb{N}$
\item\label{a2} El siguiente de un número natural es un número natural. Si $n \in \mathbb{N} \Rightarrow \sigma(n) \in \mathbb{N}$
\item\label{a3} Cero no es el siguiente de ningún número natural. $\forall n \in \mathbb{N}$, $\sigma(n) \neq 0$
\item\label{a4} Si los siguientes de dos números naturales son iguales, entonces los números naturales son iguales. $\forall m,n \in \mathbb{N}, \sigma(n) = \sigma(m) \Rightarrow m = n$
\item\label{a5} Si un subconjunto de números naturales tiene el cero y siempre que tiene un número tiene a su siguiente, entonces el subconjunto son todos los números naturales.
\end{enumerate}
\end{ndef}

\begin{nth}
Todo número natural es distinto del siguiente. $\forall n \in \nat n \neq \sigma(n)$
\end{nth}
\begin{proof}
Sea $A = \{x \in \nat : x \neq \sigma(x)\}$: \\
Como $0 \neq \sigma(0)$, resulta $0 \in A$.
Supongamos ahora $n \in A$, es decir, $n \neq \sigma(n)$, luego $\sigma(n) \neq \sigma(\sigma(n))$, por tanto, $\sigma(n) \in A$.
Luego $A = \nat$.
\end{proof}

\begin{nth}
Para cada número natural distinto de cero, existe un único número natural del que es su siguiente. $\forall n \in \nat (n \neq 0 \Rightarrow \exists! m \in \nat$ tal que $x = \sigma(m))$
\end{nth}
\begin{proof}
Sea $A = \{x \in \nat : x = 0$ o $m \in \nat$ tal que $x = \sigma(m)\}$:\\
Como $0 = 0$, resulta $0 \in A$. Supongamos ahora $n \in A$, es decir, $n = 0$ o $n = \sigma(m)$. En cualquier caso, $\sigma(n) = \sigma(n)$, por tanto $\sigma(n) \in A$. Luego $A = \nat$. \\
La unicidad es consecuencia de $A4$.
\end{proof}

\subsection{Aritmética natural}
\subsubsection{Suma de naturales}
\begin{nth}
Existe una única $+ : \nat \times \nat \rightarrow \nat$ tal que $\forall m,n \in \nat$ verifica:
\begin{itemize}
\item $m + 0 = m$
\item $m + \sigma(n) = \sigma(m + n)$
\end{itemize}
\end{nth}
\smallskip
\noindent
\begin{properties}
Para todo $m,n,p \in \nat$ se cumple:
\begin{enumerate}
\item Todo número natural es 0 o es el siguiente de un número natural.
\item $m + 0 = 0 + m = m$.
\item $m + 1 = 1 + m = \sigma(m)$.
\item $(m + n) + p = m + (n + p)$.
\item $m + n = n + m$.
\item Si $m + p = n + p$, entonces $m = n$.
\item Si $m + n = 0$, entonces $m = n = 0$.
\end{enumerate}
\end{properties}

\subsubsection{Producto de naturales}
\begin{nth}
Existe una única $\cdot : \nat \times \nat \rightarrow \nat$ tal que $\forall m,n \in \nat$ verifica:
\begin{itemize}
\item $m \cdot 0 = 0$
\item $m \cdot \sigma(n) =  m \cdot n + m$
\end{itemize}
\end{nth}